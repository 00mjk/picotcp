It is assumed that all examples include the appropriate header files
and a \textbf{main} routine that calls the \texttt{app$\_$x} functions to initialize
the example.

The most common header files are:
\begin{verbatim}
#include "pico_stack.h"
#include "pico_config.h"
#include "pico_dev_vde.h"
#include "pico_ipv4.h"
#include "pico_socket.h"
#include "pico_dev_tun.h"
#include "pico_nat.h"
#include "pico_icmp4.h"
#include "pico_dns_client.h"
#include "pico_dev_loop.h"
#include "pico_dhcp_client.h"
#include "pico_dhcp_server.h"
#include "pico_ipfilter.h"
\end{verbatim}

\section{Ping example}

\begin{verbatim}
#define NUM_PING 10

/* callback function for receiving ping reply */
void cb_ping(struct pico_icmp4_stats *s)
{
  char host[30];
  int time_sec = 0;
  int time_msec = 0;
  
  /* convert ip address from icmp4_stats structure to string */
  pico_ipv4_to_string(host, s->dst.addr);
  
  /* get time information from icmp4_stats structure */
  time_sec = s->time / 1000;
  time_msec = s->time % 1000;
  
  if (s->err == PICO_PING_ERR_REPLIED) {
  	/* print info if no error reported in icmp4_stats structure */
    dbg("%lu bytes from %s: icmp_req=%lu ttl=%lu time=%lu ms\n", \
    					s->size, host, s->seq, s->ttl, s->time);
    if (s->seq >= NUM_PING)
      exit(0);
  } else {
  	/* else, print error info */
    dbg("PING %lu to %s: Error %d\n", s->seq, host, s->err);
    exit(1);
  }
}

/* initialize the ping command */
void app_ping(char *dest)
{
  pico_icmp4_ping(dest, NUM_PING, 1000, 5000, 48, cb_ping);
}
\end{verbatim}


\section{UDP echo socket example}

\begin{verbatim}
struct pico_ip4 inaddr_any = { };

/* callback for UDP echo socket events */
void cb_udpecho(uint16_t ev, struct pico_socket *s)
{
  char recvbuf[1400];
  int read = 0;
  uint32_t peer;
  uint16_t port;

  /* process read event, data available */
  if (ev == PICO_SOCK_EV_RD) {
  	/* while data available in socket buffer, echo data to peer */
    do {
      read = pico_socket_recvfrom(s, recvbuf, 1400, &peer, &port);
      if (read > 0)
        pico_socket_sendto(s, recvbuf, r, &peer, port);
    } while(read > 0);
  }

  /* process error event, socket error occured */
  if (ev == PICO_SOCK_EV_ERR) {
    printf("Socket Error received. Bailing out.\n");
    exit(1);
  }

  printf("Received data from %08X:%u\n", peer, port);
}

/* initialize the UDP echo socket */
void app_udpecho(uint16_t source_port)
{
  struct pico_socket *s;
  uint16_t port_be = 0;
  
  /* set the source port for the socket */
  if (source_port == 0)
    port_be = short_be(5555);
  else
    port_be = short_be(source_port);

  /* open a UDP socket with the appropriate callback */
  s = pico_socket_open(PICO_PROTO_IPV4, PICO_PROTO_UDP, &cb_udpecho);
  if (!s)
    exit(1);

  /* bind the socket to port_be */
  if (pico_socket_bind(s, &inaddr_any, &port_be) != 0)
    exit(1);
}
\end{verbatim}


\section{TCP echo socket example}

\begin{verbatim}
#define BSIZE 1460

/* callback for TCP echo socket events */
void cb_tcpecho(uint16_t ev, struct pico_socket *s)
{
  char recvbuf[BSIZE];
  int read = 0, written = 0;
  int pos = 0, len = 0;
  struct pico_socket *sock_a;
  struct pico_ip4 orig;
  uint16_t port;
  char peer[30];

  /* process read event, data available */
  if (ev & PICO_SOCK_EV_RD) {
    do {
      read = pico_socket_read(s, recvbuf + len, BSIZE - len);
      if (read > 0)
        len += r;
    } while(read > 0);
  }
  
  /* process connect event, syn received */
  if (ev & PICO_SOCK_EV_CONN) {
    /* accept new connection request */
    sock_a = pico_socket_accept(s, &orig, &port);
    
   	/* convert peer IP to string */
    pico_ipv4_to_string(peer, orig.addr);
    
    /* print info */
    printf("Connection established with %s:%d.\n", peer, short_be(port));
  }

  /* process fin event, receiving socket closed */
  if (ev & PICO_SOCK_EV_FIN) {
    printf("Socket closed. Exit normally. \n");
  }

  /* process error event, socket error occured */
  if (ev & PICO_SOCK_EV_ERR) {
    printf("Socket Error received: %s. Bailing out.\n", strerror(pico_err));
    exit(1);
  }
  
  /* process close event, receiving socket received close from peer */
  if (ev & PICO_SOCK_EV_CLOSE) {
    printf("Socket received close from peer.\n");
    /* shutdown write side of socket */
    pico_socket_shutdown(s, PICO_SHUT_WR);
  }

  /* if data read, echo back */
  if (len > pos) {
    do {
      /* echo data back to peer */
      written = pico_socket_write(s, recvbuf + pos, len - pos);
      if (written > 0) {
        pos += written;
        if (pos >= len) {
          pos = 0;
          len = 0;
          written = 0;
        }
      } else {
        printf("SOCKET> ECHO write failed, dropped %d bytes\n",(len-pos));
      }
    } while(written > 0);
  }
}

/* initialize the TCP echo socket */
void app_tcpecho(uint16_t source_port)
{
  struct pico_socket *s;
  uint16_t port_be = 0;
  int backlog = 40;			/* max number of accepting connections */
  int ret;
  
  /* set the source port for the socket */
  if (source_port == 0)
    port_be = short_be(5555);
  else
    port_be = short_be(source_port);

  /* open a TCP socket with the appropriate callback */
  s = pico_socket_open(PICO_PROTO_IPV4, PICO_PROTO_TCP, &cb_tcpecho);
  if (!s)
    exit(1);

  /* bind the socket to port_be */
  ret = pico_socket_bind(s, &inaddr_any, &port_be);
  if (ret != 0)
    exit(1);

  /* start listening on socket */
  ret = pico_socket_listen(s, backlog);
  if (ret != 0)
    exit(1);
}
\end{verbatim}


\section{NAT setup example}

\begin{verbatim}
/* initialize NAT functionality and add port forward rule */
void app_nat(char *dest)
{
  char *dest = NULL;
  struct pico_ip4 ipdst, pub_addr, priv_addr;
  struct pico_ipv4_link *link;

  /* convert IP address of link where to enable NAT */
  pico_string_to_ipv4(dest, &ipdst.addr);
  
  /* get link pointer */
  link = pico_ipv4_link_get(&ipdst);
  if (!link) {
    printf("destination not found\n");
    exit(1);
  }
  
  /* enable NAT on link */
  pico_ipv4_nat_enable(link);
  
  /* add port forward rule */
  pico_string_to_ipv4("10.50.0.10", &pub_addr.addr);
  pico_string_to_ipv4("10.40.0.08", &priv_addr.addr);
  pico_ipv4_port_forward(pub_addr, short_be(5555), priv_addr, short_be(6667),
  PICO_PROTO_UDP, PICO_IPV4_FORWARD_ADD);
  
  printf("nat started\n");
}
\end{verbatim}


\section{DNS example}

\begin{verbatim}
/* callback function for receiving URL translation */
void cb_getaddr(char *ip)
{
  /* NULL indicates an error condition */
  if (!ip) {
    printf("DNS error occured: %s\n", strerror(pico_err));
    return;
  }
  printf("DNS translation to ip %s\n", ip);
  
  /* important: free the received pointer! */
  pico_free(ip);
}

/* callback function for receiving IP translation */
void cb_getname(char *url)
{
  /* NULL indicates an error condition */
  if (!url) {
    printf("DNS error occured: %s\n", strerror(pico_err));
    return;
  }
  printf("DNS translation to url %s\n", url);
  
  /* important: free the received pointer! */
  pico_free(url);
}

/* initialize the dns */
void app_dns(char *url, char *ip)
{
  struct pico_ip4 nameserver;
  
  /* optional: add custom dns nameserver */
  pico_string_to_ipv4("8.8.4.4", &nameserver.addr);
  pico_dns_client_nameserver(&nameserver, PICO_DNS_NS_ADD);

  /* request translation of URL f.e. www.google.com */  
  pico_dns_client_getaddr(url, &cb_getaddr);

  /* request translation of IP f.e. 8.8.8.8 */
  pico_dns_client_getname(ip, &cb_getname);
}
\end{verbatim}