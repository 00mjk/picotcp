\section{Socket calls}

% Short description/overview of module functions
With the socket calls, the user can open, close, bind, \ldots sockets and do read
or write operations. The provided transport protocols are UDP and TCP.

\subsection{pico$\_$socket$\_$open}

\subsubsection*{Description}
This function will be called to open a socket from the application level. The created
socket will be unbound.

\subsubsection*{Function prototype}
\begin{verbatim}
struct pico_socket *pico_socket_open(uint16_t net, uint16_t proto,
void (*wakeup)(uint16_t ev, struct pico_socket *s));
\end{verbatim}

\subsubsection*{Parameters}
\begin{itemize}[noitemsep]
\item \texttt{net} - Network protocol, \texttt{PICO$\_$PROTO$\_$IPV4} = 0, \texttt{PICO$\_$PROTO$\_$IPV6} = 41
\item \texttt{proto} - Transport protocol, \texttt{PICO$\_$PROTO$\_$TCP} = 6, \texttt{PICO$\_$PROTO$\_$UDP} = 17
\item \texttt{wakeup} - Callback function that accepts 2 parameters:
\begin{itemize}[noitemsep]
\item \texttt{ev} - Events that apply to that specific socket, see further
\item \texttt{s} - Pointer to a socket of type struct \texttt{pico$\_$socket}
\end{itemize}
\end{itemize}

\subsubsection*{Possible events for sockets}
\begin{itemize}[noitemsep]
\item \texttt{PICO$\_$SOCK$\_$EV$\_$RD} - trigerred when data arrived on the socket
\item \texttt{PICO$\_$SOCK$\_$EV$\_$WR} - trigerred when ready to write to the socket (TCP only)
\item \texttt{PICO$\_$SOCK$\_$EV$\_$CONN} - trigerred when connection is established (TCP only)
\item \texttt{PICO$\_$SOCK$\_$EV$\_$CLOSE} - trigerred when FIN packet received (TCP only)
\item \texttt{PICO$\_$SOCK$\_$EV$\_$FIN} - trigerred when the socket is closed (TCP only)
\item \texttt{PICO$\_$SOCK$\_$EV$\_$ERR} - trigerred when an error occurs
\end{itemize}

\subsubsection*{Return value}
On success, this call returns a pointer to the declared socket (\texttt{struct pico$\_$socket *}).
On error the socket is not created, \texttt{NULL} is returned, and \texttt{pico$\_$err} is set appropriately.

\subsubsection*{Errors}
\begin{itemize}[noitemsep]
\item \texttt{PICO$\_$ERR$\_$EINVAL} - invalid argument
\item \texttt{PICO$\_$ERR$\_$EPROTONOSUPPORT} - protocol not supported
\item \texttt{PICO$\_$ERR$\_$ENETUNREACH} - network unreachable 
\end{itemize}

\subsubsection*{Example}
\begin{verbatim}
sk_tcp = pico_socket_open(PICO_PROTO_IPV4, PICO_PROTO_TCP, &wakeup);
\end{verbatim}


\subsection{pico$\_$socket$\_$read}

\subsubsection*{Description}
This function will be called to read a string from a socket from the application level. The function checks whether or not the socket is bound.

\subsubsection*{Function prototype}
\begin{verbatim}
int pico_socket_read(struct pico_socket *s, void *buf, int len);
\end{verbatim}

\subsubsection*{Parameters}
\begin{itemize}[noitemsep]
\item \texttt{s} - Pointer to socket of type \texttt{struct pico$\_$socket}
\item \texttt{buf} - Void pointer to the start of a string buffer where the string will be stored
\item \texttt{len} - Length of the string that was read from the socket (in bytes)
\end{itemize}

\subsubsection*{Return value}
On success, this call returns an integer representing the number of bytes read.
On error, -1 is returned, and \texttt{pico$\_$err} is set appropriately.

\subsubsection*{Errors}
\begin{itemize}[noitemsep]
\item \texttt{PICO$\_$ERR$\_$EINVAL} - invalid argument
\item \texttt{PICO$\_$ERR$\_$EIO} - input/output error
\item \texttt{PICO$\_$ERR$\_$ESHUTDOWN} - cannot read after transport endpoint shutdown
\end{itemize}

\subsubsection*{Example}
\begin{verbatim}
bytesRead = pico_socket_read(sk_tcp, buffer, bufferLength);
\end{verbatim}



\subsection{pico$\_$socket$\_$write}

\subsubsection*{Description}
This function will be called to write a string to a socket from the application level.
This function also checks if the socket is bound, connected and that it isn't shutdown
locally. This is the preferred function to use when writing strings from application
level. 

\subsubsection*{Function prototype}
\begin{verbatim}
int pico_socket_write(struct pico_socket *s, void *buf, int len);
\end{verbatim}

\subsubsection*{Parameters}
\begin{itemize}[noitemsep]
\item \texttt{s} - Pointer to socket of type \texttt{struct pico$\_$socket}
\item \texttt{buf} - Void pointer to the start of a string buffer where the string is stored
\item \texttt{len} - Length of the string that is stored in the buffer (in bytes)
\end{itemize}

\subsubsection*{Return value}
On success, this call returns an integer representing the number of bytes written to the socket.
On error, -1 is returned, and \texttt{pico$\_$err} is set appropriately.

\subsubsection*{Errors}
\begin{itemize}[noitemsep]
\item \texttt{PICO$\_$ERR$\_$EINVAL} - invalid argument
\item \texttt{PICO$\_$ERR$\_$EIO} - input/output error
\item \texttt{PICO$\_$ERR$\_$ENOTCONN} - the socket is not connected
\item \texttt{PICO$\_$ERR$\_$ESHUTDOWN} - cannot send after transport endpoint shutdown
\item \texttt{PICO$\_$ERR$\_$EADDRNOTAVAIL} - address not available
\item \texttt{PICO$\_$ERR$\_$EHOSTUNREACH} - host is unreachable
\item \texttt{PICO$\_$ERR$\_$ENOMEM} - not enough space
\item \texttt{PICO$\_$ERR$\_$EAGAIN} - resource temporarily unavailable
\end{itemize}

\subsubsection*{Example}
\begin{verbatim}
bytesWritten = pico_socket_write(sk_tcp, buffer, bufLength);
\end{verbatim}


\subsection{pico$\_$socket$\_$sendto}

\subsubsection*{Description}
This function is be called by the \texttt{pico$\_$socket$\_$write} and \texttt{pico$\_$socket$\_$send} functions.
This function sends a string from the local address to the remote address, without checking
if the remote is connected or not.

\subsubsection*{Function prototype}
\begin{verbatim}
int pico_socket_sendto(struct pico_socket *s, const void *buf, int len,
void *dst, uint16_t remote_port);
\end{verbatim}

\subsubsection*{Parameters}
\begin{itemize}[noitemsep]
\item \texttt{s} - Pointer to socket of type \texttt{struct pico$\_$socket}
\item \texttt{buf} - Void pointer to the start of a string buffer where the string is stored
\item \texttt{len} - Length of the string that is stored in the buffer (in bytes)
\item \texttt{dst} - Pointer to the origin of the IPv4/IPv6 frame header
\item \texttt{remote$\_$port} - Portnumber of the receiving socket
\end{itemize}

\subsubsection*{Return value}
On success, this call returns an integer representing the number of bytes written to the socket.
On error, -1 is returned, and \texttt{pico$\_$err} is set appropriately.

\subsubsection*{Errors}
\begin{itemize}[noitemsep]
\item \texttt{PICO$\_$ERR$\_$EADDRNOTAVAIL} - address not available
\item \texttt{PICO$\_$ERR$\_$EINVAL} - invalid argument
\item \texttt{PICO$\_$ERR$\_$EHOSTUNREACH} - host is unreachable
\item \texttt{PICO$\_$ERR$\_$ENOMEM} - not enough space
\item \texttt{PICO$\_$ERR$\_$EAGAIN} - resource temporarily unavailable
\end{itemize}

\subsubsection*{Example}
\begin{verbatim}
bytesWritten = pico_socket_sendto(sk_tcp, buf, len, &sk_tcp->remote_addr,
sk_tcp->remote_port);
\end{verbatim}


\subsection{pico$\_$socket$\_$recvfrom}

\subsubsection*{Description}
This function is called to receive a string of data from the specified socket.
This function also checks if the socket is bound but not if it is connected or shutdown locally. 

\subsubsection*{Function prototype}
\begin{verbatim}
int pico_socket_recvfrom(struct pico_socket *s, void *buf, int len,
void *orig, uint16_t *remote_port);
\end{verbatim}

\subsubsection*{Parameters}
\begin{itemize}[noitemsep]
\item \texttt{s} - Pointer to socket of type \texttt{struct pico$\_$socket}
\item \texttt{buf} - Void pointer to the start of a string buffer where the string will be stored
\item \texttt{len} - Length of the string that will be stored in the buffer (in bytes)
\item \texttt{orig} - Pointer to the origin of the IPv4/IPv6 frame header, can be NULL
\item \texttt{remote$\_$port} - Pointer to the port number of the sender socket, can be NULL
\end{itemize}

\subsubsection*{Return value}
On success, this call returns an integer representing the number of bytes read from the socket. Also
\texttt{remote$\_$port} will contain the portnumber of the sending socket.
On error, -1 is returned, and \texttt{pico$\_$err} is set appropriately.

\subsubsection*{Errors}
\begin{itemize}[noitemsep]
\item \texttt{PICO$\_$ERR$\_$EINVAL} - invalid argument
\item \texttt{PICO$\_$ERR$\_$ESHUTDOWN} - cannot read after transport endpoint shutdown
\item \texttt{PICO$\_$ERR$\_$EADDRNOTAVAIL} - address not available
\end{itemize}

\subsubsection*{Example}
\begin{verbatim}
bytesRcvd = pico_socket_recvfrom(sk_tcp, buf, bufLen, &peer, &port);
\end{verbatim}


\subsection{pico$\_$socket$\_$send}

\subsubsection*{Description}
This function is called to send a string of data to the specified socket.
This function also checks if the socket is connected and then calls the
\texttt{pico$\_$socket$\_$sendto} function.

\subsubsection*{Function prototype}
\begin{verbatim}
int pico_socket_send(struct pico_socket *s, const void *buf, int len);
\end{verbatim}


\subsubsection*{Parameters}
\begin{itemize}[noitemsep]
\item \texttt{s} - Pointer to socket of type \texttt{struct pico$\_$socket}
\item \texttt{buf} - Void pointer to the start of a string buffer where the string is stored
\item \texttt{len} - Length of the string that is stored in the buffer (in bytes)
\end{itemize}

\subsubsection*{Return value}
On success, this call returns an integer representing the number of bytes written to
the socket. On error, -1 is returned, and \texttt{pico$\_$err} is set appropriately.

\subsubsection*{Errors}
\begin{itemize}[noitemsep]
\item \texttt{PICO$\_$ERR$\_$EINVAL} - invalid argument
\item \texttt{PICO$\_$ERR$\_$ENOTCONN} - the socket is not connected
\item \texttt{PICO$\_$ERR$\_$EADDRNOTAVAIL} - address not available
\item \texttt{PICO$\_$ERR$\_$EHOSTUNREACH} - host is unreachable
\item \texttt{PICO$\_$ERR$\_$ENOMEM} - not enough space
\item \texttt{PICO$\_$ERR$\_$EAGAIN} - resource temporarily unavailable
\end{itemize}

\subsubsection*{Example}
\begin{verbatim}
bytesRcvd = pico_socket_send(sk_tcp, buf, bufLen);
\end{verbatim}


\subsection{pico$\_$socket$\_$recv}

\subsubsection*{Description}
This function directly calls the \texttt{pico$\_$socket$\_$recvfrom} function.

\subsubsection*{Function prototype}
\begin{verbatim}
int pico_socket_recv(struct pico_socket *s, void *buf, int len);
\end{verbatim}

\subsubsection*{Parameters}
\begin{itemize}[noitemsep]
\item \texttt{s} - Pointer to socket of type \texttt{struct pico$\_$socket}
\item \texttt{buf} - Void pointer to the start of a string buffer where the string will be stored
\item \texttt{len} - Length of the string in the socket buffer (in bytes)
\end{itemize}

\subsubsection*{Return value}
On success, this call returns an integer representing the number of bytes read
from the socket. On error, -1 is returned, and \texttt{pico$\_$err} is set appropriately.

\subsubsection*{Errors}
\begin{itemize}[noitemsep]
\item \texttt{PICO$\_$ERR$\_$EINVAL} - invalid argument
\item \texttt{PICO$\_$ERR$\_$ESHUTDOWN} - cannot read after transport endpoint shutdown
\item \texttt{PICO$\_$ERR$\_$EADDRNOTAVAIL} - address not available
\end{itemize}

\subsubsection*{Example}
\begin{verbatim}
bytesRcvd = pico_socket_recv(sk_tcp, buf, bufLen);
\end{verbatim}


\subsection{pico$\_$socket$\_$bind}

\subsubsection*{Description}
This function binds a local IP-address and port to the specified socket.

\subsubsection*{Function prototype}
\begin{verbatim}
int pico_socket_bind(struct pico_socket *s, void *local_addr, uint16_t *port);
\end{verbatim}


\subsubsection*{Parameters}
\begin{itemize}[noitemsep]
\item \texttt{s} - Pointer to socket of type \texttt{struct pico$\_$socket}
\item \texttt{local$\_$addr} - Void pointer to the local IP-address
\item \texttt{port} - Local portnumber to bind with the socket
\end{itemize}

\subsubsection*{Return value}
On success, this call returns 0 after a succesfull bind.
On error, -1 is returned, and \texttt{pico$\_$err} is set appropriately.

\subsubsection*{Errors}
\begin{itemize}[noitemsep]
\item \texttt{PICO$\_$ERR$\_$EINVAL} - invalid argument
\item \texttt{PICO$\_$ERR$\_$ENOMEM} - not enough space
\item \texttt{PICO$\_$ERR$\_$ENXIO} - no such device or address
\end{itemize}

\subsubsection*{Example}
\begin{verbatim}
errMsg = pico_socket_bind(sk_tcp, &sockaddr4->addr, &sockaddr4->port);
\end{verbatim}


\subsection{pico$\_$socket$\_$connect}

\subsubsection*{Description}
This function connects a local socket to a remote socket of a server that is listening, or permanently associate a remote UDP peer as default receiver for any further outgoing traffic through this socket.

\subsubsection*{Function prototype}
\begin{verbatim}
int pico_socket_connect(struct pico_socket *s, void *srv_addr,
uint16_t remote_port);
\end{verbatim}


\subsubsection*{Parameters}
\begin{itemize}[noitemsep]
\item \texttt{s} - Pointer to socket of type \texttt{struct pico$\_$socket}
\item \texttt{srv$\_$addr} - Void pointer to the remote IP-address to connect to
\item \texttt{remote$\_$port} - Remote port number on which the socket will be connected to
\end{itemize} 

\subsubsection*{Return value}
On success, this call returns 0 after a succesfull connect.
On error, -1 is returned, and \texttt{pico$\_$err} is set appropriately.

\subsubsection*{Errors}
\begin{itemize}[noitemsep]
\item \texttt{PICO$\_$ERR$\_$EPROTONOSUPPORT} - protocol not supported
\item \texttt{PICO$\_$ERR$\_$EINVAL} - invalid argument
\item \texttt{PICO$\_$ERR$\_$EHOSTUNREACH} - host is unreachable 
\end{itemize}

\subsubsection*{Example}
\begin{verbatim}
errMsg = pico_socket_connect(sk_tcp, &sockaddr4->addr, sockaddr4->port);
\end{verbatim}


\subsection{pico$\_$socket$\_$listen}

\subsubsection*{Description}
A server can use this function when a socket is opened and bound to start listening to it.

\subsubsection*{Function prototype}
\begin{verbatim}
int pico_socket_listen(struct pico_socket *s, int backlog);
\end{verbatim}


\subsubsection*{Parameters}
\begin{itemize}[noitemsep]
\item \texttt{s} - Pointer to socket of type \texttt{struct pico$\_$socket}
\item \texttt{backlog} - Maximum connection requests
\end{itemize}

\subsubsection*{Return value}
On success, this call returns 0 after a succesfull listen start.
On error, -1 is returned, and \texttt{pico$\_$err} is set appropriately. 

\subsubsection*{Errors}
\begin{itemize}[noitemsep]
\item \texttt{PICO$\_$ERR$\_$EINVAL} - invalid argument
\item \texttt{PICO$\_$ERR$\_$EISCONN} - socket is connected
\end{itemize}

\subsubsection*{Example}
\begin{verbatim}
errMsg = pico_socket_listen(sk_tcp, 3);
\end{verbatim}


\subsection{pico$\_$socket$\_$accept}

\subsubsection*{Description}
When a server is listening on a socket and the client is trying to connect.
The server on his side will wakeup and acknowledge the connection by calling the this function.

\subsubsection*{Function prototype}
\begin{verbatim}
struct pico_socket *pico_socket_accept(struct pico_socket *s, void *orig,
uint16_t *local_port);
\end{verbatim}

\subsubsection*{Parameters}
\begin{itemize}[noitemsep]
\item \texttt{s} - Pointer to socket of type \texttt{struct pico$\_$socket}
\item \texttt{orig} - Pointer to the origin of the IPv4/IPv6 frame header
\item \texttt{local$\_$port} - Portnumber of the local socket (pointer)
\end{itemize}

\subsubsection*{Return value}
On success, this call returns the pointer to a \texttt{struct pico$\_$socket} that
represents the client thas was just connected. Also \texttt{orig} will contain the requesting
IP-address and \texttt{remote$\_$port} will contain the portnumber of the requesting socket.
On error, \texttt{NULL} is returned, and \texttt{pico$\_$err} is set appropriately.

\subsubsection*{Errors}
\begin{itemize}[noitemsep]
\item \texttt{PICO$\_$ERR$\_$EINVAL} - invalid argument
\item \texttt{PICO$\_$ERR$\_$EAGAIN} - resource temporarily unavailable
\end{itemize}

\subsubsection*{Example}
\begin{verbatim}
client = pico_socket_accept(sk_tcp, &peer, &port);
\end{verbatim}


\subsection{pico$\_$socket$\_$shutdown}

\subsubsection*{Description}
Used by the \texttt{pico$\_$socket$\_$close} function to shutdown read and write mode for
the specified socket. With this function one can close a socket for reading
and/or writing.

\subsubsection*{Function prototype}
\begin{verbatim}
int pico_socket_shutdown(struct pico_socket *s, int mode);
\end{verbatim}

\subsubsection*{Parameters}
\begin{itemize}[noitemsep]
\item \texttt{s} - Pointer to socket of type \texttt{struct pico$\_$socket}
\item \texttt{mode} - \texttt{PICO$\_$SHUT$\_$RDWR}, \texttt{PICO$\_$SHUT$\_$WR}, \texttt{PICO$\_$SHUT$\_$RD}
\end{itemize}

\subsubsection*{Return value}
On success, this call returns 0 after a succesfull socket shutdown.
On error, -1 is returned, and \texttt{pico$\_$err} is set appropriately.

\subsubsection*{Errors}
\begin{itemize}[noitemsep]
\item \texttt{PICO$\_$ERR$\_$EINVAL} - invalid argument
\end{itemize}

\subsubsection*{Example}
\begin{verbatim}
errMsg = pico_socket_shutdown(s, PICO_SHUT_RDWR);
\end{verbatim}


\subsection{pico$\_$socket$\_$close}

\subsubsection*{Description}
Function used on application level to close a socket. Always closes read and write connection.

\subsubsection*{Function prototype}
\begin{verbatim}
int pico_socket_close(struct pico_socket *s);
\end{verbatim}

\subsubsection*{Parameters}
\begin{itemize}[noitemsep]
\item \texttt{s} - Pointer to socket of type \texttt{struct pico$\_$socket}
\end{itemize}

\subsubsection*{Return value}
On success, this call returns 0 after a succesfull socket shutdown.
On error, -1 is returned, and \texttt{pico$\_$err} is set appropriately.

\subsubsection*{Errors}
\begin{itemize}[noitemsep]
\item \texttt{PICO$\_$ERR$\_$EINVAL} - invalid argument
\end{itemize}

\subsubsection*{Example}
\begin{verbatim}
errMsg = pico_socket_close(sk_tcp);
\end{verbatim}



\subsection{pico$\_$socket$\_$setoption}

\subsubsection*{Description}
Function used to set socket options.

\subsubsection*{Function prototype}
\begin{verbatim}
int pico_socket_setoption(struct pico_socket *s, int option, void *value);
\end{verbatim}

\subsubsection*{Parameters}
\begin{itemize}[noitemsep]
\item \texttt{s} - Pointer to socket of type \texttt{struct pico$\_$socket}
\item \texttt{option} - Option to be set (see further for all options)
\item \texttt{value} - Value of option (void pointer)
\end{itemize}

\subsubsection*{Available socket options}
\begin{itemize}[noitemsep]
\item \texttt{PICO$\_$TCP$\_$NODELAY} - Disables/enables the Nagle algorithm
\item \texttt{PICO$\_$IP$\_$MULTICAST$\_$IF} - (Not supported) Set link multicast datagrams are sent from, default is first added link
\item \texttt{PICO$\_$IP$\_$MULTICAST$\_$TTL} - Set TTL (0-255) of multicast datagrams, default is 1
\item \texttt{PICO$\_$IP$\_$MULTICAST$\_$LOOP} - Specifies if a copy of an outgoing multicast datagram is looped back as long as it is a member of the multicast group, default is enabled
\item \texttt{PICO$\_$IP$\_$ADD$\_$MEMBERSHIP} - Join the multicast group specified
\item \texttt{PICO$\_$IP$\_$DROP$\_$MEMBERSHIP} - Leave the multicast group specified
\end{itemize}

\subsubsection*{Return value}
On success, this call returns 0 after a succesfull setting of socket option.
On error, -1 is returned, and \texttt{pico$\_$err} is set appropriately.

\subsubsection*{Errors}
\begin{itemize}[noitemsep]
\item \texttt{PICO$\_$ERR$\_$EINVAL} - invalid argument
\end{itemize}

\subsubsection*{Example}
\begin{verbatim}
ret = pico_socket_setoption(sk_tcp, PICO_TCP_NODELAY, NULL);

uint8_t ttl = 2;
ret = pico_socket_setoption(sk_udp, PICO_IP_MULTICAST_TTL, &ttl);

uint8_t loop = 0;
ret = pico_socket_setoption(sk_udp, PICO_IP_MULTICAST_LOOP, &loop);

struct pico_ip4 inaddr_dst, inaddr_link;
struct pico_ip_mreq mreq = {{0},{0}};
pico_string_to_ipv4("224.7.7.7", &inaddr_dst.addr);
pico_string_to_ipv4("192.168.0.2", &inaddr_link.addr);
mreq.mcast_group_addr = inaddr_dst;
mreq.mcast_link_addr = inaddr_link;
ret = pico_socket_setoption(sk_udp, PICO_IP_ADD_MEMBERSHIP, &mreq);
ret = pico_socket_setoption(sk_udp, PICO_IP_DROP_MEMBERSHIP, &mreq)
\end{verbatim}


\subsection{pico$\_$socket$\_$getoption}

\subsubsection*{Description}
Function used to get socket options.

\subsubsection*{Function prototype}
\begin{verbatim}
int pico_socket_getoption(struct pico_socket *s, int option, void *value);
\end{verbatim}

\subsubsection*{Parameters}
\begin{itemize}[noitemsep]
\item \texttt{s} - Pointer to socket of type \texttt{struct pico$\_$socket}
\item \texttt{option} - Option to be set (see further for all options)
\item \texttt{value} - Value of option (void pointer)
\end{itemize}

\subsubsection*{Available socket options}
\begin{itemize}[noitemsep]
\item \texttt{PICO$\_$TCP$\_$NODELAY} - Nagle algorithm, \texttt{value} casted to \texttt{(int *)} (0 = disabled, 1 = enabled)
\item \texttt{PICO$\_$IP$\_$MULTICAST$\_$IF} - (Not supported) Link multicast datagrams are sent from
\item \texttt{PICO$\_$IP$\_$MULTICAST$\_$TTL} - TTL (0-255) of multicast datagrams
\item \texttt{PICO$\_$IP$\_$MULTICAST$\_$LOOP} - Loop back a copy of an outgoing multicast datagram, as long as it is a member of the multicast group, or not.
\end{itemize}

\subsubsection*{Return value}
On success, this call returns 0 after a succesfull getting of socket option. The value of
the option is written to \texttt{value}.
On error, -1 is returned, and \texttt{pico$\_$err} is set appropriately.

\subsubsection*{Errors}
\begin{itemize}[noitemsep]
\item \texttt{PICO$\_$ERR$\_$EINVAL} - invalid argument
\end{itemize}

\subsubsection*{Example}
\begin{verbatim}
ret = pico_socket_getoption(sk_tcp, PICO_TCP_NODELAY, &stat);

uint8_t ttl = 0;
ret = pico_socket_getoption(sk_udp, PICO_IP_MULTICAST_TTL, &ttl);

uint8_t loop = 0;
ret = pico_socket_getoption(sk_udp, PICO_IP_MULTICAST_LOOP, &loop);
\end{verbatim}
