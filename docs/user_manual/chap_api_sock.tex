\section{Socket calls}

% Short description/overview of module functions
With the socket calls, the user can open, close, bind, \ldots sockets and do read
or write operations. The provided transport protocols are UDP and TCP.

\subsection{pico$\_$socket$\_$open}

\subsubsection*{Description}
This function will be called to open a socket from the application level. The created
socket will be unbound and not connected. 

\subsubsection*{Function prototype}
\begin{verbatim}
struct pico_socket *pico_socket_open(uint16_t net, uint16_t proto,
void (*wakeup)(uint16_t ev, struct pico_socket *s));
\end{verbatim}

\subsubsection*{Parameters}
\begin{itemize}[noitemsep]
\item \texttt{net} - Network protocol, \texttt{PICO$\_$PROTO$\_$IPV4} = 0, \texttt{PICO$\_$PROTO$\_$IPV6} = 41
\item \texttt{proto} - Transport protocol, \texttt{PICO$\_$PROTO$\_$TCP} = 6, \texttt{PICO$\_$PROTO$\_$UDP} = 17
\item \texttt{wakeup} - Callback function that accepts 2 parameters:
\begin{itemize}[noitemsep]
\item \texttt{ev} - Events that apply to that specific socket, see further
\item \texttt{s} - Pointer to a socket of type struct \texttt{pico$\_$socket}
\end{itemize}
\end{itemize}

\subsubsection*{Possible events for sockets}
\begin{itemize}[noitemsep]
\item \texttt{PICO$\_$SOCK$\_$EV$\_$RD} - triggered when new data arrives on the socket. A new receive action can be taken by the socket owner because this event indicates there is new data to receive.
\item \texttt{PICO$\_$SOCK$\_$EV$\_$WR} - triggered when ready to write to the socket. Issuing a write/send call will now succeed if the buffer has enough space to allocate new outstanding data.
\item \texttt{PICO$\_$SOCK$\_$EV$\_$CONN} - triggered when connection is established (TCP only). This event is received either after a successful call to \texttt{pico$\_$socket$\_$connect} to indicate that the connection has been established, or on a listening socket, indicating that a call to \texttt{pico$\_$socket$\_$accept} may now be issued in order to accept the incoming connection from a remote host.
\item \texttt{PICO$\_$SOCK$\_$EV$\_$CLOSE} - triggered when a FIN segment is received (TCP only). This event indicates that the other endpont has closed the connection, so the local TCP layer is only allowed to send new data until a local shutdown or close is initiated. PicoTCP is able to keep the connection half-open (only for sending) after the FIN packet has been received, allowing new data to be sent in the TCP CLOSE$\_$WAIT state.
\item \texttt{PICO$\_$SOCK$\_$EV$\_$FIN} - triggered when the socket is closed. No further communication is possible from this point on the socket.
\item \texttt{PICO$\_$SOCK$\_$EV$\_$ERR} - triggered when an error occurs. 
\end{itemize}

\subsubsection*{Return value}
On success, this call returns a pointer to the declared socket (\texttt{struct pico$\_$socket *}).
On error the socket is not created, \texttt{NULL} is returned, and \texttt{pico$\_$err} is set appropriately.

\subsubsection*{Errors}
\begin{itemize}[noitemsep]
\item \texttt{PICO$\_$ERR$\_$EINVAL} - invalid argument
\item \texttt{PICO$\_$ERR$\_$EPROTONOSUPPORT} - protocol not supported
\item \texttt{PICO$\_$ERR$\_$ENETUNREACH} - network unreachable 
\end{itemize}

\subsubsection*{Example}
\begin{verbatim}
sk_tcp = pico_socket_open(PICO_PROTO_IPV4, PICO_PROTO_TCP, &wakeup);
\end{verbatim}


\subsection{pico$\_$socket$\_$read}

\subsubsection*{Description}
This function will be called to read data from a connected socket. The function checks that the socket is bound and connected before attempting to receive data. 

\subsubsection*{Function prototype}
\begin{verbatim}
int pico_socket_read(struct pico_socket *s, void *buf, int len);
\end{verbatim}

\subsubsection*{Parameters}
\begin{itemize}[noitemsep]
\item \texttt{s} - Pointer to socket of type \texttt{struct pico$\_$socket}
\item \texttt{buf} - Void pointer to the start of the buffer where the received data will be stored
\item \texttt{len} - Length of the buffer (in bytes), represents the maximum amount of bytes that can be read
\end{itemize}

\subsubsection*{Return value}
On success, this call returns an integer representing the number of bytes read.
On error, -1 is returned, and \texttt{pico$\_$err} is set appropriately.

\subsubsection*{Errors}
\begin{itemize}[noitemsep]
\item \texttt{PICO$\_$ERR$\_$EINVAL} - invalid argument
\item \texttt{PICO$\_$ERR$\_$EIO} - input/output error
\item \texttt{PICO$\_$ERR$\_$ESHUTDOWN} - cannot read after transport endpoint shutdown
\end{itemize}

\subsubsection*{Example}
\begin{verbatim}
bytesRead = pico_socket_read(sk_tcp, buffer, bufferLength);
\end{verbatim}



\subsection{pico$\_$socket$\_$write}

\subsubsection*{Description}
This function will be called to write the content of a buffer to a socket that has been previously connected.
This function checks that the socket is bound, connected and that it is allowed to send data, i.e. there hasn't been a local shutdown.
This is the preferred function to use when writing data from the application to a connected stream. 

\subsubsection*{Function prototype}
\begin{verbatim}
int pico_socket_write(struct pico_socket *s, void *buf, int len);
\end{verbatim}

\subsubsection*{Parameters}
\begin{itemize}[noitemsep]
\item \texttt{s} - Pointer to socket of type \texttt{struct pico$\_$socket}
\item \texttt{buf} - Void pointer to the start of a (constant) buffer where the data is stored
\item \texttt{len} - Length of the data buffer \texttt{buf} 
\end{itemize}

\subsubsection*{Return value}
On success, this call returns an integer representing the number of bytes written to the socket.
On error, -1 is returned, and \texttt{pico$\_$err} is set appropriately.

\subsubsection*{Errors}
\begin{itemize}[noitemsep]
\item \texttt{PICO$\_$ERR$\_$EINVAL} - invalid argument
\item \texttt{PICO$\_$ERR$\_$EIO} - input/output error
\item \texttt{PICO$\_$ERR$\_$ENOTCONN} - the socket is not connected
\item \texttt{PICO$\_$ERR$\_$ESHUTDOWN} - cannot send after transport endpoint shutdown
\item \texttt{PICO$\_$ERR$\_$EADDRNOTAVAIL} - address not available
\item \texttt{PICO$\_$ERR$\_$EHOSTUNREACH} - host is unreachable
\item \texttt{PICO$\_$ERR$\_$ENOMEM} - not enough space
\item \texttt{PICO$\_$ERR$\_$EAGAIN} - resource temporarily unavailable
\end{itemize}

\subsubsection*{Example}
\begin{verbatim}
bytesWritten = pico_socket_write(sk_tcp, buffer, bufLength);
\end{verbatim}


\subsection{pico$\_$socket$\_$sendto}

\subsubsection*{Description}
This function sends data from the local address to the remote address, without checking
whether the remote endpoint is connected. Specifying the destination is particularly useful while sending single datagrams
to different destinations upon consecutive calls. This is the preferred mechanism to send datagrams to a remote destination
using a UDP socket.

\subsubsection*{Function prototype}
\begin{verbatim}
int pico_socket_sendto(struct pico_socket *s, const void *buf, int len,
void *dst, uint16_t remote_port);
\end{verbatim}

\subsubsection*{Parameters}
\begin{itemize}[noitemsep]
\item \texttt{s} - Pointer to socket of type \texttt{struct pico$\_$socket}
\item \texttt{buf} - Void pointer to the start of the buffer
\item \texttt{len} - Length of the buffer \texttt{buf}
\item \texttt{dst} - Pointer to the origin of the IPv4/IPv6 frame header
\item \texttt{remote$\_$port} - Portnumber of the receiving socket
\end{itemize}

\subsubsection*{Return value}
On success, this call returns an integer representing the number of bytes written to the socket.
On error, -1 is returned, and \texttt{pico$\_$err} is set appropriately.

\subsubsection*{Errors}
\begin{itemize}[noitemsep]
\item \texttt{PICO$\_$ERR$\_$EADDRNOTAVAIL} - address not available
\item \texttt{PICO$\_$ERR$\_$EINVAL} - invalid argument
\item \texttt{PICO$\_$ERR$\_$EHOSTUNREACH} - host is unreachable
\item \texttt{PICO$\_$ERR$\_$ENOMEM} - not enough space
\item \texttt{PICO$\_$ERR$\_$EAGAIN} - resource temporarily unavailable
\end{itemize}

\subsubsection*{Example}
\begin{verbatim}
bytesWritten = pico_socket_sendto(sk_tcp, buf, len, &sk_tcp->remote_addr,
sk_tcp->remote_port);
\end{verbatim}


\subsection{pico$\_$socket$\_$recvfrom}

\subsubsection*{Description}
This function is called to receive data from the specified socket.
It is useful when called in the context of a non-connected socket, to receive
the information regarding the origin of the data, namely the origin address and 
the remote port number.

\subsubsection*{Function prototype}
\begin{verbatim}
int pico_socket_recvfrom(struct pico_socket *s, void *buf, int len,
void *orig, uint16_t *remote_port);
\end{verbatim}

\subsubsection*{Parameters}
\begin{itemize}[noitemsep]
\item \texttt{s} - Pointer to socket of type \texttt{struct pico$\_$socket}
\item \texttt{buf} - Void pointer to the start of the buffer
\item \texttt{len} - Maximum allowed length for the data to be stored in the buffer \texttt{buf}
\item \texttt{orig} - Pointer to the origin of the IPv4/IPv6 frame header, can be NULL
\item \texttt{remote$\_$port} - Pointer to the port number of the sender socket, can be NULL
\end{itemize}

\subsubsection*{Return value}
On success, this call returns an integer representing the number of bytes read from the socket. On success, if \texttt{orig}
is not NULL, The address of the remote endpoint is stored in the memory area pointed by \texttt{orig}. 
In the same way, \texttt{remote$\_$port} will contain the portnumber of the sending socket, unless a NULL is passed
from the caller.

On error, -1 is returned, and \texttt{pico$\_$err} is set appropriately.

\subsubsection*{Errors}
\begin{itemize}[noitemsep]
\item \texttt{PICO$\_$ERR$\_$EINVAL} - invalid argument
\item \texttt{PICO$\_$ERR$\_$ESHUTDOWN} - cannot read after transport endpoint shutdown
\item \texttt{PICO$\_$ERR$\_$EADDRNOTAVAIL} - address not available
\end{itemize}

\subsubsection*{Example}
\begin{verbatim}
bytesRcvd = pico_socket_recvfrom(sk_tcp, buf, bufLen, &peer, &port);
\end{verbatim}

\subsection{Extended Socket operations}
The interface provided by sendto/recvfrom can be extended to include more information about the network communication. 
This is especially useful in UDP communication, and whenever extended information is needed about the single datagram and its encapsulation in the networking layer.

PicoTCP offers an extra structure that can be used to set and retrieve message information while transmitting and receiving datagrams, respectively. The structure \texttt{pico$\_$msginfo} is defined as follows:
\begin{verbatim}
struct pico_msginfo {
    struct pico_device *dev;
    uint8_t ttl;
    uint8_t tos;
};
\end{verbatim}



\subsection{pico$\_$socket$\_$sendto$\_$extended}

\subsubsection*{Description}
This function is an extension of the \texttt{pico$\_$socket$\_$sendto} function described above. It's exactly the same but it adds up an additional argument to set TTL and QOS information on the outgoing packet which contains the datagram.

The usage of the extended argument makes sense in UDP context only, as the information is set at packet level, and only with UDP there is a 1:1 correspondence between datagrams and IP packets.

\subsubsection*{Function prototype}
\begin{verbatim}
int pico_socket_sendto_extended(struct pico_socket *s, const void *buf, int len,
void *dst, uint16_t remote_port, struct pico_msginfo *info);
\end{verbatim}

\subsubsection*{Parameters}
\begin{itemize}[noitemsep]
\item \texttt{s} - Pointer to socket of type \texttt{struct pico$\_$socket}
\item \texttt{buf} - Void pointer to the start of the buffer
\item \texttt{len} - Length of the data that is stored in the buffer (in bytes)
\item \texttt{dst} - Pointer to the origin of the IPv4/IPv6 frame header
\item \texttt{remote$\_$port} - Port number of the receiving socket at the remote endpoint
\item \texttt{info} - Extended information about the packet containing this datagram. Only the fields "ttl" and "tos" are taken into consideeration, while "dev" is ignored.

\end{itemize}

\subsubsection*{Return value}
On success, this call returns an integer representing the number of bytes written to the socket.
On error, -1 is returned, and \texttt{pico$\_$err} is set appropriately.

\subsubsection*{Errors}
\begin{itemize}[noitemsep]
\item \texttt{PICO$\_$ERR$\_$EADDRNOTAVAIL} - address not available
\item \texttt{PICO$\_$ERR$\_$EINVAL} - invalid argument
\item \texttt{PICO$\_$ERR$\_$EHOSTUNREACH} - host is unreachable
\item \texttt{PICO$\_$ERR$\_$ENOMEM} - not enough space
\item \texttt{PICO$\_$ERR$\_$EAGAIN} - resource temporarily unavailable
\end{itemize}

\subsubsection*{Example}
\begin{verbatim}
struct pico_msginfo info = { };
info.ttl = 5;
bytesWritten = pico_socket_sendto_extended(sk_tcp, buf, len, &sk_tcp->remote_addr,
sk_tcp->remote_port, &info);
\end{verbatim}


\subsection{pico$\_$socket$\_$recvfrom$\_$extended}

\subsubsection*{Description}
This function is an extension to the normal \texttt{pico$\_$socket$\_$recvfrom} function, which allows to retrieve additional information about the networking layer that has been involved in the delivery of the datagram.

\subsubsection*{Function prototype}
\begin{verbatim}
int pico_socket_recvfrom_extended(struct pico_socket *s, void *buf, int len,
void *orig, uint16_t *remote_port, struct pico_msginfo *info);
\end{verbatim}

\subsubsection*{Parameters}
\begin{itemize}[noitemsep]
\item \texttt{s} - Pointer to socket of type \texttt{struct pico$\_$socket}
\item \texttt{buf} - Void pointer to the start of the buffer
\item \texttt{len} - Maximum allowed length for the data to be stored in the buffer \texttt{buf}
\item \texttt{orig} - Pointer to the origin of the IPv4/IPv6 frame header, can be NULL
\item \texttt{remote$\_$port} - Pointer to the port number of the sender socket, can be NULL
\item \texttt{info} - Extended information about the incoming packet containing this datagram. The device where the packet was received is pointed by info->dev, the maximum TTL for the packet is stored in info->ttl, and finally the field info->tos keeps track of the flags in IP header's QoS.
\end{itemize}

\subsubsection*{Return value}
On success, this call returns an integer representing the number of bytes read from the socket. On success, if \texttt{orig}
is not NULL, The address of the remote endpoint is stored in the memory area pointed by \texttt{orig}. 
In the same way, \texttt{remote$\_$port} will contain the portnumber of the sending socket, unless a NULL is passed
from the caller.

On error, -1 is returned, and \texttt{pico$\_$err} is set appropriately.

\subsubsection*{Errors}
\begin{itemize}[noitemsep]
\item \texttt{PICO$\_$ERR$\_$EINVAL} - invalid argument
\item \texttt{PICO$\_$ERR$\_$ESHUTDOWN} - cannot read after transport endpoint shutdown
\item \texttt{PICO$\_$ERR$\_$EADDRNOTAVAIL} - address not available
\end{itemize}

\subsubsection*{Example}
\begin{verbatim}
struct pico_msginfo info;
bytesRcvd = pico_socket_recvfrom_extended(sk_tcp, buf, bufLen, &peer, &port, &info);
if (info && info->dev) {
    printf("Socket received a datagram via device %s, ttl:%d, tos: %08x\n", 
        info->dev->name, info->ttl, info->tos);
}
\end{verbatim}


\subsection{pico$\_$socket$\_$send}

\subsubsection*{Description}
This function is called to send data to the specified socket.
It checks if the socket is connected and then calls the
\texttt{pico$\_$socket$\_$sendto} function.

\subsubsection*{Function prototype}
\begin{verbatim}
int pico_socket_send(struct pico_socket *s, const void *buf, int len);
\end{verbatim}


\subsubsection*{Parameters}
\begin{itemize}[noitemsep]
\item \texttt{s} - Pointer to socket of type \texttt{struct pico$\_$socket}
\item \texttt{buf} - Void pointer to the start of the buffer
\item \texttt{len} - Length of the buffer \texttt{buf}
\end{itemize}

\subsubsection*{Return value}
On success, this call returns an integer representing the number of bytes written to
the socket. On error, -1 is returned, and \texttt{pico$\_$err} is set appropriately.

\subsubsection*{Errors}
\begin{itemize}[noitemsep]
\item \texttt{PICO$\_$ERR$\_$EINVAL} - invalid argument
\item \texttt{PICO$\_$ERR$\_$ENOTCONN} - the socket is not connected
\item \texttt{PICO$\_$ERR$\_$EADDRNOTAVAIL} - address not available
\item \texttt{PICO$\_$ERR$\_$EHOSTUNREACH} - host is unreachable
\item \texttt{PICO$\_$ERR$\_$ENOMEM} - not enough space
\item \texttt{PICO$\_$ERR$\_$EAGAIN} - resource temporarily unavailable
\end{itemize}

\subsubsection*{Example}
\begin{verbatim}
bytesRcvd = pico_socket_send(sk_tcp, buf, bufLen);
\end{verbatim}


\subsection{pico$\_$socket$\_$recv}

\subsubsection*{Description}
This function directly calls the \texttt{pico$\_$socket$\_$recvfrom} function.

\subsubsection*{Function prototype}
\begin{verbatim}
int pico_socket_recv(struct pico_socket *s, void *buf, int len);
\end{verbatim}

\subsubsection*{Parameters}
\begin{itemize}[noitemsep]
\item \texttt{s} - Pointer to socket of type \texttt{struct pico$\_$socket}
\item \texttt{buf} - Void pointer to the start of the buffer
\item \texttt{len} - Maximum allowed length for the data to be stored in the buffer \texttt{buf}
\end{itemize}

\subsubsection*{Return value}
On success, this call returns an integer representing the number of bytes read
from the socket. On error, -1 is returned, and \texttt{pico$\_$err} is set appropriately.

\subsubsection*{Errors}
\begin{itemize}[noitemsep]
\item \texttt{PICO$\_$ERR$\_$EINVAL} - invalid argument
\item \texttt{PICO$\_$ERR$\_$ESHUTDOWN} - cannot read after transport endpoint shutdown
\item \texttt{PICO$\_$ERR$\_$EADDRNOTAVAIL} - address not available
\end{itemize}

\subsubsection*{Example}
\begin{verbatim}
bytesRcvd = pico_socket_recv(sk_tcp, buf, bufLen);
\end{verbatim}


\subsection{pico$\_$socket$\_$bind}

\subsubsection*{Description}
This function binds a local IP-address and port to the specified socket.

\subsubsection*{Function prototype}
\begin{verbatim}
int pico_socket_bind(struct pico_socket *s, void *local_addr, uint16_t *port);
\end{verbatim}


\subsubsection*{Parameters}
\begin{itemize}[noitemsep]
\item \texttt{s} - Pointer to socket of type \texttt{struct pico$\_$socket}
\item \texttt{local$\_$addr} - Void pointer to the local IP-address
\item \texttt{port} - Local portnumber to bind with the socket
\end{itemize}

\subsubsection*{Return value}
On success, this call returns 0 after a succesfull bind.
On error, -1 is returned, and \texttt{pico$\_$err} is set appropriately.

\subsubsection*{Errors}
\begin{itemize}[noitemsep]
\item \texttt{PICO$\_$ERR$\_$EINVAL} - invalid argument
\item \texttt{PICO$\_$ERR$\_$ENOMEM} - not enough space
\item \texttt{PICO$\_$ERR$\_$ENXIO} - no such device or address
\end{itemize}

\subsubsection*{Example}
\begin{verbatim}
errMsg = pico_socket_bind(sk_tcp, &sockaddr4->addr, &sockaddr4->port);
\end{verbatim}

\subsection{pico$\_$socket$\_$getname}

\subsubsection*{Description}
This function returns the local IP-address and port previously bound to the specified socket.

\subsubsection*{Function prototype}
\begin{verbatim}
int pico_socket_getname(struct pico_socket *s, void *local_addr, uint16_t *port, 
                            uint16_t *proto);
\end{verbatim}


\subsubsection*{Parameters}
\begin{itemize}[noitemsep]
\item \texttt{s} - Pointer to socket of type \texttt{struct pico$\_$socket}
\item \texttt{local$\_$addr} - Address (IPv4 or IPv6) previously associated to this socket
\item \texttt{port} - Local portnumber associated to the socket
\item \texttt{proto} - Proto of the address returned in the \texttt{local$\_$addr} field. Can be either \texttt{PICO$\_$PROTO$\_$IPV4} or \texttt{PICO$\_$PROTO$\_$IPV6}
\end{itemize}

\subsubsection*{Return value}
On success, this call returns 0 and populates the three fields {local$\_$addr} \texttt{port} and \texttt{proto} accordingly.
On error, -1 is returned, and \texttt{pico$\_$err} is set appropriately.

\subsubsection*{Errors}
\begin{itemize}[noitemsep]
\item \texttt{PICO$\_$ERR$\_$EINVAL} - invalid argument(s) provided
\end{itemize}

\subsubsection*{Example}
\begin{verbatim}
errMsg = pico_socket_getname(sk_tcp, address, &port, &proto);
if (errMsg == 0) {
    if (proto == PICO_PROTO_IPV4)
        addr4 = (struct pico_ip4 *)address;
    else
        addr6 = (struct pico_ip6 *)address;
}
\end{verbatim}

\subsection{pico$\_$socket$\_$getpeername}

\subsubsection*{Description}
This function returns the IP-address of the remote peer connected to the specified socket.

\subsubsection*{Function prototype}
\begin{verbatim}
int pico_socket_getpeername(struct pico_socket *s, void *remote_addr, uint16_t *port, 
                            uint16_t *proto);
\end{verbatim}


\subsubsection*{Parameters}
\begin{itemize}[noitemsep]
\item \texttt{s} - Pointer to socket of type \texttt{struct pico$\_$socket}
\item \texttt{remote$\_$addr} - Address (IPv4 or IPv6) associated to the socket remote endpoint
\item \texttt{port} - Local portnumber associated to the socket
\item \texttt{proto} - Proto of the address returned in the \texttt{local$\_$addr} field. Can be either \texttt{PICO$\_$PROTO$\_$IPV4} or \texttt{PICO$\_$PROTO$\_$IPV6}
\end{itemize}

\subsubsection*{Return value}
On success, this call returns 0 and populates the three fields {local$\_$addr} \texttt{port} and \texttt{proto} accordingly.
On error, -1 is returned, and \texttt{pico$\_$err} is set appropriately.

\subsubsection*{Errors}
\begin{itemize}[noitemsep]
\item \texttt{PICO$\_$ERR$\_$EINVAL} - invalid argument(s) provided
\item \texttt{PICO$\_$ERR$\_$ENOTCONN} - the socket is not connected to any peer
\end{itemize}

\subsubsection*{Example}
\begin{verbatim}
errMsg = pico_socket_getpeername(sk_tcp, address, &port, &proto);
if (errMsg == 0) {
    if (proto == PICO_PROTO_IPV4)
        addr4 = (struct pico_ip4 *)address;
    else
        addr6 = (struct pico_ip6 *)address;
}
\end{verbatim}


\subsection{pico$\_$socket$\_$connect}

\subsubsection*{Description}
This function connects a local socket to a remote socket of a server that is listening, or permanently associate a remote UDP peer as default receiver for any further outgoing traffic through this socket.

\subsubsection*{Function prototype}
\begin{verbatim}
int pico_socket_connect(struct pico_socket *s, void *srv_addr,
uint16_t remote_port);
\end{verbatim}


\subsubsection*{Parameters}
\begin{itemize}[noitemsep]
\item \texttt{s} - Pointer to socket of type \texttt{struct pico$\_$socket}
\item \texttt{srv$\_$addr} - Void pointer to the remote IP-address to connect to
\item \texttt{remote$\_$port} - Remote port number on which the socket will be connected to
\end{itemize} 

\subsubsection*{Return value}
On success, this call returns 0 after a succesfull connect.
On error, -1 is returned, and \texttt{pico$\_$err} is set appropriately.

\subsubsection*{Errors}
\begin{itemize}[noitemsep]
\item \texttt{PICO$\_$ERR$\_$EPROTONOSUPPORT} - protocol not supported
\item \texttt{PICO$\_$ERR$\_$EINVAL} - invalid argument
\item \texttt{PICO$\_$ERR$\_$EHOSTUNREACH} - host is unreachable 
\end{itemize}

\subsubsection*{Example}
\begin{verbatim}
errMsg = pico_socket_connect(sk_tcp, &sockaddr4->addr, sockaddr4->port);
\end{verbatim}


\subsection{pico$\_$socket$\_$listen}

\subsubsection*{Description}
A server can use this function when a socket is opened and bound to start listening to it.

\subsubsection*{Function prototype}
\begin{verbatim}
int pico_socket_listen(struct pico_socket *s, int backlog);
\end{verbatim}


\subsubsection*{Parameters}
\begin{itemize}[noitemsep]
\item \texttt{s} - Pointer to socket of type \texttt{struct pico$\_$socket}
\item \texttt{backlog} - Maximum connection requests
\end{itemize}

\subsubsection*{Return value}
On success, this call returns 0 after a succesfull listen start.
On error, -1 is returned, and \texttt{pico$\_$err} is set appropriately. 

\subsubsection*{Errors}
\begin{itemize}[noitemsep]
\item \texttt{PICO$\_$ERR$\_$EINVAL} - invalid argument
\item \texttt{PICO$\_$ERR$\_$EISCONN} - socket is connected
\end{itemize}

\subsubsection*{Example}
\begin{verbatim}
errMsg = pico_socket_listen(sk_tcp, 3);
\end{verbatim}


\subsection{pico$\_$socket$\_$accept}

\subsubsection*{Description}
When a server is listening on a socket and the client is trying to connect.
The server on his side will wakeup and acknowledge the connection by calling the this function.

\subsubsection*{Function prototype}
\begin{verbatim}
struct pico_socket *pico_socket_accept(struct pico_socket *s, void *orig,
uint16_t *local_port);
\end{verbatim}

\subsubsection*{Parameters}
\begin{itemize}[noitemsep]
\item \texttt{s} - Pointer to socket of type \texttt{struct pico$\_$socket}
\item \texttt{orig} - Pointer to the origin of the IPv4/IPv6 frame header
\item \texttt{local$\_$port} - Portnumber of the local socket (pointer)
\end{itemize}

\subsubsection*{Return value}
On success, this call returns the pointer to a \texttt{struct pico$\_$socket} that
represents the client thas was just connected. Also \texttt{orig} will contain the requesting
IP-address and \texttt{remote$\_$port} will contain the portnumber of the requesting socket.
On error, \texttt{NULL} is returned, and \texttt{pico$\_$err} is set appropriately.

\subsubsection*{Errors}
\begin{itemize}[noitemsep]
\item \texttt{PICO$\_$ERR$\_$EINVAL} - invalid argument
\item \texttt{PICO$\_$ERR$\_$EAGAIN} - resource temporarily unavailable
\end{itemize}

\subsubsection*{Example}
\begin{verbatim}
client = pico_socket_accept(sk_tcp, &peer, &port);
\end{verbatim}


\subsection{pico$\_$socket$\_$shutdown}

\subsubsection*{Description}
Used by the \texttt{pico$\_$socket$\_$close} function to shutdown read and write mode for
the specified socket. With this function one can close a socket for reading
and/or writing.

\subsubsection*{Function prototype}
\begin{verbatim}
int pico_socket_shutdown(struct pico_socket *s, int mode);
\end{verbatim}

\subsubsection*{Parameters}
\begin{itemize}[noitemsep]
\item \texttt{s} - Pointer to socket of type \texttt{struct pico$\_$socket}
\item \texttt{mode} - \texttt{PICO$\_$SHUT$\_$RDWR}, \texttt{PICO$\_$SHUT$\_$WR}, \texttt{PICO$\_$SHUT$\_$RD}
\end{itemize}

\subsubsection*{Return value}
On success, this call returns 0 after a succesfull socket shutdown.
On error, -1 is returned, and \texttt{pico$\_$err} is set appropriately.

\subsubsection*{Errors}
\begin{itemize}[noitemsep]
\item \texttt{PICO$\_$ERR$\_$EINVAL} - invalid argument
\end{itemize}

\subsubsection*{Example}
\begin{verbatim}
errMsg = pico_socket_shutdown(s, PICO_SHUT_RDWR);
\end{verbatim}


\subsection{pico$\_$socket$\_$close}

\subsubsection*{Description}
Function used on application level to close a socket. Always closes read and write connection.

\subsubsection*{Function prototype}
\begin{verbatim}
int pico_socket_close(struct pico_socket *s);
\end{verbatim}

\subsubsection*{Parameters}
\begin{itemize}[noitemsep]
\item \texttt{s} - Pointer to socket of type \texttt{struct pico$\_$socket}
\end{itemize}

\subsubsection*{Return value}
On success, this call returns 0 after a succesfull socket shutdown.
On error, -1 is returned, and \texttt{pico$\_$err} is set appropriately.

\subsubsection*{Errors}
\begin{itemize}[noitemsep]
\item \texttt{PICO$\_$ERR$\_$EINVAL} - invalid argument
\end{itemize}

\subsubsection*{Example}
\begin{verbatim}
errMsg = pico_socket_close(sk_tcp);
\end{verbatim}



\subsection{pico$\_$socket$\_$setoption}

\subsubsection*{Description}
Function used to set socket options.

\subsubsection*{Function prototype}
\begin{verbatim}
int pico_socket_setoption(struct pico_socket *s, int option, void *value);
\end{verbatim}

\subsubsection*{Parameters}
\begin{itemize}[noitemsep]
\item \texttt{s} - Pointer to socket of type \texttt{struct pico$\_$socket}
\item \texttt{option} - Option to be set (see further for all options)
\item \texttt{value} - Value of option (void pointer)
\end{itemize}

\subsubsection*{Available socket options}
\begin{itemize}[noitemsep]
\item \texttt{PICO$\_$TCP$\_$NODELAY} - Disables/enables the Nagle algorithm (TCP Only). 
\item \texttt{PICO$\_$SOCKET$\_$OPT$\_$KEEPCNT} - Set number of probes for TCP keepalive
\item \texttt{PICO$\_$SOCKET$\_$OPT$\_$KEEPIDLE} - Set timeout value for TCP keepalive probes (in ms)
\item \texttt{PICO$\_$SOCKET$\_$OPT$\_$KEEPINTVL} - Set interval between TCP keepalive retries in case of no reply (in ms)
\item \texttt{PICO$\_$SOCKET$\_$OPT$\_$LINGER} - Set linger time for TCP TIME$\_$WAIT state (in ms)
\item \texttt{PICO$\_$SOCKET$\_$OPT$\_$RCVBUF} - Set receive buffer size for the socket
\item \texttt{PICO$\_$SOCKET$\_$OPT$\_$RCVBUF} - Set receive buffer size for the socket
\item \texttt{PICO$\_$SOCKET$\_$OPT$\_$RCVBUF} - Set receive buffer size for the socket
\item \texttt{PICO$\_$SOCKET$\_$OPT$\_$SNDBUF} - Set send buffer size for the socket 
\item \texttt{PICO$\_$IP$\_$MULTICAST$\_$IF} - (Not supported) Set link multicast datagrams are sent from, default is first added link
\item \texttt{PICO$\_$IP$\_$MULTICAST$\_$TTL} - Set TTL (0-255) of multicast datagrams, default is 1
\item \texttt{PICO$\_$IP$\_$MULTICAST$\_$LOOP} - Specifies if a copy of an outgoing multicast datagram is looped back as long as it is a member of the multicast group, default is enabled
\item \texttt{PICO$\_$IP$\_$ADD$\_$MEMBERSHIP} - Join the multicast group specified
\item \texttt{PICO$\_$IP$\_$DROP$\_$MEMBERSHIP} - Leave the multicast group specified
\end{itemize}

\subsubsection*{Return value}
On success, this call returns 0 after a succesfull setting of socket option.
On error, -1 is returned, and \texttt{pico$\_$err} is set appropriately.

\subsubsection*{Errors}
\begin{itemize}[noitemsep]
\item \texttt{PICO$\_$ERR$\_$EINVAL} - invalid argument
\end{itemize}

\subsubsection*{Example}
\begin{verbatim}
ret = pico_socket_setoption(sk_tcp, PICO_TCP_NODELAY, NULL);

uint8_t ttl = 2;
ret = pico_socket_setoption(sk_udp, PICO_IP_MULTICAST_TTL, &ttl);

uint8_t loop = 0;
ret = pico_socket_setoption(sk_udp, PICO_IP_MULTICAST_LOOP, &loop);

struct pico_ip4 inaddr_dst, inaddr_link;
struct pico_ip_mreq mreq = {{0},{0}};
pico_string_to_ipv4("224.7.7.7", &inaddr_dst.addr);
pico_string_to_ipv4("192.168.0.2", &inaddr_link.addr);
mreq.mcast_group_addr = inaddr_dst;
mreq.mcast_link_addr = inaddr_link;
ret = pico_socket_setoption(sk_udp, PICO_IP_ADD_MEMBERSHIP, &mreq);
ret = pico_socket_setoption(sk_udp, PICO_IP_DROP_MEMBERSHIP, &mreq)
\end{verbatim}


\subsection{pico$\_$socket$\_$getoption}

\subsubsection*{Description}
Function used to get socket options.

\subsubsection*{Function prototype}
\begin{verbatim}
int pico_socket_getoption(struct pico_socket *s, int option, void *value);
\end{verbatim}

\subsubsection*{Parameters}
\begin{itemize}[noitemsep]
\item \texttt{s} - Pointer to socket of type \texttt{struct pico$\_$socket}
\item \texttt{option} - Option to be set (see further for all options)
\item \texttt{value} - Value of option (void pointer)
\end{itemize}

\subsubsection*{Available socket options}
\begin{itemize}[noitemsep]
\item \texttt{PICO$\_$TCP$\_$NODELAY} - Nagle algorithm, \texttt{value} casted to \texttt{(int *)} (0 = disabled, 1 = enabled)
\item \texttt{PICO$\_$SOCKET$\_$OPT$\_$RCVBUF} - Read current receive buffer size for the socket
\item \texttt{PICO$\_$SOCKET$\_$OPT$\_$SNDBUF} - Read current receive buffer size for the socket
\item \texttt{PICO$\_$IP$\_$MULTICAST$\_$IF} - (Not supported) Link multicast datagrams are sent from
\item \texttt{PICO$\_$IP$\_$MULTICAST$\_$TTL} - TTL (0-255) of multicast datagrams
\item \texttt{PICO$\_$IP$\_$MULTICAST$\_$LOOP} - Loop back a copy of an outgoing multicast datagram, as long as it is a member of the multicast group, or not.
\end{itemize}

\subsubsection*{Return value}
On success, this call returns 0 after a succesfull getting of socket option. The value of
the option is written to \texttt{value}.
On error, -1 is returned, and \texttt{pico$\_$err} is set appropriately.

\subsubsection*{Errors}
\begin{itemize}[noitemsep]
\item \texttt{PICO$\_$ERR$\_$EINVAL} - invalid argument
\end{itemize}

\subsubsection*{Example}
\begin{verbatim}
ret = pico_socket_getoption(sk_tcp, PICO_TCP_NODELAY, &stat);

uint8_t ttl = 0;
ret = pico_socket_getoption(sk_udp, PICO_IP_MULTICAST_TTL, &ttl);

uint8_t loop = 0;
ret = pico_socket_getoption(sk_udp, PICO_IP_MULTICAST_LOOP, &loop);
\end{verbatim}
