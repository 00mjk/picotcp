\section{TFTP}

% Short description/overview of module functions
This module provides support for Trivial File Transfer Protocol (TFTP).
The support includes client and server implementation, and allows one transfer at a time.

Flows must be split up into TFTP blocks on the sender side, and reassembled from block len
on the receiving side. Please note that a block whose size is less than the block size indicates
the end of the transfer. 

To indicate the end of a transfer where the content is aligned with the block size, an additional  
transmission of zero bytes must follow the flow.


\subsection{pico\_tftp\_listen}

\subsubsection*{Description}
Start up a TFTP server listening for GET/PUT requests on the given port.
The function pointer passed as callback in the \texttt{cb} argument will be invoked upon a new
transfer request received from the network, and the call will pass the information about:
\begin{itemize}[noitemsep]
\item The address of the remote peer asking for a transfer
\item The remote port of the peer
\item The type of transfer requested, via the \texttt{opcode} parameter being either \texttt{PICO$\_$TFTP$\_$RRQ} or \texttt{PICO$\_$TFTP$\_$WRQ}, for get or put requests respectively.
\end{itemize}


\subsubsection*{Function prototype}
\begin{verbatim}
int pico_tftp_listen(uint16_t family, int (*cb)(union pico_address *addr, uint16_t port, uint16_t opcode, char *filename));
\end{verbatim}

\subsubsection*{Parameters}
\begin{itemize}[noitemsep]
\item \texttt{family} - The chosen socket family. Accepted values are \texttt{PICO$\_$PROTO$\_$IPV4} and \texttt{PICO$\_$PROTO$\_$IPV6}
\item \texttt{cb} - a pointer to the callback function, defined by the user, that will be called upon a new transfer request.
\end{itemize}

\subsubsection*{Errors}
In case of failure, -1 is returned, and the value of pico$\_$err
is set accordingly.

\subsubsection*{Example}

\begin{verbatim}
/* Example of a TFTP listening service callback */

int tftp_listen_cb(union pico_address *addr, uint16_t port, uint16_t opcode, char *filename)
{
    printf("TFTP listen callback.\n");
    if (opcode == PICO_TFTP_RRQ) {
        printf("Received TFTP get request for %s\n", filename);
        if(pico_tftp_start_tx(addr, port, PICO_PROTO_IPV4, filename, cb_tftp_tx) < 0) {
            fprintf(stderr, "TFTP: Error in initialization\n");
            exit(1);
        }
    } else if (opcode == PICO_TFTP_WRQ) {
        printf("Received TFTP put request for %s\n", filename);
        if(pico_tftp_start_rx(addr, port, PICO_PROTO_IPV4, filename, cb_tftp) < 0) {
            fprintf(stderr, "TFTP: Error in initialization\n");
            exit(1);
        }
    } else {
        printf ("Received invalid TFTP request %d\n", opcode);
        return -1;
    }
    return 0;
}

\end{verbatim}


\subsection{pico\_tftp\_start\_tx}

\subsubsection*{Description}
Start a TFTP transfer. The action can be unsolicited (client PUT operation) or solicited (server responding to a GET request).
In either case, the transfer will happen one block at a time, and the callback provided by the user will be called to notify the acknowledgement for the successful
transfer of the last block, or whenever an error occurs. Any error during the TFTP transfer will cancel the transfer itself.

\subsubsection*{Function prototype}
\begin{verbatim}
int pico_tftp_start_tx(union pico_address *a, uint16_t port, uint16_t family,
        char *filename, int (*user_cb)(uint16_t err, uint8_t *block, uint32_t len));
\end{verbatim}

\subsubsection*{Parameters}
\begin{itemize}[noitemsep]
\item \texttt{a} - The address of the peer to be contacted. In case of a solicited transfer, it must match the address where the request came from.
\item \texttt{port} - The port on the remote peer
\item \texttt{family} - The chosen socket family. Accepted values are \texttt{PICO$\_$PROTO$\_$IPV4} and \texttt{PICO$\_$PROTO$\_$IPV6}
\item \texttt{filename} - The name of the file to be transfered. In case of solicited transfer, it must match the filename provided during the request
\item \texttt{user\_cb} - The callback provided by the user to be called upon each block transfer, or in case of error.
\end{itemize}

\subsubsection*{Return value}
In case of success, 0. In case of failure, -1 is returned and pico$\_$err is set accordingly.

%\subsubsection*{Example}

\subsection{pico\_tftp\_send}
\subsubsection*{Description}
Send the next block during an active TFTP transfer. This is ideally called every time the user callback is triggered by the protocol, indicating that the transfer of the last block has been acknowledged. The user should not call this function unless it's solicited by the protocol during an active transmit session.


\subsubsection*{Function prototype}
\begin{verbatim}
int pico_tftp_send(const uint8_t *data, int len);
\end{verbatim}


\subsubsection*{Parameters}
\begin{itemize}[noitemsep]
\item \texttt{data} - the content of the block to be transferred.
\item \texttt{len} - the size of the buffer being transmitted. If < BLOCKSIZE, the transfer is concluded. In order to terminate a transfer where the content is aligned to BLOCKSIZE, a zero-sized \texttt{pico\_tftp\_send} must be called at the end of the transfer.
\end{itemize}

\subsubsection*{Return value}
In case of success, the number of bytes transmitted is returned. In case of failure, -1 is returned and pico$\_$err is set accordingly.


\subsection{pico\_tftp\_start\_rx}

\subsubsection*{Description}
Start a TFTP transfer. The action can be unsolicited (client GET operation) or solicited (server responding to a PUT request).
In either case, the transfer will happen one block at a time, and the callback provided by the user will be called upon successful
transfer of a block, whose content can be directly accessed via the \texttt{block} field,  or whenever an error occurs. Any error during the TFTP transfer will cancel the transfer itself.

\subsubsection*{Function prototype}
\begin{verbatim}
int pico_tftp_start_rx(union pico_address *a, uint16_t port, uint16_t family,
        char *filename, int (*user_cb)(uint16_t err, uint8_t *block, uint32_t len));
\end{verbatim}

\subsubsection*{Parameters}
\begin{itemize}[noitemsep]
\item \texttt{a} - The address of the peer to be contacted. In case of a solicited transfer, it must match the address where the request came from.
\item \texttt{port} - The port on the remote peer
\item \texttt{family} - The chosen socket family. Accepted values are \texttt{PICO$\_$PROTO$\_$IPV4} and \texttt{PICO$\_$PROTO$\_$IPV6}
\item \texttt{filename} - The name of the file to be transfered. In case of solicited transfer, it must match the filename provided during the request
\item \texttt{user\_cb} - The callback provided by the user to be called upon each block transfer, or in case of error. This is the call where the incoming data is processed. When len is less than the block size, the transfer is over.
\end{itemize}

\subsubsection*{Return value}
In case of success, 0. In case of failure, -1 is returned and pico$\_$err is set accordingly.

%\subsubsection*{Example}


\subsection{pico\_tftp\_close}
\subsubsection*{Description}
Terminate any ongoing transmission (also notifying the other endpoint), and closes the listening socket (if in server mode).
After a call to this function, the transmission is terminated, and the server won't be able to accept new connections.

\subsubsection*{Function prototype}
\begin{verbatim}
void pico_tftp_close(void)
\end{verbatim}



