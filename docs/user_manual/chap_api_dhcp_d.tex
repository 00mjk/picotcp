\section{DHCP server}

% Short description/overview of module functions


\subsection{pico\_dhcp\_server\_initiate}

\subsubsection*{Description}
This function starts a simple DHCP server. 

\subsubsection*{Function prototype}
\texttt{int pico\_dhcp\_server\_initiate(struct pico\_dhcpd\_settings* settings);}

\subsubsection*{Parameters}
\begin{itemize}[noitemsep]
\item \texttt{settings} - a pointer to a struct \texttt{pico\_dhcpd\_settings}, in which the following members matter to the user : 
\begin{itemize}[noitemsep]
\item \texttt{struct pico\_device *dev} - a pointer to the device on which the dhcp server must operate
\item \texttt{struct pico\_ip4 my\_ip} - the IP assigned to the server
\item \texttt{struct pico\_ip4 netmask} - the netmask the server must advertise
\item \texttt{uint32\_t pool\_start} - the first IP address that may be assigned
\item \texttt{uint32\_t pool\_end} - the last IP address that may be assigned
\item \texttt{uint32\_t lease\_time} - the advertised lease time in seconds
\end{itemize}
\end{itemize}

\subsubsection*{Return value}
On successful startup of the dhcp server, 0 is returned.
On error, -1 is returned, and \texttt{pico$\_$err} is set appropriately.

\subsubsection*{Errors}
\begin{itemize}[noitemsep]
%everything from :
%pico_socket_open
\item PICO$\_$ERR$\_$EPROTONOSUPPORT - protocol not supported
\item PICO$\_$ERR$\_$ENETUNREACH - network unreachable 
%pico_socket_bind
\item PICO$\_$ERR$\_$EINVAL - invalid argument
\item PICO$\_$ERR$\_$ENXIO - no such device or address
\end{itemize}

\subsubsection*{Example}
\begin{verbatim}
struct pico_dhcpd_settings s = {0};
s.dev = ethernet;
s.my_ip.addr = long_be(0x0a280003);
s.netmask.addr = long_be(0xffffff00);
s.pool_start = (s.my_ip.addr & long_be(0xffffff00)) | long_be(0x00000064);
s.pool_end = (s.my_ip.addr & long_be(0xffffff00)) | long_be(0x000000ff);
pico_dhcp_server_initiate(&s);
\end{verbatim}


