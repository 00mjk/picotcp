% TASS Belgium NV
%
% PicoTCP User Documentation main file
% ====================================

%\documentclass[11pt, a4paper, openright]{paper}



% Philippe Mariman
%
% LAYOUT TEXT
% ===========

%\documentclass[11pt, a4paper, openright,oneside]{book}
\documentclass[11pt, a4paper,oneside]{report}

\usepackage[english]{babel}
\usepackage[latin1]{inputenc}
%\usepackage[T1]{fontenc}
\usepackage{graphicx}
\usepackage{natbib}
%\usepackage{hyperref}
\usepackage[hang,flushmargin]{footmisc} 

\usepackage{fullpage}
\parskip 4pt		% sets spacing between paragraphs
\parindent 0pt		% sets leading space for paragraphs

\makeatletter
\renewcommand{\@makechapterhead}[1]{%
%\vspace*{50 pt}%
{\setlength{\parindent}{0pt} \raggedright \normalfont
\bfseries\Huge\thechapter.\ #1
\par\nobreak\vspace{40 pt}}}
\makeatother

% TEMPS

%\usepackage{tikz}
%\usepackage[latin1]{inputenc}
%\usepackage{graphicx}
%\usepackage[hang,flushmargin]{footmisc}
%\usepackage{pdfpages}
%\usepackage{tabularx}
%\usepackage{lscape}
%\usepackage{longtable}
%\usepackage{verbatim}
%\usepackage{moreverb}
%\usepackage{listings}
%\usepackage{draftcopy}
%\usepackage{hyperref}
\usepackage{longtable}

%% to print watermark
% \usepackage{draftwatermark}
% \SetWatermarkText{TASS Confidential}
% \SetWatermarkScale{3}
% \SetWatermarkLightness{0.9}

% to adjust the space between titles and text
\usepackage[compact]{titlesec}
\titlespacing{\section}{0pt}{*5}{*2}
\titlespacing{\subsection}{0pt}{*4}{*1}
\titlespacing{\subsubsection}{0pt}{*1}{*0}

% to minimize space between list items
\usepackage{enumitem}

% To use hyperlinks
\usepackage{hyperref}
% limit toc depth until sections
\setcounter{tocdepth}{1}


\begin{document}

\title{PicoTCP User Documentation}
\author{Copyright \copyright 2013 TASS Belgium NV. All right reserved.}
\maketitle
\date{\today}
\maketitle

\thispagestyle{empty}

Disclaimer
This document is distributed under the terms of Creative Commons CC BY-ND 3.0.
You are free to share unmodified copies of this document, as long as the copyright
statement is kept. The full license text is available 
\href{http://creativecommons.org/licenses/by-nd/3.0/} {here}


\pagenumbering{arabic}

\selectlanguage{english}

\tableofcontents

%\chapter{Introduction}
%\label{chap:intro}
%PicoTCP is a complete TCP/IP stack designed for embedded devices and
intended to be run on several different architectures and networking
hardware. The architecture of the stack allows to select the features
needed for any particular use easily, taking into account the sizing and
the performance of the platform where the code runs. Even if it is designed
to respect size and performance constraint, the chosen approach is to
respect the latest standards in the telecommunications research, including
the latest proposals, in order to respect the highest standards for
today's inter-networking communications. PicoTCP is distributed in the form
of a library to be integrated with application and form a combination for
any hardware-specific firmware.


The man characteristic of the library are the following:
\begin{itemize}
\item \textbf{Modularity} Each component of the stack is deployed in a
separate module, allowing the user to select at compile time what needs to
be included for any specific platform, depending on the particular use case.
We know that saving memory and resources is often mission-critical for a
project, and the approach used in the PicoTCP is fully oriented to saving
up to the last byte of memory.
\item \textbf{Code Quality} Every single component that is added to the
stack goes through a complete set of validation tests. All the newly
introduced code gets scanned and proof-checked by three separate levels of
quality enforcement. The process related to the validation of the code is
one of the largest task for the engineering team. The top-down design of a
new module has to pass the review of our senior architects, because it has
to comply with the general guidelines. The development of the smaller
components is done in a test-driven fashion, where each function call has
its own unit test. Finally, functional non-regression tests are performed
when the feature development is complete, and all the tests are automatically
scheduled to run several times per day to check for functional regressions.
\item \textbf{Adherence to the standards} The design of the protocols
included in the stack are done following step by step the guidelines
provided by the International Engineering Task Force (IETF) with regards to
inter-networking communication. A strong adherence to the standards guarantees a
good integration with all the existing TCP/IP stacks, when communicating
both with other embedded devices and with the PC/server world.
\item \textbf{Features} Our engineering team is never satisfied until all
the corners of the protocols specifications are covered in the code.
A fully-featured protocol implementation including all those non-mandatory
features means better data-transfer performances, coverage of rare/unique 
network scenarios and topologies and a better integration with all types of
networking hardware devices.
\item \textbf{Transparency} The availability of the source code to the Free
Software community is an important added value for PicoTCP. Our programmers
are proud of the aestethic of their code, and they show it with no
hesitation to the attention of the rest of the world.
PicoTCP constantly receives peer-reviews and constructive comments on the
design and the development choices from the academic world and from several
hundreds of hobbists and professionals who read the code. We strongly
believe that software is not about keeping things secret: whenever one is
convinced by the quality of their work, there is absolutely nothing to hide.
\item \textbf{Simplicity} The APIs provided to access the library
facilities, both from the applications and from the device drivers, are
small and well documented. The goal of such a library must be to facilitate
the integration with the surroundings and minimize the time used to combine
the stack with existing code. The support required to port to a new
architecture is so small that is reduced to a set of macros defined in a
header file specific for the platform.
\end{itemize}




\chapter{Overview}
\label{chap:overview}
PicoTCP is a complete TCP/IP stack designed for embedded devices and
intended to be run on several different architectures and networking
hardware. The architecture of the stack allows to select the features
needed for any particular use easily, taking into account the sizing and
the performance of the platform where the code runs. Even if it is designed
to respect size and performance constraint, the chosen approach is to
respect the latest standards in the telecommunications research, including
the latest proposals, in order to respect the highest standards for
today's inter-networking communications. PicoTCP is distributed in the form
of a library to be integrated with application and form a combination for
any hardware-specific firmware.


The man characteristic of the library are the following:
\begin{itemize}
\item \textbf{Modularity} Each component of the stack is deployed in a
separate module, allowing the user to select at compile time what needs to
be included for any specific platform, depending on the particular use case.
We know that saving memory and resources is often mission-critical for a
project, and the approach used in the PicoTCP is fully oriented to saving
up to the last byte of memory.
\item \textbf{Code Quality} Every single component that is added to the
stack goes through a complete set of validation tests. All the newly
introduced code gets scanned and proof-checked by three separate levels of
quality enforcement. The process related to the validation of the code is
one of the largest task for the engineering team. The top-down design of a
new module has to pass the review of our senior architects, because it has
to comply with the general guidelines. The development of the smaller
components is done in a test-driven fashion, where each function call has
its own unit test. Finally, functional non-regression tests are performed
when the feature development is complete, and all the tests are automatically
scheduled to run several times per day to check for functional regressions.
\item \textbf{Adherence to the standards} The design of the protocols
included in the stack are done following step by step the guidelines
provided by the International Engineering Task Force (IETF) with regards to
inter-networking communication. A strong adherence to the standards guarantees a
good integration with all the existing TCP/IP stacks, when communicating
both with other embedded devices and with the PC/server world.
\item \textbf{Features} Our engineering team is never satisfied until all
the corners of the protocols specifications are covered in the code.
A fully-featured protocol implementation including all those non-mandatory
features means better data-transfer performances, coverage of rare/unique 
network scenarios and topologies and a better integration with all types of
networking hardware devices.
\item \textbf{Transparency} The availability of the source code to the Free
Software community is an important added value for PicoTCP. Our programmers
are proud of the aestethic of their code, and they show it with no
hesitation to the attention of the rest of the world.
PicoTCP constantly receives peer-reviews and constructive comments on the
design and the development choices from the academic world and from several
hundreds of hobbists and professionals who read the code. We strongly
believe that software is not about keeping things secret: whenever one is
convinced by the quality of their work, there is absolutely nothing to hide.
\item \textbf{Simplicity} The APIs provided to access the library
facilities, both from the applications and from the device drivers, are
small and well documented. The goal of such a library must be to facilitate
the integration with the surroundings and minimize the time used to combine
the stack with existing code. The support required to port to a new
architecture is so small that is reduced to a set of macros defined in a
header file specific for the platform.
\end{itemize}





\chapter{Usage and platform integration}
\label{chap:usage}
\section{Requirements and Configuration}

PicoTCP is designed to be portable and versatile. Modules can be activated at
compile-time, or excluded from the compilation in order to reduce the build size
or save resources at runtime. This characteristic allows an embedded
application to create different types of appliances, starting from a small
forwarding multi-protocol switch, to fully-featured TCP hosts, supporting
internal applets as well as generic POSIX-compliant socket interfaces.


\section{Supported features}
\begin{itemize}
\item \textbf{Device layer} Facilities for device driver are offered in a simple
					structure and API.
\item \textbf{ARP} The stack can use the "Address Resolution Protocol" to retrieve
					the MAC addresses of other hosts in the network.
\item \textbf{IPv4} The network layer supports the IPv4 network layer
          protocol. An API is provided in order to access all the
          addressing and routing related functionalities.
\item \textbf{ICMP} Also the "Internet Control Message Protocol" is implemented. This
					protocol provides the system to send error messages, do a ping, ...
\item \textbf{NAT} The stack supports "Network Address Translation" to hide addresses
					from internal networks to the outside. The API also supports functions
					for port forwarding.
\item \textbf{multicast sockets} The stack supports multicast (one-to-many)
          sockets and addresses in order to send and receive data to/from
          multicast groups.
\item \textbf{IGMP} As an integration for the multicast features above, IGMP version 2
          is supported to manage the membership to multicast groups.
\item \textbf{UDP} The stack can use the "User Datagram Protocol" as a transport protocol
					for connection-less communication between sockets.
\item \textbf{TCP} The stack supports the connection-oriented "Transport Control Protocol"
					for reliable communications. The TCP implementation is fully
          featured and the most commonly used extensions are included.
\item \textbf{Sockets} The user applications on different host use the socket API to communicate.
					The socket API is based on the latest POSIX (1-2008)
          specifications, while not being fully compliant due to the fact
          that it is designed to run in a single threading unit. Blocking
          functionalities are reproduced via callback triggering as
          described in the socket API documentation.
\item \textbf{DNS client} A small DNS client is provided to resolve an IP address for a given name.
					The API supports setting several DNS servers and a small cache.
\item \textbf{SNTP client} The stack supports synchronizing time over the network using sntp with a precision of at least 10msec. Similar to unix a gettimeofday function is implemented.
\item \textbf{DHCP client} A DHCP client can request an IP lease from a DHCP server to set the IP
					adress of the device.
\item \textbf{DHCP server} Also a small DHCP server is included to hand out IP addresses to hosts
					in the network.
\item \textbf{Linux development and test facilities} The stack is developed entirely on a Linux system.
					Several tools are easily available and/or included to develop and test user applications.
					(tun/tap devices, vde, tcp benchmark test, ...)
\end{itemize}

\section{Enabling modules}
Each module, option and feature included in the code base must be explicitly
enabled by defining a specific PICO$\_$SUPPORT$\_$ preprocessor variable.
If the default Makefile is used to compile PicoTCP, this can be done using
command line options when running make. The syntax required to compile the
protocol in a library (the default Makefile target) is the following:

\texttt{make [MAKE$\_$ARG=VALUE] [...] }

\subsection{Compile-time options}
A few compile-time options can be specified using the command line
arguments of make to modify the result of the build. Global options that
affect the build are the following:

\begin{longtable}{ | l | l | l | p{5cm} | }
\hline
{\bf Argument} &
{\bf Possible values} &
{\bf Default value} &
{\bf Description} 
\\ \hline

DEBUG&
0,1&
1&
When enabled (=1), the resulting library will contain debug symbols. The
size of the library will be much larger than the production build, but it
will be possible to run the stack into a debugger to inspect its behaviour.
When the option is disabled (=0), the library will be optimized for size in
flash, resulting in a smaller binary to be used in production. \\ \hline

PREFIX&
any valid path&
./build&
The target directory where the library and all the objects will be placed
after the compilation. \\ \hline

ENDIAN&
little, big &
little&
Force to build against little-endian or big-endian architecture. \\ \hline

CROSS$\_$COMPILE&
compiler prefix&
&
Use a cross compile prefix when calling the binaries needed to build.
\\ \hline

TCP&
0,1&
1&
Enables the support for Transmission Control Protocol by allowing the usage 
of stream sockets.
\\ \hline

UDP&
0,1&
1&
Enables the support for User Datagram Protocol by allowing the usage 
of datagram sockets.
\\ \hline

IPV4&
0,1&
1&
Enables the support for basic IP networking functionalities. At least one network protocol
is required for most of the features to work, as all types of sockets depend
on the networking layer.
\\ \hline

NAT&
0,1&
1&
Activates the support for network address translation to IPv4.
\\ \hline

ICMP4&
0,1&
1&
Enables the support for control messages over IPv4, (not including the ping functionalities).
\\ \hline

MCAST&
0,1&
1&
If enabled, the support for multicast sockets will be included in the resulting library.
\\ \hline

DEVLOOP&
0,1&
1&
If enabled, a loopback device will be added to the stack, and can be configured to run local traffic.
\\ \hline

PING&
0,1&
1&
When activated, the ping API will be available to test whether the hosts on the network are reachable.
Requires ICMP4 support.
\\ \hline

CRC&
0,1&
0&
If enabled, CRC values are validated at IP and transport layer.
\\ \hline

IPFRAG&
0,1&
1&
Enables the support for fragmentation and reassembly of UDP packets.
\\ \hline

IPFILTER&
0,1&
1&
If enabled, provides basic filtering.
\\ \hline

DNS$\_$CLIENT&
0,1&
1&
This feature is required to resolve host names into IP addresses and vice-versa.
\\ \hline

SNTP$\_$CLIENT&
0,1&
1&
Enables snychronising the local time to a given ntp server.
\\ \hline

DHCP$\_$CLIENT&
0,1&
1&
When activated, it will be possible to get the IP address for network devices automatically, when a DHCP
server is present on the network.
\\ \hline

DHCP$\_$SERVER&
0,1&
1&
If activated, it will be possible to run a small DHCP server to provide addresses for automatic configuration
to the other hosts in the network.
\\ \hline

HTTP$\_$CLIENT&
0,1&
1&
Activates a basic HTTP client.
\\ \hline

HTTP$\_$SERVER&
0,1&
1&
Activates a basic HTTP server.
\\ \hline

\end{longtable}

\subsection{Architecture support}
By default, the stack will be compiled to run in a process on a POSIX
system, e.g. to be linked to a Linux application. To change this behavior
and produce a library linked to a specific board-support package (BSP)
among those supported, it is sufficient to set the command line argument
variable ARCH to a specific value. The architectures supported by the stack
are the following:


\begin{longtable}{ | l | l | p{5cm} | }
\hline
{\bf ARCH keyword} &
{\bf CPU} &
{\bf Reference hardware}
\\ \hline

stm32&
ARM Cortex M4-F&
ST Microelectronics evaluation board "STM32f4 Discovery"
\\ \hline

stellaris&
ARM Cortex LM3S-6965&
Texas Instrument Evaluation Kit "Codesourcery LM3S6965 ETH"
\\ \hline

\end{longtable}



\section{Target requirements}
PicoTCP can run on several different hardware architectures and can be
integrated with virtually any operating system or within a standalone
application. It is possible to run PicoTCP on big-endian as well as
little-endian CPU configurations. PicoTCP uses gcc-specific tags that may
not be compatible with other compilers. The amount of resources needed
may vary depending on the modules that are compiled-in. However, adapting
to a specific hardware platform or for a particular use may require some
integration effort.

\subsection{Porting PicoTCP to a target system}

\begin{center}
\textbf{Warning: ensure that the Board Support Package provided by your
hardware supplier is distributed under the terms of a license compatible
with the PicoTCP license, described in the Appendix of this document.}
\end{center}

PicoTCP relies on a simple set of system-specific calls that must be
implemented externally from the target. Briefly, the interface needed for
the stack to run is composed by:
\begin{itemize}
\item A mechanism to allocate dynamic memory on the system
\item A stable time-source to update its internal counters
\end{itemize} 

For the memory allocation interface, two symbols have to be defined by the system:
\begin{verbatim}
  void *pico_zalloc(int size) - (memory allocation) 
  void pico_free(void *ptr) - (memory release)
\end{verbatim}

\begin{itemize}
\item \texttt{pico$\_$zalloc} Must allocate an object of the given size size in memory
and set the content of the allocated memory to zero. A pointer to the address 0 will
indicate an allocation failure.
\item \texttt{pico$\_$free} Must release the memory assigned to the object previously
allocated at the address ptr.
\end{itemize}


For the time keeping, the following objects must be defined by the system:
\begin{itemize}%[noitemsep]
\item
%\begin{verbatim}
\texttt{static inline unsigned long PICO$\_$TIME(void)}\\
%\end{verbatim}
Returns current time expressed in seconds

\item
%\begin{verbatim}
\texttt{static inline unsigned long PICO$\_$TIME$\_$MS(void)}\\
%\end{verbatim}
Returns current time expressed in milliseconds

\item
%\begin{verbatim}
\texttt{static inline void PICO$\_$IDLE(void)}\\
%\end{verbatim}
Sleep between two consecutive iterations inside the main protocol loop
(e.g. to yield the CPU to some other functionality on the sytem)
\end{itemize}

As an alternative to defining the time-keeping procedure in the asynchronous
functions \texttt{PICO$\_$TIME()} and \texttt{PICO$\_$TIME$\_$MS()}, it is possible
to use an interrupt handler linked to a fixed interval time source, increasing the
volatile global variable \texttt{pico$\_$tick}. If done this way, the two
functions may return the values of \texttt{(pico$\_$tick / 1000)} and
\texttt{pico$\_$tick}, respectively.

Finally, whenever debug information is needed, the system will have to provide a
\texttt{dbg()} function that accepts the same variadic arguments model as a standard \texttt{printf()}.


\subsection{Defining a new architecture support}
If all the above requirements are satisfied, PicoTCP expects those
functions to be mapped to existing code in the BSP of the architecture.
An easy way to do so is by means of a new architecture-specific header
file under the \texttt{include/arch} subdirectory.
Since all the functions above must already be implemented outside the
PicoTCP tree, the library will have to be linked to the system support
library, either during compilation or at at a subsequent stage when the resulting
firmware is being generated. For this reason, a prototype of all the
functions used to implement the functionalities requested by the BSP
must be included from the architecture support header file or incorporated
into the file itself.

For instance, if the BSP for an architecture called "foobar" provides the 
following functions:
\begin{verbatim}
  void *custom_allocate_and_zero(int size);
  void *custom_free(void *mem);
  int print_serial_debug(...);
\end{verbatim}
and an interrupt handler is attached to a time source in order to increment
the \texttt{pico$\_$tick} variable every millisecond, a possible architecture-specific 
file (under \texttt{arch/pico$\_$foobar.h}) should look like the following:

\begin{verbatim}
  /* repeat the prototypes used */
  extern void *custom_allocate_and_zero(int size);
  extern void *custom_free(void *mem);
  extern int print_serial_debug(...);

  #define dbg print_serial_debug
  #define pico_zalloc(x) custom_allocate_and_zero(x)
  #define pico_free(x) custom_free(x)

  static inline unsigned long PICO_TIME(void)
  {
    return pico_tick / 1000;
  }

  static inline unsigned long PICO_TIME_MS(void)
  {
    return pico_tick;
  }

  static inline void PICO_IDLE(void)
  {
    unsigned long tick_now = pico_tick;
    while(tick_now == pico_tick);
  }

\end{verbatim}

Once the architecture-specific file is created, it is time to add the
architecture-specific support to the \texttt{pico$\_$config.h} file, the
same way it is done for the existing architectures, using an additional 
preprocessor elif block:

\begin{verbatim}
#elif defined FOOBAR
#include "arch/pico_foobar.h"
\end{verbatim}

From this point on, it is sufficient to define a preprocessor variable with
the keyword chosen for the architecture, all in capitals (\texttt{FOOBAR} in 
this example case). The final step is to create a block in the main PicoTCP 
makefile that also sets the compiler flags needed to produce objects that are
compatible with and/or optimized for the foobar architecture. Additionally,
this block also contains the definition of the keyword preprocessor macro in
order to have the correct arch-specific header included:

\begin{verbatim}
ifeq ($(ARCH),foobar)
  CFLAGS+=-mcustom-foobar-code -DFOOBAR
endif
\end{verbatim}

To compile for the foobar architecture, it is now sufficient to run

\begin{verbatim}
  make ARCH=foobar
\end{verbatim}


\section{Network devices integration}
Every device driver must define its own interface to communicate with the stack.
This interface is accessed via the \texttt{pico$\_$device} structure. Every device implements
an instance of this structure by populating the following mandatory fields:

\begin{itemize}
\item \textbf{overhead} - A positive integer indicating the amount of bytes required by the
device driver to implement its header. This is used whenever a network layer allocates a new
packet to be sent through this device. If a value is specified here, it will be possible for
the device to seek back in the frame scheduled for sending, and subsequently copy any header
information in front of it. Devices dealing with pure stack frames or subparts of it
(e.g. Ethernet) should have overhead set to 0.

\item The callback \textbf{send} - must be a pointer to a function internally defined in the
device driver module. This function will be called every time a frame must be injected in the
network. The module can implement a generic \textbf{send} function for all the registered devices, as
the device field will be passed as the first argument. The callback prototype is the following:

 \texttt{int (*send)(struct pico$\_$device *self, void *buf, int len);}

If the device can immediately inject the frame at address \texttt{buf} of length \texttt{len},
it returns back to the caller the length of the frame injected. If the device is currently busy,
this function can safely return 0, and the stack will retry the same operation again later.

\item The callback \textbf{poll} - must be a pointer to a function internally defined in the
device driver module. This function will be called periodically by the stack, to request a
synchronization on the incoming frames. The prototype is the following:

  \texttt{int (*poll)(struct pico$\_$device *self, int loop$\_$score);}

The poll function must check if the device is ready to receive frames, and for each frame that
is directed to the stack, it will call the library function \texttt{pico$\_$stack$\_$recv()}.
This function will deliver the received frame to the stack.

The \texttt{loop$\_$score} variable represents the maximum amount of frames that the stack can process
during this call, i.e. the maximum amount of calls to \texttt{pico$\_$stack$\_$recv()} that
can be performed during this iterations. The device driver should loop around the packet
delivery operation and decrease the loop$\_$score by one every time a frame is delivered to
the stack. If during the iteration all the score was used, poll will return 0.

\textbf{NOTE:} The poll function must return \textbf{immediately} and must never block on
hardware-specific operations. If the device is interrupt-driven, the integration will have
to provide a mechanism to defer the reception until the next call back to poll. Calling
\texttt{pico$\_$stack$\_$recv()} is only allowed from inside the \texttt{poll()} callback,
thus a two-halves interface interrupt management design is required, and any memory structure
shared between the two halves must be protected against concurrent access accordingly.

\item The callback \textbf{destroy} - a pointer to a function that deallocates the device
structure itself and frees all the structures that were possibly allocated by the driver 
during device creation.
\end{itemize}

A device driver will have a simple two-functions library API exported in a header file using
the same name, in the modules directory. The two functions to export will be:

\begin{itemize}
\item A \textbf{create} function, accepting any argument required for the internal device configuration,
that returns a pointer to the newly allocated device. The function must allocate the device and
finally call the library function \texttt{pico$\_$device$\_$init()} in order to register the
device into the stack.
The \texttt{pico$\_$device$\_$init()} function accepts the following arguments:

\begin{itemize}
\item the device allocated just before
\item a null-terminated string containing a unique device name for the device to be inserted in
the system (e.g. "eth0")
\item a pointer to an Ethernet address in the form of a previously allocated \texttt{pico$\_$ethdev}
structure, containing the hardware address to be used by the stack for datalink addressing.
If no hardware-specific address is provided to \texttt{pico$\_$device$\_$init()} is provided
(i.e. a \texttt{NULL} pointer is passed), the newly created device will be directly attached
to the network layer and it will have to provide and process valid IP packets without further
encapsulation.
\end{itemize}

\item A destroy routine, accepting the previously allocated device pointer to free all the associated structures.

\end{itemize}

The way to expand the device driver interface is by simply creating a new specific structure
that contains it and thus inherits all the capabilities of the standard structure but also holds
the required hardware-specific information. The three callbacks will always receive a pointer
to the beginning of the \texttt{pico$\_$device} structure, but the memory area that follows
the structure can be used to keep track of the device hardware-specific context.

Naming conventions must be followed for the two functions exposed to the user interface to
create and destroy the device. The functions must be named \texttt{pico$\_$X$\_$create()} and
\texttt{pico$\_$X$\_$destroy()}, where X is the unique name of the device driver.

As an example of a very simple device driver, directly attached to the networking layer using
the valid naming convention for the \textbf{send/poll/create/destroy} interfaces are contained in the source
file \texttt{modules/pico$\_$dev$\_$null.c} and its header \texttt{modules/pico$\_$dev$\_$null.h}.



\chapter{API Documentation}
\label{chap:api_doc}
The following sections will describe the API for picoTCP.
\section{IPv4 functions}

% Short description/overview of module functions

\subsection{pico$\_$ipv4$\_$to$\_$string}

\subsubsection*{Description}
Convert the internet host address IP to a string in IPv4 dotted-decimal notation.
The result is stored in the char array that ipbuf points to. The given IP address argument must be in network order (i.e. 0xC0A80101 becomes 192.168.1.1).

\subsubsection*{Function prototype}
\begin{verbatim}
int pico_ipv4_to_string(char *ipbuf, const uint32_t ip);
\end{verbatim}

\subsubsection*{Parameters}
\begin{itemize}[noitemsep]
\item \texttt{ipbuf} - Char array to store the result in.
\item \texttt{ip} - Internet host address in integer notation.
\end{itemize}

\subsubsection*{Return value}
On success, this call returns 0 if the conversion was successful.
On error, -1 is returned and \texttt{pico$\_$err} is set appropriately.

\subsubsection*{Errors}
\begin{itemize}[noitemsep]
\item \texttt{PICO$\_$ERR$\_$EINVAL} - invalid argument
\end{itemize}

\subsubsection*{Example}
\begin{verbatim}
ret = pico_ipv4_to_string(buf, ip);
\end{verbatim}



\subsection{pico$\_$string$\_$to$\_$ipv4}

\subsubsection*{Description}
Convert the IPv4 dotted-decimal notation into binary form. The result is stored in the
\texttt{int} that IP points to. Little endian or big endian is not taken into account.
The address supplied in \texttt{ipstr} can have one of the following
forms: a.b.c.d, a.b.c or a.b.

\subsubsection*{Function prototype}
\begin{verbatim}
int pico_string_to_ipv4(const char *ipstr, uint32_t *ip); 
\end{verbatim}

\subsubsection*{Parameters}
\begin{itemize}[noitemsep]
\item \texttt{ipstr} - Pointer to the IP string.
\item \texttt{ip} - Int pointer to store the result in.
\end{itemize}

\subsubsection*{Return value}
On success, this call returns 0 if the conversion was successful.
On error, -1 is returned and \texttt{pico$\_$err} is set appropriately.

\subsubsection*{Errors}
\begin{itemize}[noitemsep]
\item \texttt{PICO$\_$ERR$\_$EINVAL} - invalid argument
\end{itemize}

\subsubsection*{Example}
\begin{verbatim}
ret = pico_string_to_ipv4(buf, *ip);
\end{verbatim}


\subsection{pico$\_$ipv4$\_$valid$\_$netmask}

\subsubsection*{Description}
Check if the provided mask if valid.

\subsubsection*{Function prototype}
\begin{verbatim}
int pico_ipv4_valid_netmask(uint32_t mask);
\end{verbatim}

\subsubsection*{Parameters}
\begin{itemize}[noitemsep]
\item \texttt{mask} - The netmask in integer notation.
\end{itemize}

\subsubsection*{Return value}
On success, this call returns the netmask in CIDR notation is returned if the netmask is valid.
On error, -1 is returned and \texttt{pico$\_$err} is set appropriately.

\subsubsection*{Errors}
\begin{itemize}[noitemsep]
\item \texttt{PICO$\_$ERR$\_$EINVAL} - invalid argument
\end{itemize}

\subsubsection*{Example}
\begin{verbatim}
ret = pico_ipv4_valid_netmask(netmask);
\end{verbatim}


\subsection{pico$\_$ipv4$\_$is$\_$unicast}

\subsubsection*{Description}
Check if the provided address is unicast or multicast.

\subsubsection*{Function prototype}
\begin{verbatim}
int pico_ipv4_is_unicast(uint32_t address);
\end{verbatim}

\subsubsection*{Parameters}
\begin{itemize}[noitemsep]
\item \texttt{address} - Internet host address in integer notation.
\end{itemize}

\subsubsection*{Return value}
Returns 1 if unicast, 0 if multicast.

%\subsubsection*{Errors}

\subsubsection*{Example}
\begin{verbatim}
ret = pico_ipv4_is_unicast(address);
\end{verbatim}



\subsection{pico$\_$ipv4$\_$source$\_$find}

\subsubsection*{Description}
Find the source IP for the link associated to the specified destination.
This function will use the currently configured routing table to identify the link that would be used to transmit any traffic directed to the given IP address.

\subsubsection*{Function prototype}
\begin{verbatim}
struct pico_ip4 *pico_ipv4_source_find(struct pico_ip4 *dst);
\end{verbatim}

\subsubsection*{Parameters}
\begin{itemize}[noitemsep]
\item \texttt{address} - Pointer to the destination internet host address as \texttt{struct pico$\_$ip4}.
\end{itemize}

\subsubsection*{Return value}
On success, this call returns the source IP as \texttt{struct pico$\_$ip4}.
If the source can not be found, \texttt{NULL} is returned and \texttt{pico$\_$err} is set appropriately.

\subsubsection*{Errors}
\begin{itemize}[noitemsep]
\item \texttt{PICO$\_$ERR$\_$EINVAL} - invalid argument
\item \texttt{PICO$\_$ERR$\_$EHOSTUNREACH} - host is unreachable
\end{itemize}

\subsubsection*{Example}
\begin{verbatim}
src = pico_ipv4_source_find(dst);
\end{verbatim}




\subsection{pico$\_$ipv4$\_$link$\_$add }

\subsubsection*{Description}
Add a new local device dev inteface, f.e. eth0, with IP address 'address' and netmask 'netmask'. A device may have more than one link configured, i.e. to access multiple networks on the same link.

\subsubsection*{Function prototype}
\begin{verbatim}
int pico_ipv4_link_add(struct pico_device *dev, struct pico_ip4 address,
struct pico_ip4 netmask);
\end{verbatim}

\subsubsection*{Parameters}
\begin{itemize}[noitemsep]
\item \texttt{dev} - Local device.
\item \texttt{address} - Pointer to the internet host address as \texttt{struct pico$\_$ip4}.
\item \texttt{netmask} - Netmask of the address.
\end{itemize}

\subsubsection*{Return value}
On success, this call returns 0.
On error, -1 is returned and \texttt{pico$\_$err} is set appropriately.

\subsubsection*{Errors}
\begin{itemize}[noitemsep]
\item \texttt{PICO$\_$ERR$\_$EINVAL} - invalid argument
\item \texttt{PICO$\_$ERR$\_$ENOMEM} - not enough space
\item \texttt{PICO$\_$ERR$\_$ENETUNREACH} - network unreachable
\item \texttt{PICO$\_$ERR$\_$EHOSTUNREACH} - host is unreachable
\end{itemize}

\subsubsection*{Example}
\begin{verbatim}
ret = pico_ipv4_link_add(dev, address, netmask);
\end{verbatim}



\subsection{pico$\_$ipv4$\_$link$\_$del}

\subsubsection*{Description}
Remove the link associated to the local device that was previously configured, corresponding to the IP address 'address'.

\subsubsection*{Function prototype}
\begin{verbatim}
int pico_ipv4_link_del(struct pico_device *dev, struct pico_ip4 address); 
\end{verbatim}

\subsubsection*{Parameters}
\begin{itemize}[noitemsep]
\item \texttt{dev} - Local device.
\item \texttt{address} - Pointer to the internet host address as \texttt{struct pico$\_$ip4}.
\end{itemize}

\subsubsection*{Return value}
On success, this call returns 0.
On error, -1 is returned and \texttt{pico$\_$err} is set appropriately.

\subsubsection*{Errors}
\begin{itemize}[noitemsep]
\item \texttt{PICO$\_$ERR$\_$EINVAL} - invalid argument
\item \texttt{PICO$\_$ERR$\_$ENXIO} - no such device or address
\end{itemize}

\subsubsection*{Example}
\begin{verbatim}
ret = pico_ipv4_link_del(dev, address);
\end{verbatim}



\subsection{pico$\_$ipv4$\_$link$\_$find}

\subsubsection*{Description}
Find the local device associated to the local IP address 'address'.

\subsubsection*{Function prototype}
\begin{verbatim}
struct pico_device *pico_ipv4_link_find(struct pico_ip4 *address);
\end{verbatim}

\subsubsection*{Parameters}
\begin{itemize}[noitemsep]
\item \texttt{address} - Pointer to the internet host address as \texttt{struct pico$\_$ip4}.
\end{itemize}

\subsubsection*{Return value}
On success, this call returns the local device.
On error, \texttt{NULL} is returned and \texttt{pico$\_$err} is set appropriately.

\subsubsection*{Errors}
\begin{itemize}[noitemsep]
\item \texttt{PICO$\_$ERR$\_$EINVAL} - invalid argument
\item \texttt{PICO$\_$ERR$\_$ENXIO} - no such device or address
\end{itemize}

\subsubsection*{Example}
\begin{verbatim}
dev = pico_ipv4_link_find(address);
\end{verbatim}



\subsection{pico$\_$ipv4$\_$nat$\_$enable}

\subsubsection*{Description}
This function enables NAT functionality on the passed IPv4 link.
Forwarded packets from an internal network will have the public IP address from the passed link
and a translated port number for transmission on the external network.
Usual operation requires at least one additional link for the internal network,
which is used as a gateway for the internal hosts.

\subsubsection*{Function prototype}
\begin{verbatim}
int pico_ipv4_nat_enable(struct pico_ipv4_link *link)
\end{verbatim}

\subsubsection*{Parameters}
\begin{itemize}[noitemsep]
\item \texttt{link} - Pointer to a link \texttt{pico$\_$ipv4$\_$link}.
\end{itemize}

\subsubsection*{Return value}
On success, this call returns 0.
On error, -1 is returned and \texttt{pico$\_$err} is set appropriately.

\subsubsection*{Errors}
\begin{itemize}[noitemsep]
\item \texttt{PICO$\_$ERR$\_$EINVAL} - invalid argument
\end{itemize}

\subsubsection*{Example}
\begin{verbatim}
ret = pico_ipv4_nat_enable(&external_link);
\end{verbatim}



\subsection{pico$\_$ipv4$\_$nat$\_$disable}

\subsubsection*{Description}
Disables the NAT functionality.

\subsubsection*{Function prototype}
\begin{verbatim}
int pico_ipv4_nat_disable(void);
\end{verbatim}

%\subsubsection*{Parameters}

\subsubsection*{Return value}
Always returns 0.

%\subsubsection*{Errors}
%\subsubsection*{Example}


\subsection{pico$\_$ipv4$\_$port$\_$forward}

\subsubsection*{Description}
This function adds or deletes a rule in the IP forwarding table. Internally in the stack,
a one-direction NAT entry will be made.

\subsubsection*{Function prototype}
\begin{verbatim}
int pico_ipv4_port_forward(struct pico_ip4 pub_addr, uint16_t pub_port,
struct pico_ip4 priv_addr, uint16_t priv_port, uint8_t proto,
uint8_t flag)
\end{verbatim}

\subsubsection*{Parameters}
\begin{itemize}[noitemsep]
\item \texttt{pub$\_$addr} - Public IP address, must be identical to the address of the external link.
\item \texttt{pub$\_$port} - Public port to be translated.
\item \texttt{priv$\_$addr} - Private IP address of the host on the internal network.
\item \texttt{priv$\_$port} - Private port of the host on the internal network.
\item \texttt{proto} - Protocol identifier, see supported list below.
\item \texttt{flag} - Option for function call: create \texttt{PICO$\_$IPV4$\_$FORWARD$\_$ADD} (= 1) \\
or delete \texttt{PICO$\_$IPV4$\_$FORWARD$\_$DEL} (= 0).
\end{itemize}

\subsubsection*{Protocol list}
\begin{itemize}[noitemsep]
\item \texttt{PICO$\_$PROTO$\_$ICMP4}
\item \texttt{PICO$\_$PROTO$\_$TCP}
\item \texttt{PICO$\_$PROTO$\_$UDP}
\end{itemize}

\subsubsection*{Return value}
On success, this call 0 after a successful entry of the forward rule.
On error, -1 is returned and \texttt{pico$\_$err} is set appropriately.

\subsubsection*{Errors}
\begin{itemize}[noitemsep]
\item \texttt{PICO$\_$ERR$\_$EINVAL} - invalid argument
\item \texttt{PICO$\_$ERR$\_$ENOMEM} - not enough space
\item \texttt{PICO$\_$ERR$\_$EAGAIN} - not successful, try again
\end{itemize}

\subsubsection*{Example}
\begin{verbatim}
ret = pico_ipv4_port_forward(ext_link_addr, ext_port, host_addr,
host_port, PICO_PROTO_UDP, 1);
\end{verbatim}



\subsection{pico$\_$ipv4$\_$route$\_$add}

\subsubsection*{Description}
Add a new route to the destination IP address from the local device link, f.e. eth0.

\subsubsection*{Function prototype}
\begin{verbatim}
int pico_ipv4_route_add(struct pico_ip4 address, struct pico_ip4 netmask,
struct pico_ip4 gateway, int metric, struct pico_ipv4_link *link);
\end{verbatim}

\subsubsection*{Parameters}
\begin{itemize}[noitemsep]
\item \texttt{address} - Pointer to the destination internet host address as \texttt{struct pico$\_$ip4}.
\item \texttt{netmask} - Netmask of the address. If zeroed, the call assumes the meaning of adding a default gateway.
\item \texttt{gateway} - Gateway of the address network. If zeroed, no gateway will be associated to this route, and the traffic towards the destination will be simply forwarded towards the given device.
\item \texttt{metric} - Metric for this route.
\item \texttt{link} - Local device interface. If a valid gateway is specified, this parameter is not mandatory, otherwise \texttt{NULL} can not be used.
\end{itemize}

\subsubsection*{Return value}
On success, this call returns 0. On error, -1 is returned and \texttt{pico$\_$err} is set appropriately. 
%if the route already exists or no memory could be allocated. 

\subsubsection*{Errors}
\begin{itemize}[noitemsep]
\item \texttt{PICO$\_$ERR$\_$EINVAL} - invalid argument
\item \texttt{PICO$\_$ERR$\_$ENOMEM} - not enough space
\item \texttt{PICO$\_$ERR$\_$EHOSTUNREACH} - host is unreachable
\item \texttt{PICO$\_$ERR$\_$ENETUNREACH} - network unreachable
\end{itemize}

\subsubsection*{Example}
\begin{verbatim}
ret = pico_ipv4_route_add(dst, netmask, gateway, metric, link);
\end{verbatim}



\subsection{pico$\_$ipv4$\_$route$\_$del}

\subsubsection*{Description}
Remove the route to the destination IP address from the local device link, f.e. etho0.

\subsubsection*{Function prototype}
\begin{verbatim}
int pico_ipv4_route_del(struct pico_ip4 address, struct pico_ip4 netmask, int metric);
\end{verbatim}

\subsubsection*{Parameters}
\begin{itemize}[noitemsep]
\item \texttt{address} - Pointer to the destination internet host address as struct \texttt{pico$\_$ip4}.
\item \texttt{netmask} - Netmask of the address.
\item \texttt{metric} - Metric of the route.
\end{itemize}

\subsubsection*{Return value}
On success, this call returns 0 if the route is found.
On error, -1 is returned and \texttt{pico$\_$err} is set appropriately.

\subsubsection*{Errors}
\begin{itemize}[noitemsep]
\item \texttt{PICO$\_$ERR$\_$EINVAL} - invalid argument
\end{itemize}

\subsubsection*{Example}
\begin{verbatim}
ret = pico_ipv4_route_del(dst, netmask, metric);
\end{verbatim}



\subsection{pico$\_$ipv4$\_$route$\_$get$\_$gateway}

\subsubsection*{Description}
This function gets the gateway address for the given destination IP address, if set.

\subsubsection*{Function prototype}
\begin{verbatim}
struct pico_ip4 pico_ipv4_route_get_gateway(struct pico_ip4 *addr)
\end{verbatim}

\subsubsection*{Parameters}
\begin{itemize}[noitemsep]
\item \texttt{address} - Pointer to the destination internet host address as struct \texttt{pico$\_$ip4}.
\end{itemize}

\subsubsection*{Return value}
On success the gateway address is returned.
On error a \texttt{null} address is returned (\texttt{0.0.0.0}) and \texttt{pico$\_$err} is set appropriately.

\subsubsection*{Errors}
\begin{itemize}[noitemsep]
\item \texttt{PICO$\_$ERR$\_$EINVAL} - invalid argument
\item \texttt{PICO$\_$ERR$\_$EHOSTUNREACH} - host is unreachable
\end{itemize}

\subsubsection*{Example}
\begin{verbatim}
gateway_addr = pico_ip4 pico_ipv4_route_get_gateway(&dest_addr)
\end{verbatim}


\subsection{pico$\_$icmp4$\_$ping}

\subsubsection*{Description}
This function sends out a number of ping echo requests and checks if the replies are received correctly.
The information from the replies is passed to the callback function after a successful reception.
If a timeout expires before a reply is received, the callback is called with the error condition.

\subsubsection*{Function prototype}
\begin{verbatim}
int pico_icmp4_ping(char *dst, int count, int interval, int timeout, int size,
void (*cb)(struct pico_icmp4_stats *));
\end{verbatim}

\subsubsection*{Parameters}
\begin{itemize}[noitemsep]
\item \texttt{dst} - Pointer to the destination internet host address as text string
\item \texttt{count} - Number of pings going to be send
\item \texttt{interval} - Time between two transmissions (in ms)
\item \texttt{timeout} - Timeout period untill reply received (in ms)
\item \texttt{size} - Size of data buffer in bytes
\item \texttt{cb} - Callback for ICMP ping
\end{itemize}

\subsubsection*{Data structure \texttt{struct pico$\_$icmp4$\_$stats}}
\begin{verbatim}
struct pico_icmp4_stats
{
  struct pico_ip4 dst;
  unsigned long size;
  unsigned long seq;
  unsigned long time;
  unsigned long ttl;
  int err;
};
\end{verbatim}
With \textbf{err} values:
\begin{itemize}[noitemsep]
\item \texttt{PICO$\_$PING$\_$ERR$\_$REPLIED} (value 0)
\item \texttt{PICO$\_$PING$\_$ERR$\_$TIMEOUT} (value 1)
\item \texttt{PICO$\_$PING$\_$ERR$\_$UNREACH} (value 2)
\item \texttt{PICO$\_$PING$\_$ERR$\_$PENDING} (value 0xFFFF)
\end{itemize}

\subsubsection*{Return value}
On success, this call returns a positive number, which is the ID of the ping operation just started.
On error, -1 is returned and \texttt{pico$\_$err} is set appropriately.

\subsubsection*{Errors}
\begin{itemize}[noitemsep]
\item \texttt{PICO$\_$ERR$\_$EINVAL} - invalid argument
\item \texttt{PICO$\_$ERR$\_$ENOMEM} - not enough space
\end{itemize}

\subsubsection*{Example}
\begin{verbatim}
id = pico_icmp4_ping(dst_addr, 30, 10, 100, 1000, callback);
\end{verbatim}


\subsection{pico$\_$icmp4$\_$ping$\_$abort}

\subsubsection*{Description}
This function aborts an ongoing ping operation that has previously started using pico$\_$icmp4$\_$ping().

\subsubsection*{Function prototype}
\begin{verbatim}
int pico_icmp4_ping_abort(int id);
\end{verbatim}

\subsubsection*{Parameters}
\begin{itemize}[noitemsep]
    \item \texttt{id} - identification number for the ping operation. This has been returned by \texttt{pico$\_$icmp4$\_$ping()} and it is intended to distinguish the operation to be cancelled.
\end{itemize}

\subsubsection*{Return value}
On success, this call returns 0. 
On error, -1 is returned and \texttt{pico$\_$err} is set appropriately.

\subsubsection*{Errors}
\begin{itemize}[noitemsep]
\item \texttt{PICO$\_$ERR$\_$EINVAL} - invalid argument
\end{itemize}

\subsubsection*{Example}
\begin{verbatim}
ret = pico_icmp4_ping_abort(id);
\end{verbatim}


\section{Socket calls}

% Short description/overview of module functions
With the socket calls, the user can open, close, bind, \ldots sockets and do read
or write operations. The provided transport protocols are UDP and TCP.

\subsection{pico$\_$socket$\_$open}

\subsubsection*{Description}
This function will be called to open a socket from the application level. The created
socket will be unbound.

\subsubsection*{Function prototype}
\begin{verbatim}
struct pico_socket *pico_socket_open(uint16_t net, uint16_t proto,
void (*wakeup)(uint16_t ev, struct pico_socket *s));
\end{verbatim}

\subsubsection*{Parameters}
\begin{itemize}[noitemsep]
\item \texttt{net} - Network protocol, \texttt{PICO$\_$PROTO$\_$IPV4} = 0, \texttt{PICO$\_$PROTO$\_$IPV6} = 41
\item \texttt{proto} - Transport protocol, \texttt{PICO$\_$PROTO$\_$TCP} = 6, \texttt{PICO$\_$PROTO$\_$UDP} = 17
\item \texttt{wakeup} - Callback function that accepts 2 parameters:
\begin{itemize}[noitemsep]
\item \texttt{ev} - Events that apply to that specific socket, see further
\item \texttt{s} - Pointer to a socket of type struct \texttt{pico$\_$socket}
\end{itemize}
\end{itemize}

\subsubsection*{Possible events for sockets}
\begin{itemize}[noitemsep]
\item \texttt{PICO$\_$SOCK$\_$EV$\_$RD} - triggered when new data arrives on the socket. A new receive action can be taken by the socket owner because this event indicates there is new data to receive.
\item \texttt{PICO$\_$SOCK$\_$EV$\_$WR} - triggered when ready to write to the socket. Issuing a write/send call will now succeed if the buffer has enough space to allocate new outstanding data.
\item \texttt{PICO$\_$SOCK$\_$EV$\_$CONN} - triggered when connection is established (TCP only). This event is received either after a successful call to \texttt{pico$\_$socket$\_$connect} to indicate that the connection has been established, or on a listening socket, indicating that a call to \texttt{pico$\_$socket$\_$accept} may now be issued in order to accept the incoming connection from a remote host.
\item \texttt{PICO$\_$SOCK$\_$EV$\_$CLOSE} - triggered when a FIN segment is received (TCP only). This event indicates that the other endpont has closed the connection, so the local TCP layer is only allowed to send new data until a local shutdown or close is initiated. PicoTCP is able to keep the connection half-open (only for sending) after the FIN packet has been received, allowing new data to be sent in the TCP CLOSE$\_$WAIT state.
\item \texttt{PICO$\_$SOCK$\_$EV$\_$FIN} - triggered when the socket is closed. No further communication is possible from this point on the socket.
\item \texttt{PICO$\_$SOCK$\_$EV$\_$ERR} - triggered when an error occurs. 
\end{itemize}

\subsubsection*{Return value}
On success, this call returns a pointer to the declared socket (\texttt{struct pico$\_$socket *}).
On error the socket is not created, \texttt{NULL} is returned, and \texttt{pico$\_$err} is set appropriately.

\subsubsection*{Errors}
\begin{itemize}[noitemsep]
\item \texttt{PICO$\_$ERR$\_$EINVAL} - invalid argument
\item \texttt{PICO$\_$ERR$\_$EPROTONOSUPPORT} - protocol not supported
\item \texttt{PICO$\_$ERR$\_$ENETUNREACH} - network unreachable 
\end{itemize}

\subsubsection*{Example}
\begin{verbatim}
sk_tcp = pico_socket_open(PICO_PROTO_IPV4, PICO_PROTO_TCP, &wakeup);
\end{verbatim}


\subsection{pico$\_$socket$\_$read}

\subsubsection*{Description}
This function will be called to read a string from a socket from the application level. The function checks whether or not the socket is bound.

\subsubsection*{Function prototype}
\begin{verbatim}
int pico_socket_read(struct pico_socket *s, void *buf, int len);
\end{verbatim}

\subsubsection*{Parameters}
\begin{itemize}[noitemsep]
\item \texttt{s} - Pointer to socket of type \texttt{struct pico$\_$socket}
\item \texttt{buf} - Void pointer to the start of a string buffer where the string will be stored
\item \texttt{len} - Length of the string that was read from the socket (in bytes)
\end{itemize}

\subsubsection*{Return value}
On success, this call returns an integer representing the number of bytes read.
On error, -1 is returned, and \texttt{pico$\_$err} is set appropriately.

\subsubsection*{Errors}
\begin{itemize}[noitemsep]
\item \texttt{PICO$\_$ERR$\_$EINVAL} - invalid argument
\item \texttt{PICO$\_$ERR$\_$EIO} - input/output error
\item \texttt{PICO$\_$ERR$\_$ESHUTDOWN} - cannot read after transport endpoint shutdown
\end{itemize}

\subsubsection*{Example}
\begin{verbatim}
bytesRead = pico_socket_read(sk_tcp, buffer, bufferLength);
\end{verbatim}



\subsection{pico$\_$socket$\_$write}

\subsubsection*{Description}
This function will be called to write a string to a socket from the application level.
This function also checks if the socket is bound, connected and that it isn't shutdown
locally. This is the preferred function to use when writing strings from application
level. 

\subsubsection*{Function prototype}
\begin{verbatim}
int pico_socket_write(struct pico_socket *s, void *buf, int len);
\end{verbatim}

\subsubsection*{Parameters}
\begin{itemize}[noitemsep]
\item \texttt{s} - Pointer to socket of type \texttt{struct pico$\_$socket}
\item \texttt{buf} - Void pointer to the start of a string buffer where the string is stored
\item \texttt{len} - Length of the string that is stored in the buffer (in bytes)
\end{itemize}

\subsubsection*{Return value}
On success, this call returns an integer representing the number of bytes written to the socket.
On error, -1 is returned, and \texttt{pico$\_$err} is set appropriately.

\subsubsection*{Errors}
\begin{itemize}[noitemsep]
\item \texttt{PICO$\_$ERR$\_$EINVAL} - invalid argument
\item \texttt{PICO$\_$ERR$\_$EIO} - input/output error
\item \texttt{PICO$\_$ERR$\_$ENOTCONN} - the socket is not connected
\item \texttt{PICO$\_$ERR$\_$ESHUTDOWN} - cannot send after transport endpoint shutdown
\item \texttt{PICO$\_$ERR$\_$EADDRNOTAVAIL} - address not available
\item \texttt{PICO$\_$ERR$\_$EHOSTUNREACH} - host is unreachable
\item \texttt{PICO$\_$ERR$\_$ENOMEM} - not enough space
\item \texttt{PICO$\_$ERR$\_$EAGAIN} - resource temporarily unavailable
\end{itemize}

\subsubsection*{Example}
\begin{verbatim}
bytesWritten = pico_socket_write(sk_tcp, buffer, bufLength);
\end{verbatim}


\subsection{pico$\_$socket$\_$sendto}

\subsubsection*{Description}
This function is be called by the \texttt{pico$\_$socket$\_$write} and \texttt{pico$\_$socket$\_$send} functions.
This function sends a string from the local address to the remote address, without checking
if the remote is connected or not.

\subsubsection*{Function prototype}
\begin{verbatim}
int pico_socket_sendto(struct pico_socket *s, const void *buf, int len,
void *dst, uint16_t remote_port);
\end{verbatim}

\subsubsection*{Parameters}
\begin{itemize}[noitemsep]
\item \texttt{s} - Pointer to socket of type \texttt{struct pico$\_$socket}
\item \texttt{buf} - Void pointer to the start of a string buffer where the string is stored
\item \texttt{len} - Length of the string that is stored in the buffer (in bytes)
\item \texttt{dst} - Pointer to the origin of the IPv4/IPv6 frame header
\item \texttt{remote$\_$port} - Portnumber of the receiving socket
\end{itemize}

\subsubsection*{Return value}
On success, this call returns an integer representing the number of bytes written to the socket.
On error, -1 is returned, and \texttt{pico$\_$err} is set appropriately.

\subsubsection*{Errors}
\begin{itemize}[noitemsep]
\item \texttt{PICO$\_$ERR$\_$EADDRNOTAVAIL} - address not available
\item \texttt{PICO$\_$ERR$\_$EINVAL} - invalid argument
\item \texttt{PICO$\_$ERR$\_$EHOSTUNREACH} - host is unreachable
\item \texttt{PICO$\_$ERR$\_$ENOMEM} - not enough space
\item \texttt{PICO$\_$ERR$\_$EAGAIN} - resource temporarily unavailable
\end{itemize}

\subsubsection*{Example}
\begin{verbatim}
bytesWritten = pico_socket_sendto(sk_tcp, buf, len, &sk_tcp->remote_addr,
sk_tcp->remote_port);
\end{verbatim}


\subsection{pico$\_$socket$\_$recvfrom}

\subsubsection*{Description}
This function is called to receive a string of data from the specified socket.
It is useful when called in the context of a non-connected socket, to receive
the information regarding the origin of the data, namely the origin address and 
the remote port number.

\subsubsection*{Function prototype}
\begin{verbatim}
int pico_socket_recvfrom(struct pico_socket *s, void *buf, int len,
void *orig, uint16_t *remote_port);
\end{verbatim}

\subsubsection*{Parameters}
\begin{itemize}[noitemsep]
\item \texttt{s} - Pointer to socket of type \texttt{struct pico$\_$socket}
\item \texttt{buf} - Void pointer to the start of a string buffer where the string will be stored
\item \texttt{len} - Length of the string that will be stored in the buffer (in bytes)
\item \texttt{orig} - Pointer to the origin of the IPv4/IPv6 frame header, can be NULL
\item \texttt{remote$\_$port} - Pointer to the port number of the sender socket, can be NULL
\end{itemize}

\subsubsection*{Return value}
On success, this call returns an integer representing the number of bytes read from the socket. On success, if \texttt{orig}
is not NULL, The address of the remote endpoint is stored in the memory area pointed by \texttt{orig}. 
In the same way, \texttt{remote$\_$port} will contain the portnumber of the sending socket, unless a NULL is passed
from the caller.

On error, -1 is returned, and \texttt{pico$\_$err} is set appropriately.

\subsubsection*{Errors}
\begin{itemize}[noitemsep]
\item \texttt{PICO$\_$ERR$\_$EINVAL} - invalid argument
\item \texttt{PICO$\_$ERR$\_$ESHUTDOWN} - cannot read after transport endpoint shutdown
\item \texttt{PICO$\_$ERR$\_$EADDRNOTAVAIL} - address not available
\end{itemize}

\subsubsection*{Example}
\begin{verbatim}
bytesRcvd = pico_socket_recvfrom(sk_tcp, buf, bufLen, &peer, &port);
\end{verbatim}


\subsection{pico$\_$socket$\_$send}

\subsubsection*{Description}
This function is called to send a string of data to the specified socket.
This function also checks if the socket is connected and then calls the
\texttt{pico$\_$socket$\_$sendto} function.

\subsubsection*{Function prototype}
\begin{verbatim}
int pico_socket_send(struct pico_socket *s, const void *buf, int len);
\end{verbatim}


\subsubsection*{Parameters}
\begin{itemize}[noitemsep]
\item \texttt{s} - Pointer to socket of type \texttt{struct pico$\_$socket}
\item \texttt{buf} - Void pointer to the start of a string buffer where the string is stored
\item \texttt{len} - Length of the string that is stored in the buffer (in bytes)
\end{itemize}

\subsubsection*{Return value}
On success, this call returns an integer representing the number of bytes written to
the socket. On error, -1 is returned, and \texttt{pico$\_$err} is set appropriately.

\subsubsection*{Errors}
\begin{itemize}[noitemsep]
\item \texttt{PICO$\_$ERR$\_$EINVAL} - invalid argument
\item \texttt{PICO$\_$ERR$\_$ENOTCONN} - the socket is not connected
\item \texttt{PICO$\_$ERR$\_$EADDRNOTAVAIL} - address not available
\item \texttt{PICO$\_$ERR$\_$EHOSTUNREACH} - host is unreachable
\item \texttt{PICO$\_$ERR$\_$ENOMEM} - not enough space
\item \texttt{PICO$\_$ERR$\_$EAGAIN} - resource temporarily unavailable
\end{itemize}

\subsubsection*{Example}
\begin{verbatim}
bytesRcvd = pico_socket_send(sk_tcp, buf, bufLen);
\end{verbatim}


\subsection{pico$\_$socket$\_$recv}

\subsubsection*{Description}
This function directly calls the \texttt{pico$\_$socket$\_$recvfrom} function.

\subsubsection*{Function prototype}
\begin{verbatim}
int pico_socket_recv(struct pico_socket *s, void *buf, int len);
\end{verbatim}

\subsubsection*{Parameters}
\begin{itemize}[noitemsep]
\item \texttt{s} - Pointer to socket of type \texttt{struct pico$\_$socket}
\item \texttt{buf} - Void pointer to the start of a string buffer where the string will be stored
\item \texttt{len} - Length of the string in the socket buffer (in bytes)
\end{itemize}

\subsubsection*{Return value}
On success, this call returns an integer representing the number of bytes read
from the socket. On error, -1 is returned, and \texttt{pico$\_$err} is set appropriately.

\subsubsection*{Errors}
\begin{itemize}[noitemsep]
\item \texttt{PICO$\_$ERR$\_$EINVAL} - invalid argument
\item \texttt{PICO$\_$ERR$\_$ESHUTDOWN} - cannot read after transport endpoint shutdown
\item \texttt{PICO$\_$ERR$\_$EADDRNOTAVAIL} - address not available
\end{itemize}

\subsubsection*{Example}
\begin{verbatim}
bytesRcvd = pico_socket_recv(sk_tcp, buf, bufLen);
\end{verbatim}


\subsection{pico$\_$socket$\_$bind}

\subsubsection*{Description}
This function binds a local IP-address and port to the specified socket.

\subsubsection*{Function prototype}
\begin{verbatim}
int pico_socket_bind(struct pico_socket *s, void *local_addr, uint16_t *port);
\end{verbatim}


\subsubsection*{Parameters}
\begin{itemize}[noitemsep]
\item \texttt{s} - Pointer to socket of type \texttt{struct pico$\_$socket}
\item \texttt{local$\_$addr} - Void pointer to the local IP-address
\item \texttt{port} - Local portnumber to bind with the socket
\end{itemize}

\subsubsection*{Return value}
On success, this call returns 0 after a succesfull bind.
On error, -1 is returned, and \texttt{pico$\_$err} is set appropriately.

\subsubsection*{Errors}
\begin{itemize}[noitemsep]
\item \texttt{PICO$\_$ERR$\_$EINVAL} - invalid argument
\item \texttt{PICO$\_$ERR$\_$ENOMEM} - not enough space
\item \texttt{PICO$\_$ERR$\_$ENXIO} - no such device or address
\end{itemize}

\subsubsection*{Example}
\begin{verbatim}
errMsg = pico_socket_bind(sk_tcp, &sockaddr4->addr, &sockaddr4->port);
\end{verbatim}


\subsection{pico$\_$socket$\_$connect}

\subsubsection*{Description}
This function connects a local socket to a remote socket of a server that is listening, or permanently associate a remote UDP peer as default receiver for any further outgoing traffic through this socket.

\subsubsection*{Function prototype}
\begin{verbatim}
int pico_socket_connect(struct pico_socket *s, void *srv_addr,
uint16_t remote_port);
\end{verbatim}


\subsubsection*{Parameters}
\begin{itemize}[noitemsep]
\item \texttt{s} - Pointer to socket of type \texttt{struct pico$\_$socket}
\item \texttt{srv$\_$addr} - Void pointer to the remote IP-address to connect to
\item \texttt{remote$\_$port} - Remote port number on which the socket will be connected to
\end{itemize} 

\subsubsection*{Return value}
On success, this call returns 0 after a succesfull connect.
On error, -1 is returned, and \texttt{pico$\_$err} is set appropriately.

\subsubsection*{Errors}
\begin{itemize}[noitemsep]
\item \texttt{PICO$\_$ERR$\_$EPROTONOSUPPORT} - protocol not supported
\item \texttt{PICO$\_$ERR$\_$EINVAL} - invalid argument
\item \texttt{PICO$\_$ERR$\_$EHOSTUNREACH} - host is unreachable 
\end{itemize}

\subsubsection*{Example}
\begin{verbatim}
errMsg = pico_socket_connect(sk_tcp, &sockaddr4->addr, sockaddr4->port);
\end{verbatim}


\subsection{pico$\_$socket$\_$listen}

\subsubsection*{Description}
A server can use this function when a socket is opened and bound to start listening to it.

\subsubsection*{Function prototype}
\begin{verbatim}
int pico_socket_listen(struct pico_socket *s, int backlog);
\end{verbatim}


\subsubsection*{Parameters}
\begin{itemize}[noitemsep]
\item \texttt{s} - Pointer to socket of type \texttt{struct pico$\_$socket}
\item \texttt{backlog} - Maximum connection requests
\end{itemize}

\subsubsection*{Return value}
On success, this call returns 0 after a succesfull listen start.
On error, -1 is returned, and \texttt{pico$\_$err} is set appropriately. 

\subsubsection*{Errors}
\begin{itemize}[noitemsep]
\item \texttt{PICO$\_$ERR$\_$EINVAL} - invalid argument
\item \texttt{PICO$\_$ERR$\_$EISCONN} - socket is connected
\end{itemize}

\subsubsection*{Example}
\begin{verbatim}
errMsg = pico_socket_listen(sk_tcp, 3);
\end{verbatim}


\subsection{pico$\_$socket$\_$accept}

\subsubsection*{Description}
When a server is listening on a socket and the client is trying to connect.
The server on his side will wakeup and acknowledge the connection by calling the this function.

\subsubsection*{Function prototype}
\begin{verbatim}
struct pico_socket *pico_socket_accept(struct pico_socket *s, void *orig,
uint16_t *local_port);
\end{verbatim}

\subsubsection*{Parameters}
\begin{itemize}[noitemsep]
\item \texttt{s} - Pointer to socket of type \texttt{struct pico$\_$socket}
\item \texttt{orig} - Pointer to the origin of the IPv4/IPv6 frame header
\item \texttt{local$\_$port} - Portnumber of the local socket (pointer)
\end{itemize}

\subsubsection*{Return value}
On success, this call returns the pointer to a \texttt{struct pico$\_$socket} that
represents the client thas was just connected. Also \texttt{orig} will contain the requesting
IP-address and \texttt{remote$\_$port} will contain the portnumber of the requesting socket.
On error, \texttt{NULL} is returned, and \texttt{pico$\_$err} is set appropriately.

\subsubsection*{Errors}
\begin{itemize}[noitemsep]
\item \texttt{PICO$\_$ERR$\_$EINVAL} - invalid argument
\item \texttt{PICO$\_$ERR$\_$EAGAIN} - resource temporarily unavailable
\end{itemize}

\subsubsection*{Example}
\begin{verbatim}
client = pico_socket_accept(sk_tcp, &peer, &port);
\end{verbatim}


\subsection{pico$\_$socket$\_$shutdown}

\subsubsection*{Description}
Used by the \texttt{pico$\_$socket$\_$close} function to shutdown read and write mode for
the specified socket. With this function one can close a socket for reading
and/or writing.

\subsubsection*{Function prototype}
\begin{verbatim}
int pico_socket_shutdown(struct pico_socket *s, int mode);
\end{verbatim}

\subsubsection*{Parameters}
\begin{itemize}[noitemsep]
\item \texttt{s} - Pointer to socket of type \texttt{struct pico$\_$socket}
\item \texttt{mode} - \texttt{PICO$\_$SHUT$\_$RDWR}, \texttt{PICO$\_$SHUT$\_$WR}, \texttt{PICO$\_$SHUT$\_$RD}
\end{itemize}

\subsubsection*{Return value}
On success, this call returns 0 after a succesfull socket shutdown.
On error, -1 is returned, and \texttt{pico$\_$err} is set appropriately.

\subsubsection*{Errors}
\begin{itemize}[noitemsep]
\item \texttt{PICO$\_$ERR$\_$EINVAL} - invalid argument
\end{itemize}

\subsubsection*{Example}
\begin{verbatim}
errMsg = pico_socket_shutdown(s, PICO_SHUT_RDWR);
\end{verbatim}


\subsection{pico$\_$socket$\_$close}

\subsubsection*{Description}
Function used on application level to close a socket. Always closes read and write connection.

\subsubsection*{Function prototype}
\begin{verbatim}
int pico_socket_close(struct pico_socket *s);
\end{verbatim}

\subsubsection*{Parameters}
\begin{itemize}[noitemsep]
\item \texttt{s} - Pointer to socket of type \texttt{struct pico$\_$socket}
\end{itemize}

\subsubsection*{Return value}
On success, this call returns 0 after a succesfull socket shutdown.
On error, -1 is returned, and \texttt{pico$\_$err} is set appropriately.

\subsubsection*{Errors}
\begin{itemize}[noitemsep]
\item \texttt{PICO$\_$ERR$\_$EINVAL} - invalid argument
\end{itemize}

\subsubsection*{Example}
\begin{verbatim}
errMsg = pico_socket_close(sk_tcp);
\end{verbatim}



\subsection{pico$\_$socket$\_$setoption}

\subsubsection*{Description}
Function used to set socket options.

\subsubsection*{Function prototype}
\begin{verbatim}
int pico_socket_setoption(struct pico_socket *s, int option, void *value);
\end{verbatim}

\subsubsection*{Parameters}
\begin{itemize}[noitemsep]
\item \texttt{s} - Pointer to socket of type \texttt{struct pico$\_$socket}
\item \texttt{option} - Option to be set (see further for all options)
\item \texttt{value} - Value of option (void pointer)
\end{itemize}

\subsubsection*{Available socket options}
\begin{itemize}[noitemsep]
\item \texttt{PICO$\_$TCP$\_$NODELAY} - Disables/enables the Nagle algorithm
\item \texttt{PICO$\_$IP$\_$MULTICAST$\_$IF} - (Not supported) Set link multicast datagrams are sent from, default is first added link
\item \texttt{PICO$\_$IP$\_$MULTICAST$\_$TTL} - Set TTL (0-255) of multicast datagrams, default is 1
\item \texttt{PICO$\_$IP$\_$MULTICAST$\_$LOOP} - Specifies if a copy of an outgoing multicast datagram is looped back as long as it is a member of the multicast group, default is enabled
\item \texttt{PICO$\_$IP$\_$ADD$\_$MEMBERSHIP} - Join the multicast group specified
\item \texttt{PICO$\_$IP$\_$DROP$\_$MEMBERSHIP} - Leave the multicast group specified
\end{itemize}

\subsubsection*{Return value}
On success, this call returns 0 after a succesfull setting of socket option.
On error, -1 is returned, and \texttt{pico$\_$err} is set appropriately.

\subsubsection*{Errors}
\begin{itemize}[noitemsep]
\item \texttt{PICO$\_$ERR$\_$EINVAL} - invalid argument
\end{itemize}

\subsubsection*{Example}
\begin{verbatim}
ret = pico_socket_setoption(sk_tcp, PICO_TCP_NODELAY, NULL);

uint8_t ttl = 2;
ret = pico_socket_setoption(sk_udp, PICO_IP_MULTICAST_TTL, &ttl);

uint8_t loop = 0;
ret = pico_socket_setoption(sk_udp, PICO_IP_MULTICAST_LOOP, &loop);

struct pico_ip4 inaddr_dst, inaddr_link;
struct pico_ip_mreq mreq = {{0},{0}};
pico_string_to_ipv4("224.7.7.7", &inaddr_dst.addr);
pico_string_to_ipv4("192.168.0.2", &inaddr_link.addr);
mreq.mcast_group_addr = inaddr_dst;
mreq.mcast_link_addr = inaddr_link;
ret = pico_socket_setoption(sk_udp, PICO_IP_ADD_MEMBERSHIP, &mreq);
ret = pico_socket_setoption(sk_udp, PICO_IP_DROP_MEMBERSHIP, &mreq)
\end{verbatim}


\subsection{pico$\_$socket$\_$getoption}

\subsubsection*{Description}
Function used to get socket options.

\subsubsection*{Function prototype}
\begin{verbatim}
int pico_socket_getoption(struct pico_socket *s, int option, void *value);
\end{verbatim}

\subsubsection*{Parameters}
\begin{itemize}[noitemsep]
\item \texttt{s} - Pointer to socket of type \texttt{struct pico$\_$socket}
\item \texttt{option} - Option to be set (see further for all options)
\item \texttt{value} - Value of option (void pointer)
\end{itemize}

\subsubsection*{Available socket options}
\begin{itemize}[noitemsep]
\item \texttt{PICO$\_$TCP$\_$NODELAY} - Nagle algorithm, \texttt{value} casted to \texttt{(int *)} (0 = disabled, 1 = enabled)
\item \texttt{PICO$\_$IP$\_$MULTICAST$\_$IF} - (Not supported) Link multicast datagrams are sent from
\item \texttt{PICO$\_$IP$\_$MULTICAST$\_$TTL} - TTL (0-255) of multicast datagrams
\item \texttt{PICO$\_$IP$\_$MULTICAST$\_$LOOP} - Loop back a copy of an outgoing multicast datagram, as long as it is a member of the multicast group, or not.
\end{itemize}

\subsubsection*{Return value}
On success, this call returns 0 after a succesfull getting of socket option. The value of
the option is written to \texttt{value}.
On error, -1 is returned, and \texttt{pico$\_$err} is set appropriately.

\subsubsection*{Errors}
\begin{itemize}[noitemsep]
\item \texttt{PICO$\_$ERR$\_$EINVAL} - invalid argument
\end{itemize}

\subsubsection*{Example}
\begin{verbatim}
ret = pico_socket_getoption(sk_tcp, PICO_TCP_NODELAY, &stat);

uint8_t ttl = 0;
ret = pico_socket_getoption(sk_udp, PICO_IP_MULTICAST_TTL, &ttl);

uint8_t loop = 0;
ret = pico_socket_getoption(sk_udp, PICO_IP_MULTICAST_LOOP, &loop);
\end{verbatim}

\section{DHCP client}

% Short description/overview of module functions
A DHCP client for obtaining a dynamic IP address.
When initiating a negotiation the user is passed an identifier,
which must then be passed to all future calls to \texttt{pico\_dhcp} functions.
(Currently DHCP can only be run on one interface. Future versions may support
DHCP on multiple interfaces, and the functions described here are already prepared for that.)


\subsection{pico\_dhcp\_initiate\_negotiation}

\subsubsection*{Description}
Initiate a DHCP negotiation. The user passes a callback-function, which will be called
when DHCP has succeeded or failed.

\subsubsection*{Function prototype}
\begin{verbatim}
void * pico_dhcp_initiate_negotiation(struct pico_device* device,
void (*callback)(void* cli, int code));
\end{verbatim}


\subsubsection*{Parameters}
\begin{itemize}[noitemsep]
\item \texttt{device} - the device on which a negotiation should be started
\item \texttt{callback} - the function which will be called in case of success or failure.
Note that this function can be called multiple times. An example would be if initially DHCP
succeeded, but then the DHCP server was removed from the network long enough for the lease
to expire, and later added again to the network. The callback would be called 3 times in
this example: first with code \texttt{PICO\_DHCP\_SUCCESS}, then with \texttt{PICO\_DHCP\_RESET},
and finally again with \texttt{PICO\_DHCP\_SUCCESS}.
Also note that this callback may already be called before \texttt{pico\_dhcp\_initiate\_negotiation}
has returned, e.g. in case of failure to open a socket.
It accepts two parameters : 
\begin{itemize}[noitemsep]
\item \texttt{cli} - the identifier of the negotiation
\item \texttt{code} - the reason the callback occurred, see further
\end{itemize}
\end{itemize}

\subsubsection*{Possible DHCP codes}
\begin{itemize}[noitemsep]
\item \texttt{PICO\_DHCP\_SUCCESS} - DHCP succeeded, the user can start using the assigned address,
which can be obtained by calling \texttt{pico\_dhcp\_get\_address}.
\item \texttt{PICO\_DHCP\_ERROR} - an error occurred. DHCP is unable to recover from this error.
\texttt{pico$\_$err} is set appropriately.
\item \texttt{PICO\_DHCP\_RESET} - DHCP was unable to renew its lease, and the lease expired.
The user must immediately stop using the previously assigned IP, and wait for DHCP to obtain a
new lease. DHCP will automatically start negotiations again.
\end{itemize}

\subsubsection*{Return value}
A \texttt{void*} identifying the negotiation. This must be passed to all calls related to DHCP.
This is to create the possibility of initiating DHCP negotiations on multiple devices (currently not supported).

\subsubsection*{Errors}   % ORGANIZE
All errors are reported through the callback-function described above.
\begin{itemize}[noitemsep]
\item \texttt{PICO$\_$ERR$\_$EADDRNOTAVAIL} - address not available		% pico_socket_sendto
\item \texttt{PICO$\_$ERR$\_$EINVAL} - invalid argument
\item \texttt{PICO$\_$ERR$\_$EHOSTUNREACH} - host is unreachable
\item \texttt{PICO$\_$ERR$\_$ENOMEM} - not enough space
\item \texttt{PICO$\_$ERR$\_$EAGAIN} - resource temporarily unavailable
\item \texttt{PICO$\_$ERR$\_$EPROTONOSUPPORT} - protocol not supported	% pico_socket_open
\item \texttt{PICO$\_$ERR$\_$ENETUNREACH} - network unreachable 
\item \texttt{PICO$\_$ERR$\_$EINVAL} - invalid argument					% pico_socket_bind
\item \texttt{PICO$\_$ERR$\_$ENXIO} - no such device or address
\item \texttt{PICO$\_$ERR$\_$EOPNOTSUPP} - operation not supported on socket
\end{itemize}

\subsubsection*{Example}
\begin{verbatim}
void* identifier = pico_dhcp_initiate_negotiation(dev, &callback_dhcpclient);
\end{verbatim}


\subsection{pico\_dhcp\_get\_address}

\subsubsection*{Description}
Get the address that was assigned through DHCP. This function should only be called after
a callback occurred with code \texttt{PICO\_DHCP\_SUCCESS}. 

\subsubsection*{Function prototype}
\texttt{struct pico\_ip4 pico\_dhcp\_get\_address(void* cli);}

\subsubsection*{Parameters}
\begin{itemize}[noitemsep]
\item \texttt{cli} - the negotiation identifier that was returned from \texttt{pico\_dhcp\_initiate\_negotiations}.
\end{itemize}

\subsubsection*{Return value}
\texttt{struct pico\_ip4} - the address that was assigned

%\subsubsection*{Errors}

\subsubsection*{Example}
\begin{verbatim}
struct pico_ip4 address = pico_dhcp_get_address(identifier);
\end{verbatim}


\subsection{pico\_dhcp\_get\_gateway}

\subsubsection*{Description}
Get the address of the gateway that was assigned through DHCP. This function should
only be called after a callback occurred with code \texttt{PICO\_DHCP\_SUCCESS}. 

\subsubsection*{Function prototype}
\texttt{struct pico\_ip4 pico\_dhcp\_get\_gateway(void* cli);}

\subsubsection*{Parameters}
\begin{itemize}[noitemsep]
\item \texttt{cli} : the negotiation identifier that was returned from
\texttt{pico\_dhcp\_initiate\_negotiations}.
\end{itemize}

\subsubsection*{Return value}
\begin{itemize}[noitemsep]
\item \texttt{struct pico\_ip4} - the address of the gateway that should be used. 
\end{itemize}

%\subsubsection*{Errors}

\subsubsection*{Example}
\begin{verbatim}
struct pico_ip4 gateway = pico_dhcp_get_gateway(identifier);
\end{verbatim}

\section{DHCP server}

% Short description/overview of module functions


\subsection{pico\_dhcp\_server\_initiate}

\subsubsection*{Description}
This function starts a simple DHCP server. 

\subsubsection*{Function prototype}
\texttt{int pico\_dhcp\_server\_initiate(struct pico\_dhcpd\_settings *settings);}

\subsubsection*{Parameters}
\begin{itemize}[noitemsep]
\item \texttt{settings} - a pointer to a struct \texttt{pico\_dhcpd\_settings}, in which the following members matter to the user : 
\begin{itemize}[noitemsep]
\item \texttt{struct pico\_ip4 my\_ip} - the IP address of the device performing DHCP. Only IPs of this network will be served.
\item \texttt{uint32\_t pool\_start} - the lowest host number that may be assigned, defaults to 100 if not provided.
\item \texttt{uint32\_t pool\_end} - the highest host number that may be assigned, defaults to 254 if not provided.
\item \texttt{uint32\_t lease\_time} - the advertised lease time in seconds, defaults to 120 if not provided.
\end{itemize}
\end{itemize}

\subsubsection*{Return value}
On successful startup of the dhcp server, 0 is returned.
On error, -1 is returned, and \texttt{pico$\_$err} is set appropriately.

\subsubsection*{Errors}
\begin{itemize}[noitemsep]
%everything from :
%pico_socket_open
\item PICO$\_$ERR$\_$EPROTONOSUPPORT - protocol not supported
\item PICO$\_$ERR$\_$ENETUNREACH - network unreachable 
%pico_socket_bind
\item PICO$\_$ERR$\_$EINVAL - invalid argument
\item PICO$\_$ERR$\_$ENXIO - no such device or address
\end{itemize}

\subsection{pico\_dhcp\_server\_destroy}

\subsubsection*{Description}
This function stops a previously started DHCP server on the given device. 

\subsubsection*{Function prototype}
\texttt{int pico\_dhcp\_server\_destroy(struct pico\_device *dev);}

\subsubsection*{Parameters}
\begin{itemize}[noitemsep]
\item \texttt{dev} - a pointer to a struct \texttt{pico\_device}, to identify a previously started DHCP server that must be terminated. 
\end{itemize}

\subsubsection*{Return value}
On success, 0 is returned.
On error, -1 is returned, and \texttt{pico$\_$err} is set appropriately.

\subsubsection*{Errors}
\begin{itemize}[noitemsep]
\item PICO$\_$ERR$\_$ENOENT - there was no DHCP server running on the given device.
\end{itemize}

\subsubsection*{Example}
\begin{verbatim}
struct pico_dhcpd_settings s = { };

s.my_ip.addr = long_be(0x0a280001); /* 10.40.0.1 */

pico_dhcp_server_initiate(&s);
\end{verbatim}



\section{DNS client}

% Short description/overview of module functions


\subsection{pico$\_$dns$\_$client$\_$nameserver}

\subsubsection*{Description}
Function to add or remove nameservers.

\subsubsection*{Function prototype}
\begin{verbatim}
int pico_dns_client_nameserver(struct pico_ip4 *ns, uint8_t flag);
\end{verbatim}

\subsubsection*{Parameters}
\begin{itemize}[noitemsep]
\item \texttt{ns} - Pointer to the address of the name server.
\item \texttt{flag} - Flag to indicate addition or removal (see further).
\end{itemize}

\subsubsection*{Flags}
\begin{itemize}[noitemsep]
\item \texttt{PICO$\_$DNS$\_$NS$\_$ADD} - to add a nameserver
\item \texttt{PICO$\_$DNS$\_$NS$\_$DEL} - to remove a nameserver
\end{itemize}

\subsubsection*{Return value}
On success, this call returns 0 if the nameserver operation has succeeded.
On error, -1 is returned and \texttt{pico$\_$err} is set appropriately.

\subsubsection*{Errors}
\begin{itemize}[noitemsep]
\item \texttt{PICO$\_$ERR$\_$EINVAL} - invalid argument
\item \texttt{PICO$\_$ERR$\_$ENOMEM} - not enough space
\item \texttt{PICO$\_$ERR$\_$EAGAIN} - resource temporarily unavailable
\end{itemize}

\subsubsection*{Example}
\begin{verbatim}
ret = pico_dns_client_nameserver(&addr_ns, flag);
\end{verbatim}



\subsection{pico$\_$dns$\_$client$\_$getaddr}

\subsubsection*{Description}
Function to translate an url text string to an internet host address IP. 

\subsubsection*{Function prototype}
\begin{verbatim}
int pico_dns_client_getaddr(const char *url, void (*callback)(char *ip));
\end{verbatim}

\subsubsection*{Parameters}
\begin{itemize}[noitemsep]
\item \texttt{url} - Pointer to text string containing url text string (e.g. www.google.com)
\item \texttt{calback} - Callback function, receiving the internet host address IP
\end{itemize}

\subsubsection*{Return value}
On success, this call returns 0 if the request is sent.
On error, -1 is returned and \texttt{pico$\_$err} is set appropriately.

\subsubsection*{Errors}
\begin{itemize}[noitemsep]
\item \texttt{PICO$\_$ERR$\_$EINVAL} - invalid argument
\item \texttt{PICO$\_$ERR$\_$ENOMEM} - not enough space
\item \texttt{PICO$\_$ERR$\_$EAGAIN} - resource temporarily unavailable
\end{itemize}

\subsubsection*{Example}
\begin{verbatim}
ret = pico_dns_client_getaddr("www.google.com", callback);
\end{verbatim}



\subsection{pico$\_$dns$\_$client$\_$getname}

\subsubsection*{Description}
Function to translate an internet host address IP to an url text string.

\subsubsection*{Function prototype}
\begin{verbatim}
int pico_dns_client_getname(const char *ip, void (*callback)(char *url));
\end{verbatim}

\subsubsection*{Parameters}
\begin{itemize}[noitemsep]
\item \texttt{ip} - Pointer to text string containing an internet host address IP (e.g. 8.8.4.4)
\item \texttt{callback} - Callback function, receiving the url text string
\end{itemize}

\subsubsection*{Return value}
On success, this call returns 0 if the request is sent.
On error, -1 is returned and \texttt{pico$\_$err} is set appropriately.

\subsubsection*{Errors}
\begin{itemize}[noitemsep]
\item \texttt{PICO$\_$ERR$\_$EINVAL} - invalid argument
\item \texttt{PICO$\_$ERR$\_$ENOMEM} - not enough space
\item \texttt{PICO$\_$ERR$\_$EAGAIN} - resource temporarily unavailable
\end{itemize}

\subsubsection*{Example}
\begin{verbatim}
ret = pico_dns_client_getname("8.8.4.4", callback);
\end{verbatim}
\section{IGMP}

% Short description/overview of module functions
This module allows the user to join and leave IGMP multicast groups. The module is based on the IGMP version 3 protocol and it's backwards compatible with version 2. Version 1 is not supported. 
The IGMP module is completly driven from socket calls (\ref{socket:setoption}) and non of the IGMP application interface functions should be called from the user himself. If however, by any reason, it's necessary for the user to do this, the following function call is provided:

\subsection{pico\_igmp\_state\_change}

\subsubsection*{Description}
Change the state of the host to Non-member, Idle member or Delaying member.

\subsubsection*{Function prototype}
\begin{verbatim}
int pico_igmp_state_change(struct pico_ip4 *mcast_link, struct pico_ip4 *mcast_group,
 uint8_t filter_mode, struct pico_tree *_MCASTFilter, uint8_t state)
\end{verbatim}

\subsubsection*{Parameters}
\begin{itemize}[noitemsep]
\item \texttt{mcast\_link} - the link on which that multicast group should be joined.
\item \texttt{mcast\_group} - the address of the multicast group you want to join.
\item \texttt{filter\_mode} - the kind of source filtering, if applied.
\item \texttt{\_MCASTFilter} - list of multicast sources on which source filtering might be applied. 
\item \texttt{state} - the prefered new state.
\end{itemize}

\subsubsection*{Errors}
In case of failure, -1 is returned, and the value of pico$\_$err
is set as follows:

\begin{itemize}[noitemsep]
\item \texttt{PICO$\_$ERR$\_$EINVAL}          - Invalid argument provided
\item \texttt{PICO$\_$ERR$\_$ENOMEM}          - Not enough space
\item \texttt{PICO$\_$ERR$\_$EPROTONOSUPPORT} - Invalid protocol (or protocol version) found on the link
\item \texttt{PICO$\_$ERR$\_$EFAULT}          - Internal error
\end{itemize}

%\subsubsection*{Example}

%\subsubsection*{Errors}

%\subsubsection*{Example}

\section{IP Filter}

% Short description/overview of module functions
This module allows the user to add and remove filters. The user can filter packets based on interface, protocol, outgoing address, outgoing netmask, incomming address, incomming netmask, outgoing port, incomming port, priority and type of service. There are four types of filters: ACCEPT, PRIORITY, REJECT, DROP. When creating a PRIORITY filter, it is necessary to give a priority value in a range between '-10' and '10', '0' as default priority.


\subsection{pico$\_$ipv4$\_$filter$\_$add}

\subsubsection*{Description}
Function to add a filter.

\subsubsection*{Function prototype}
\begin{verbatim}
int pico_ipv4_filter_add(struct pico_device *dev, uint8_t proto,
  struct pico_ip4 out_addr, struct pico_ip4 out_addr_netmask,
  struct pico_ip4 in_addr, struct pico_ip4 in_addr_netmask, uint16_t out_port,
  uint16_t in_port, int8_t priority, uint8_t tos, enum filter_action action);
\end{verbatim}

\subsubsection*{Parameters}
\begin{itemize}[noitemsep]
\item \texttt{dev} - interface to be filtered
\item \texttt{proto} - protocol to be filtered
\item \texttt{out$\_$addr} - outgoing address to be filtered
\item \texttt{out$\_$addr$\_$netmask} - outgoing address-netmask to be filtered
\item \texttt{in$\_$addr} - incomming address to be filtered
\item \texttt{in$\_$addr$\_$netmask} - incomming address-netmask to be filtered
\item \texttt{out$\_$port} - outgoing port to be filtered
\item \texttt{in$\_$port} - incomming port to be filtered
\item \texttt{priority} - priority to be filtered
\item \texttt{tos} - type of service to be filtered
\item \texttt{action} - type of action for the filter: ACCEPT, PRIORITY, REJECT and DROP. ACCEPT, filters all packets selected by the filter. PRIORITY is not yet implemented. REJECT drops all packets and send an ICMP message 'Packet Filtered' (Communication Administratively Prohibited). DROP will discard the packet silently.
\end{itemize}

\subsubsection*{Return value}
On success, this call returns the filter$\_$id from the generated filter. This id must be used when deleting the filter.
On error, -1 is returned and \texttt{pico$\_$err} is set appropriately.

\subsubsection*{Example}
\begin{verbatim}
/* block all incoming traffic on port 5555 *?
filter_id = pico_ipv4_filter_add(NULL, 6, NULL, NULL, NULL, NULL, 0, 5555, 0, 0, FILTER_REJECT);
\end{verbatim}

\subsubsection*{Errors}

\begin{itemize}[noitemsep]
\item \texttt{PICO$\_$ERR$\_$EINVAL} - invalid argument
\end{itemize}


\subsection{pico$\_$ipv4$\_$filter$\_$del}

\subsubsection*{Description}
Function to delete a filter.

\subsubsection*{Function prototype}
\begin{verbatim}
int pico_ipv4_filter_del(int filter_id)
\end{verbatim}

\subsubsection*{Parameters}
\begin{itemize}[noitemsep]
\item \texttt{filter$\_$id} - the id of the filter you want to delete.
\end{itemize}

\subsubsection*{Return value}
On success, this call returns 0.
On error, -1 is returned and \texttt{pico$\_$err} is set appropriately.

\subsubsection*{Errors}

\begin{itemize}[noitemsep]
\item \texttt{PICO$\_$ERR$\_$EINVAL} - invalid argument
\item \texttt{PICO$\_$ERR$\_$EPERM} - operation not permitted
\end{itemize}

\subsubsection*{Example}
\begin{verbatim}
ret = pico_ipv4_filter_del(filter_id);
\end{verbatim}


%\subsubsection*{Parameters}
%\subsubsection*{Return value}
%\subsubsection*{Errors}
%\subsubsection*{Example}


\section{ZMQ calls} 
% Short description/overview of module functions
The zmq library has been implemented in a slightly different way because it is not
possible to use blocking calls.

\subsection{zmq$\_$socket}

\subsubsection*{Description}
This function creates a new zmq socket of a particular type.

\subsubsection*{Function prototype}
\begin{verbatim}
void *zmq_socket(void* context, int type);
\end{verbatim}

\subsubsection*{Parameters}
\begin{itemize}[noitemsep]
\item \texttt{context} - Context of the socket
\item \texttt{type} - Possible types at this moment:
\begin{itemize}[noitemsep]
\item \texttt{ZMQ$\_$PUB}
\end{itemize}
\end{itemize}

\subsubsection*{Possible events for sockets}
\begin{itemize}[noitemsep]
\item \texttt{PICO$\_$SOCK$\_$EV$\_$RD} - bla bla.
\end{itemize}

\subsubsection*{Return value}
On success, this call returns a pointer to the declared socket (\texttt{struct pico$\_$socket *}).
On error the socket is not created, \texttt{NULL} is returned, and \texttt{pico$\_$err} is set appropriately.

\subsubsection*{Errors}
\begin{itemize}[noitemsep]
\item \texttt{PICO$\_$ERR$\_$EINVAL} - invalid argument
\item \texttt{PICO$\_$ERR$\_$EPROTONOSUPPORT} - protocol not supported
\item \texttt{PICO$\_$ERR$\_$ENETUNREACH} - network unreachable 
\end{itemize}

\subsubsection*{Example}
\begin{verbatim}
sk_tcp = pico_socket_open(PICO_PROTO_IPV4, PICO_PROTO_TCP, &wakeup);
\end{verbatim}


\subsection{pico$\_$socket$\_$read}



\chapter{Examples}
\label{chap:examples}
The following sections will give code examples of PicoTCP.
It is assumed that all examples include the appropriate header files
and a \textbf{main} routine that calls the \texttt{app$\_$x} functions to initialize
the example.

The most common header files are:
\begin{verbatim}
#include "pico_stack.h"
#include "pico_config.h"
#include "pico_dev_vde.h"
#include "pico_ipv4.h"
#include "pico_socket.h"
#include "pico_dev_tun.h"
#include "pico_nat.h"
#include "pico_icmp4.h"
#include "pico_dns_client.h"
#include "pico_dev_loop.h"
#include "pico_dhcp_client.h"
#include "pico_dhcp_server.h"
#include "pico_ipfilter.h"
\end{verbatim}

\section{Ping example}

\begin{verbatim}
#define NUM_PING 10

/* callback function for receiving ping reply */
void cb_ping(struct pico_icmp4_stats *s)
{
  char host[30];
  int time_sec = 0;
  int time_msec = 0;
  
  /* convert ip address from icmp4_stats structure to string */
  pico_ipv4_to_string(host, s->dst.addr);
  
  /* get time information from icmp4_stats structure */
  time_sec = s->time / 1000;
  time_msec = s->time % 1000;
  
  if (s->err == PICO_PING_ERR_REPLIED) {
  	/* print info if no error reported in icmp4_stats structure */
    dbg("%lu bytes from %s: icmp_req=%lu ttl=%lu time=%lu ms\n", \
    					s->size, host, s->seq, s->ttl, s->time);
    if (s->seq >= NUM_PING)
      exit(0);
  } else {
  	/* else, print error info */
    dbg("PING %lu to %s: Error %d\n", s->seq, host, s->err);
    exit(1);
  }
}

/* initialize the ping command */
void app_ping(char *dest)
{
  pico_icmp4_ping(dest, NUM_PING, 1000, 5000, 48, cb_ping);
}
\end{verbatim}


\section{UDP echo socket example}

\begin{verbatim}
struct pico_ip4 inaddr_any = { };

/* callback for UDP echo socket events */
void cb_udpecho(uint16_t ev, struct pico_socket *s)
{
  char recvbuf[1400];
  int read = 0;
  uint32_t peer;
  uint16_t port;

  /* process read event, data available */
  if (ev == PICO_SOCK_EV_RD) {
  	/* while data available in socket buffer, echo data to peer */
    do {
      read = pico_socket_recvfrom(s, recvbuf, 1400, &peer, &port);
      if (read > 0)
        pico_socket_sendto(s, recvbuf, r, &peer, port);
    } while(read > 0);
  }

  /* process error event, socket error occured */
  if (ev == PICO_SOCK_EV_ERR) {
    printf("Socket Error received. Bailing out.\n");
    exit(1);
  }

  printf("Received data from %08X:%u\n", peer, port);
}

/* initialize the UDP echo socket */
void app_udpecho(uint16_t source_port)
{
  struct pico_socket *s;
  uint16_t port_be = 0;
  
  /* set the source port for the socket */
  if (source_port == 0)
    port_be = short_be(5555);
  else
    port_be = short_be(source_port);

  /* open a UDP socket with the appropriate callback */
  s = pico_socket_open(PICO_PROTO_IPV4, PICO_PROTO_UDP, &cb_udpecho);
  if (!s)
    exit(1);

  /* bind the socket to port_be */
  if (pico_socket_bind(s, &inaddr_any, &port_be) != 0)
    exit(1);
}
\end{verbatim}


\section{TCP echo socket example}

\begin{verbatim}
#define BSIZE 1460

/* callback for TCP echo socket events */
void cb_tcpecho(uint16_t ev, struct pico_socket *s)
{
  char recvbuf[BSIZE];
  int read = 0, written = 0;
  int pos = 0, len = 0;
  struct pico_socket *sock_a;
  struct pico_ip4 orig;
  uint16_t port;
  char peer[30];

  /* process read event, data available */
  if (ev & PICO_SOCK_EV_RD) {
    do {
      read = pico_socket_read(s, recvbuf + len, BSIZE - len);
      if (read > 0)
        len += r;
    } while(read > 0);
  }
  
  /* process connect event, syn received */
  if (ev & PICO_SOCK_EV_CONN) {
    /* accept new connection request */
    sock_a = pico_socket_accept(s, &orig, &port);
    
   	/* convert peer IP to string */
    pico_ipv4_to_string(peer, orig.addr);
    
    /* print info */
    printf("Connection established with %s:%d.\n", peer, short_be(port));
  }

  /* process fin event, receiving socket closed */
  if (ev & PICO_SOCK_EV_FIN) {
    printf("Socket closed. Exit normally. \n");
  }

  /* process error event, socket error occured */
  if (ev & PICO_SOCK_EV_ERR) {
    printf("Socket Error received: %s. Bailing out.\n", strerror(pico_err));
    exit(1);
  }
  
  /* process close event, receiving socket received close from peer */
  if (ev & PICO_SOCK_EV_CLOSE) {
    printf("Socket received close from peer.\n");
    /* shutdown write side of socket */
    pico_socket_shutdown(s, PICO_SHUT_WR);
  }

  /* if data read, echo back */
  if (len > pos) {
    do {
      /* echo data back to peer */
      written = pico_socket_write(s, recvbuf + pos, len - pos);
      if (written > 0) {
        pos += written;
        if (pos >= len) {
          pos = 0;
          len = 0;
          written = 0;
        }
      } else {
        printf("SOCKET> ECHO write failed, dropped %d bytes\n",(len-pos));
      }
    } while(written > 0);
  }
}

/* initialize the TCP echo socket */
void app_tcpecho(uint16_t source_port)
{
  struct pico_socket *s;
  uint16_t port_be = 0;
  int backlog = 40;			/* max number of accepting connections */
  int ret;
  
  /* set the source port for the socket */
  if (source_port == 0)
    port_be = short_be(5555);
  else
    port_be = short_be(source_port);

  /* open a TCP socket with the appropriate callback */
  s = pico_socket_open(PICO_PROTO_IPV4, PICO_PROTO_TCP, &cb_tcpecho);
  if (!s)
    exit(1);

  /* bind the socket to port_be */
  ret = pico_socket_bind(s, &inaddr_any, &port_be);
  if (ret != 0)
    exit(1);

  /* start listening on socket */
  ret = pico_socket_listen(s, backlog);
  if (ret != 0)
    exit(1);
}
\end{verbatim}


\section{NAT setup example}

\begin{verbatim}
/* initialize NAT functionality and add port forward rule */
void app_nat(char *dest)
{
  char *dest = NULL;
  struct pico_ip4 ipdst, pub_addr, priv_addr;
  struct pico_ipv4_link *link;

  /* convert IP address of link where to enable NAT */
  pico_string_to_ipv4(dest, &ipdst.addr);
  
  /* get link pointer */
  link = pico_ipv4_link_get(&ipdst);
  if (!link) {
    printf("destination not found\n");
    exit(1);
  }
  
  /* enable NAT on link */
  pico_ipv4_nat_enable(link);
  
  /* add port forward rule */
  pico_string_to_ipv4("10.50.0.10", &pub_addr.addr);
  pico_string_to_ipv4("10.40.0.08", &priv_addr.addr);
  pico_ipv4_port_forward(pub_addr, short_be(5555), priv_addr, short_be(6667),
  PICO_PROTO_UDP, PICO_IPV4_FORWARD_ADD);
  
  printf("nat started\n");
}
\end{verbatim}


\section{DNS example}

\begin{verbatim}
/* identifier struct */
struct dns_identifier {
  uint8_t id;
  /* ... */
};

/* callback function of URL translation */
void cb_getaddr(char *ip, void *arg)
{
  struct dns_identifier *id_getaddr = (struct dns_identifier *) arg;
  
  /* NULL indicates an error condition */
  if (!ip) {
    printf("DNS error occured: %s\n", strerror(pico_err));
    return;
  }
  printf("DNS translation to ip %s (id %u)\n", ip, id_getaddr ? id_getaddr->id : 0);
  
  /* important: free the received pointers! */
  pico_free(ip);
  if (id_getaddr)
    pico_free(id_getaddr);
}

/* callback function of IP translation */
void cb_getname(char *url)
{
  struct dns_identifier *id_getname = (struct dns_identifier *) arg;
  
  /* NULL indicates an error condition */
  if (!url) {
    printf("DNS error occured: %s\n", strerror(pico_err));
    return;
  }
  printf("DNS translation to url %s (id %u)\n", url, id_getname ? id_getname->id : 0);
  
  /* important: free the received pointers! */
  pico_free(url);
  if (id_getname)
    pico_free(id_getname);
}

/* initialize the dns */
void app_dns(char *url, char *ip)
{
  struct pico_ip4 nameserver = { };
  struct dns_identifier *id_getaddr = NULL, *id_getname = NULL;
  
  /* optional: add custom dns nameserver */
  pico_string_to_ipv4("8.8.4.4", &nameserver.addr);
  pico_dns_client_nameserver(&nameserver, PICO_DNS_NS_ADD);

  /* request translation of URL f.e. www.google.com */  
  id_getaddr = pico_zalloc(sizeof(struct dns_identifier));
  id_getaddr->id = 1;
  pico_dns_client_getaddr(url, &cb_getaddr, id_getaddr);

  /* request translation of IP f.e. 8.8.8.8 */
  id_getname = pico_zalloc(sizeof(struct dns_identifier));
  id_getname->id = 2;
  pico_dns_client_getname(ip, &cb_getname, id_getname);
}
\end{verbatim}


\section{DHCP client example}

\begin{verbatim}
int main(void)
{
  uint8_t mac_eth0[6] = {0x12, 0x34, 0x56, 0x78, 0x9a, 0xbc};
  uint8_t mac_eth1[6] = {0xcb, 0xa9, 0x87, 0x65, 0x43, 0x21};
  char s_addr_eth0[16] = { }, s_addr_eth1[16] = { };
  void *identifier_eth0 = NULL, *identifier_eth1 = NULL;
  uint32_t xid_eth0 = 0, xid_eth1 = 0;
  struct pico_device *eth0 = NULL, *eth1 = NULL;
  struct pico_ip4 addr_eth0 = { }, addr_eth1 = { };

  /* see section 2.5 Network devices integration */
  eth0 = pico_device_create("eth0", mac_eth0);
  eth1 = pico_device_create("eth1", mac_eth1);
  
  pico_stack_init();
  
  if (pico_dhcp_initiate_negotiation(eth0, &cb_dhcpclient, &xid_eth0) < 0) {
      printf("DHCPC: error initiating negotiation: %s\n", strerror(pico_err));
      exit(255);
  } 
  if (pico_dhcp_initiate_negotiation(eth1, &cb_dhcpclient, &xid_eth1) < 0) {
      printf("DHCPC: error initiating negotiation: %s\n", strerror(pico_err));
      exit(255);
  }
  
  for(;;) {
    pico_stack_tick();
    /* did both devices get a successful lease? */
    if (xid_eth0 && xid_eth1)
      break;
    PICO_IDLE();
  }
  
  identifier_eth0 = pico_dhcp_get_identifier(xid_eth0);
  addr_eth0 = pico_dhcp_get_address(identifier_eth0);
  pico_ipv4_to_string(s_addr_eth0, addr_eth0.addr);
  printf("Device %s got leased IP %s\n", eth0->name, s_addr_eth0);
  
  identifier_eth1 = pico_dhcp_get_identifier(xid_eth1);
  addr_eth1 = pico_dhcp_get_address(identifier_eth1);
  pico_ipv4_to_string(s_addr_eth1, addr_eth1.addr);
  printf("Device %s got leased IP %s\n", eth1->name, s_addr_eth1);
  
  return 0;
}
\end{verbatim}

\section{HTTP Client example}
\begin{verbatim}
static char *url_filename = NULL;

static int http_save_file(void *data, int len)
{
  int fd = open(url_filename, O_WRONLY |O_CREAT | O_TRUNC, 0660);
  int w, e;
  if (fd < 0)
    return fd;

  printf("Saving data to : %s\n",url_filename);
  w = write(fd, data, len);
  e = errno;
  close(fd);
  errno = e;
  return w;
}
void wget_callback(uint16_t ev, uint16_t conn)
{
  char data[1024 * 1024]; // MAX: 1M
  static int _length = 0;

  if(ev & EV_HTTP_CON)
  {
    printf(">>> Connected to the client \n");
    /* you can let the client use the default generated header
       or you can create you own string header (compatible with HTTP/1.x */
    pico_http_client_sendHeader(conn,NULL,HTTP_HEADER_DEFAULT);
  }

  if(ev & EV_HTTP_REQ)
  {
    struct pico_http_header * header = pico_http_client_readHeader(conn);
    printf("Received header from server...\n");
    printf("Server response : %d\n",header->responseCode);
    printf("Location : %s\n",header->location);
    printf("Transfer-Encoding : %d\n",header->transferCoding);
    printf("Size/Chunk : %d\n",header->contentLengthOrChunk);
  }

  if(ev & EV_HTTP_BODY)
  {
    int len;

    printf("Reading data...\n");
    /*
      Data is passed to you without you worrying if the transfer is 
      chunked or the content-length was specified.
    */
    while((len = pico_http_client_readData(conn,data + _length,1024)))
    {
      _length += len;
    }
  }

  if(ev & EV_HTTP_CLOSE)
  {
    struct pico_http_header * header = pico_http_client_readHeader(conn);
    int len;
    printf("Connection was closed...\n");
    printf("Reading remaining data, if any ...\n");
    while((len = pico_http_client_readData(conn,data,1000u)) && len > 0)
    {
      _length += len;
    }
    printf("Read a total data of : %d bytes \n",_length);

    if(header->transferCoding == HTTP_TRANSFER_CHUNKED)
    {
      if(header->contentLengthOrChunk)
      {
        printf("Last chunk data not fully read !\n");
        exit(1);
      }
      else
      {
        printf("Transfer ended with a zero chunk! OK !\n");
      }
    } else
    {
      if(header->contentLengthOrChunk == _length)
      {
        printf("Received the full : %d \n",_length);
      }
      else
      {
        printf("Received %d , waiting for %d\n",_length, header->contentLengthOrChunk);
        exit(1);
      }
    }

    if (!url_filename) {
      printf("Failed to get local filename\n");
      exit(1);
    }

    if (http_save_file(data, _length) < _length) {
      printf("Failed to save file: %s\n", strerror(errno));
      exit(1);
    }
    pico_http_client_close(conn);
    exit(0);
  }

  if(ev & EV_HTTP_ERROR)
  {
    printf("Connection error (probably dns failed : check the routing table), trying to close the client...\n");
    pico_http_client_close(conn);
    exit(1u);
  }

  if(ev & EV_HTTP_DNS)
  {
    printf("The DNS query was successful ... \n");
  }
}

void app_wget(char *arg)
{
  char * url;
  cpy_arg(&url, arg);

  if(!url)
  {
    fprintf(stderr, " wget expects the url to be received\n");
    exit(1);
  }
  
  // when opening the http client it will internally parse the url passed
  if(pico_http_client_open(url,wget_callback) < 0)
  {
    fprintf(stderr," error opening the url : %s, please check the format\n",url);
    exit(1);
  }
  url_filename = basename(url);
}
\end{verbatim}

\section{HTTP Server example}
\begin{verbatim}

#define SIZE 4*1024

void serverWakeup(uint16_t ev,uint16_t conn)
{
  static FILE * f;
  char buffer[SIZE];

  if(ev & EV_HTTP_CON)
  {
      printf("New connection received....\n");
      pico_http_server_accept();
  }

  if(ev & EV_HTTP_REQ) // new header received
  {
    int read;
    char * resource;
    printf("Header request was received...\n");
    printf("> Resource : %s\n",pico_http_getResource(conn));
    resource = pico_http_getResource(conn);

    if(strcmp(resource,"/")==0 || strcmp(resource,"index.html") == 0 || strcmp(resource,"/index.html") == 0)
    {
          // Accepting request
          printf("Accepted connection...\n");
          pico_http_respond(conn,HTTP_RESOURCE_FOUND);
          f = fopen("test/examples/index.html","r");

          if(!f)
          {
            fprintf(stderr,"Unable to open the file /test/examples/index.html\n");
            exit(1);
          }

          read = fread(buffer,1,SIZE,f);
          pico_http_submitData(conn,buffer,read);
    }
    else
    { // reject
      printf("Rejected connection...\n");
      pico_http_respond(conn,HTTP_RESOURCE_NOT_FOUND);
    }

  }

  if(ev & EV_HTTP_PROGRESS) // submitted data was sent
  {
    uint16_t sent, total;
    pico_http_getProgress(conn,&sent,&total);
    printf("Chunk statistics : %d/%d sent\n",sent,total);
  }

  if(ev & EV_HTTP_SENT) // submitted data was fully sent
  {
    int read;
    read = fread(buffer,1,SIZE,f);
    printf("Chunk was sent...\n");
    if(read > 0)
    {
        printf("Sending another chunk...\n");
        pico_http_submitData(conn,buffer,read);
    }
    else
    {
        printf("Last chunk !\n");
        pico_http_submitData(conn,NULL,0);// send the final chunk
        fclose(f);
    }
  }

  if(ev & EV_HTTP_CLOSE)
  {
    printf("Close request...\n");
    pico_http_close(conn);
  }

  if(ev & EV_HTTP_ERROR)
  {
    printf("Error on server...\n");
    pico_http_close(conn);
  }
}
/* simple server example that serves the index(.html) page */
void app_httpd(char *arg)
{
  /* transfer encoding with this server is always chunked and you can 
     submit chunks to the client, without needing to specify the content-length of the
     body response */ 
  if( pico_http_server_start(0,serverWakeup) < 0)
  {
    fprintf(stderr,"Unable to start the server on port 80\n");
  }
}
\end{verbatim}




\appendix

% Do not include license
%\chapter{License}
%\label{chap:license}
%Unless you have received a written document by PicoTCP copyright holders stating otherwise,
the software described in this document is distributed under the terms of the GNU General 
Public License version 2 only.

The terms of the license are reported below.

\begin{center}
{\bf\large GNU General Public license}
{\bf Version 2, June 1991}
\end{center}

\begin{center}
{\parindent 0in

Copyright \copyright\ 1989, 1991 Free Software Foundation, Inc.

\bigskip

51 Franklin Street, Fifth Floor, Boston, MA  02110-1301, USA

\bigskip

Everyone is permitted to copy and distribute verbatim copies
of this license document, but changing it is not allowed.
}
\end{center}

\begin{center}
{\bf\large Preamble}
\end{center}


The licenses for most software are designed to take away your freedom to
share and change it.  By contrast, the GNU General Public License is
intended to guarantee your freedom to share and change free software---to
make sure the software is free for all its users.  This General Public
License applies to most of the Free Software Foundation's software and to
any other program whose authors commit to using it.  (Some other Free
Software Foundation software is covered by the GNU Library General Public
License instead.)  You can apply it to your programs, too.

When we speak of free software, we are referring to freedom, not price.
Our General Public Licenses are designed to make sure that you have the
freedom to distribute copies of free software (and charge for this service
if you wish), that you receive source code or can get it if you want it,
that you can change the software or use pieces of it in new free programs;
and that you know you can do these things.

To protect your rights, we need to make restrictions that forbid anyone to
deny you these rights or to ask you to surrender the rights.  These
restrictions translate to certain responsibilities for you if you
distribute copies of the software, or if you modify it.

For example, if you distribute copies of such a program, whether gratis or
for a fee, you must give the recipients all the rights that you have.  You
must make sure that they, too, receive or can get the source code.  And
you must show them these terms so they know their rights.

We protect your rights with two steps: (1) copyright the software, and (2)
offer you this license which gives you legal permission to copy,
distribute and/or modify the software.

Also, for each author's protection and ours, we want to make certain that
everyone understands that there is no warranty for this free software.  If
the software is modified by someone else and passed on, we want its
recipients to know that what they have is not the original, so that any
problems introduced by others will not reflect on the original authors'
reputations.

Finally, any free program is threatened constantly by software patents.
We wish to avoid the danger that redistributors of a free program will
individually obtain patent licenses, in effect making the program
proprietary.  To prevent this, we have made it clear that any patent must
be licensed for everyone's free use or not licensed at all.

The precise terms and conditions for copying, distribution and
modification follow.

\begin{center}
{\Large \sc Terms and Conditions For Copying, Distribution and
  Modification}
\end{center}


%\renewcommand{\theenumi}{\alpha{enumi}}
\begin{enumerate}

\addtocounter{enumi}{-1}

\item 

This License applies to any program or other work which contains a notice
placed by the copyright holder saying it may be distributed under the
terms of this General Public License.  The ``Program'', below, refers to
any such program or work, and a ``work based on the Program'' means either
the Program or any derivative work under copyright law: that is to say, a
work containing the Program or a portion of it, either verbatim or with
modifications and/or translated into another language.  (Hereinafter,
translation is included without limitation in the term ``modification''.)
Each licensee is addressed as ``you''.

Activities other than copying, distribution and modification are not
covered by this License; they are outside its scope.  The act of
running the Program is not restricted, and the output from the Program
is covered only if its contents constitute a work based on the
Program (independent of having been made by running the Program).
Whether that is true depends on what the Program does.

\item You may copy and distribute verbatim copies of the Program's source
  code as you receive it, in any medium, provided that you conspicuously
  and appropriately publish on each copy an appropriate copyright notice
  and disclaimer of warranty; keep intact all the notices that refer to
  this License and to the absence of any warranty; and give any other
  recipients of the Program a copy of this License along with the Program.

You may charge a fee for the physical act of transferring a copy, and you
may at your option offer warranty protection in exchange for a fee.

\item

You may modify your copy or copies of the Program or any portion
of it, thus forming a work based on the Program, and copy and
distribute such modifications or work under the terms of Section 1
above, provided that you also meet all of these conditions:

\begin{enumerate}

\item 

You must cause the modified files to carry prominent notices stating that
you changed the files and the date of any change.

\item

You must cause any work that you distribute or publish, that in
whole or in part contains or is derived from the Program or any
part thereof, to be licensed as a whole at no charge to all third
parties under the terms of this License.

\item
If the modified program normally reads commands interactively
when run, you must cause it, when started running for such
interactive use in the most ordinary way, to print or display an
announcement including an appropriate copyright notice and a
notice that there is no warranty (or else, saying that you provide
a warranty) and that users may redistribute the program under
these conditions, and telling the user how to view a copy of this
License.  (Exception: if the Program itself is interactive but
does not normally print such an announcement, your work based on
the Program is not required to print an announcement.)

\end{enumerate}


These requirements apply to the modified work as a whole.  If
identifiable sections of that work are not derived from the Program,
and can be reasonably considered independent and separate works in
themselves, then this License, and its terms, do not apply to those
sections when you distribute them as separate works.  But when you
distribute the same sections as part of a whole which is a work based
on the Program, the distribution of the whole must be on the terms of
this License, whose permissions for other licensees extend to the
entire whole, and thus to each and every part regardless of who wrote it.

Thus, it is not the intent of this section to claim rights or contest
your rights to work written entirely by you; rather, the intent is to
exercise the right to control the distribution of derivative or
collective works based on the Program.

In addition, mere aggregation of another work not based on the Program
with the Program (or with a work based on the Program) on a volume of
a storage or distribution medium does not bring the other work under
the scope of this License.

\item
You may copy and distribute the Program (or a work based on it,
under Section 2) in object code or executable form under the terms of
Sections 1 and 2 above provided that you also do one of the following:

\begin{enumerate}

\item

Accompany it with the complete corresponding machine-readable
source code, which must be distributed under the terms of Sections
1 and 2 above on a medium customarily used for software interchange; or,

\item

Accompany it with a written offer, valid for at least three
years, to give any third party, for a charge no more than your
cost of physically performing source distribution, a complete
machine-readable copy of the corresponding source code, to be
distributed under the terms of Sections 1 and 2 above on a medium
customarily used for software interchange; or,

\item

Accompany it with the information you received as to the offer
to distribute corresponding source code.  (This alternative is
allowed only for noncommercial distribution and only if you
received the program in object code or executable form with such
an offer, in accord with Subsection b above.)

\end{enumerate}


The source code for a work means the preferred form of the work for
making modifications to it.  For an executable work, complete source
code means all the source code for all modules it contains, plus any
associated interface definition files, plus the scripts used to
control compilation and installation of the executable.  However, as a
special exception, the source code distributed need not include
anything that is normally distributed (in either source or binary
form) with the major components (compiler, kernel, and so on) of the
operating system on which the executable runs, unless that component
itself accompanies the executable.

If distribution of executable or object code is made by offering
access to copy from a designated place, then offering equivalent
access to copy the source code from the same place counts as
distribution of the source code, even though third parties are not
compelled to copy the source along with the object code.

\item
You may not copy, modify, sublicense, or distribute the Program
except as expressly provided under this License.  Any attempt
otherwise to copy, modify, sublicense or distribute the Program is
void, and will automatically terminate your rights under this License.
However, parties who have received copies, or rights, from you under
this License will not have their licenses terminated so long as such
parties remain in full compliance.

\item
You are not required to accept this License, since you have not
signed it.  However, nothing else grants you permission to modify or
distribute the Program or its derivative works.  These actions are
prohibited by law if you do not accept this License.  Therefore, by
modifying or distributing the Program (or any work based on the
Program), you indicate your acceptance of this License to do so, and
all its terms and conditions for copying, distributing or modifying
the Program or works based on it.

\item
Each time you redistribute the Program (or any work based on the
Program), the recipient automatically receives a license from the
original licensor to copy, distribute or modify the Program subject to
these terms and conditions.  You may not impose any further
restrictions on the recipients' exercise of the rights granted herein.
You are not responsible for enforcing compliance by third parties to
this License.

\item
If, as a consequence of a court judgment or allegation of patent
infringement or for any other reason (not limited to patent issues),
conditions are imposed on you (whether by court order, agreement or
otherwise) that contradict the conditions of this License, they do not
excuse you from the conditions of this License.  If you cannot
distribute so as to satisfy simultaneously your obligations under this
License and any other pertinent obligations, then as a consequence you
may not distribute the Program at all.  For example, if a patent
license would not permit royalty-free redistribution of the Program by
all those who receive copies directly or indirectly through you, then
the only way you could satisfy both it and this License would be to
refrain entirely from distribution of the Program.

If any portion of this section is held invalid or unenforceable under
any particular circumstance, the balance of the section is intended to
apply and the section as a whole is intended to apply in other
circumstances.

It is not the purpose of this section to induce you to infringe any
patents or other property right claims or to contest validity of any
such claims; this section has the sole purpose of protecting the
integrity of the free software distribution system, which is
implemented by public license practices.  Many people have made
generous contributions to the wide range of software distributed
through that system in reliance on consistent application of that
system; it is up to the author/donor to decide if he or she is willing
to distribute software through any other system and a licensee cannot
impose that choice.

This section is intended to make thoroughly clear what is believed to
be a consequence of the rest of this License.

\item
If the distribution and/or use of the Program is restricted in
certain countries either by patents or by copyrighted interfaces, the
original copyright holder who places the Program under this License
may add an explicit geographical distribution limitation excluding
those countries, so that distribution is permitted only in or among
countries not thus excluded.  In such case, this License incorporates
the limitation as if written in the body of this License.

\item
The Free Software Foundation may publish revised and/or new versions
of the General Public License from time to time.  Such new versions will
be similar in spirit to the present version, but may differ in detail to
address new problems or concerns.

Each version is given a distinguishing version number.  If the Program
specifies a version number of this License which applies to it and ``any
later version'', you have the option of following the terms and conditions
either of that version or of any later version published by the Free
Software Foundation.  If the Program does not specify a version number of
this License, you may choose any version ever published by the Free Software
Foundation.

\item
If you wish to incorporate parts of the Program into other free
programs whose distribution conditions are different, write to the author
to ask for permission.  For software which is copyrighted by the Free
Software Foundation, write to the Free Software Foundation; we sometimes
make exceptions for this.  Our decision will be guided by the two goals
of preserving the free status of all derivatives of our free software and
of promoting the sharing and reuse of software generally.

\begin{center}
{\Large\sc
No Warranty
}
\end{center}

\item
{\sc Because the program is licensed free of charge, there is no warranty
for the program, to the extent permitted by applicable law.  Except when
otherwise stated in writing the copyright holders and/or other parties
provide the program ``as is'' without warranty of any kind, either expressed
or implied, including, but not limited to, the implied warranties of
merchantability and fitness for a particular purpose.  The entire risk as
to the quality and performance of the program is with you.  Should the
program prove defective, you assume the cost of all necessary servicing,
repair or correction.}

\item
{\sc In no event unless required by applicable law or agreed to in writing
will any copyright holder, or any other party who may modify and/or
redistribute the program as permitted above, be liable to you for damages,
including any general, special, incidental or consequential damages arising
out of the use or inability to use the program (including but not limited
to loss of data or data being rendered inaccurate or losses sustained by
you or third parties or a failure of the program to operate with any other
programs), even if such holder or other party has been advised of the
possibility of such damages.}

\end{enumerate}


\begin{center}
{\Large\sc End of Terms and Conditions}
\end{center}


\pagebreak[2]

\section*{Appendix: How to Apply These Terms to Your New Programs}

If you develop a new program, and you want it to be of the greatest
possible use to the public, the best way to achieve this is to make it
free software which everyone can redistribute and change under these
terms.

  To do so, attach the following notices to the program.  It is safest to
  attach them to the start of each source file to most effectively convey
  the exclusion of warranty; and each file should have at least the
  ``copyright'' line and a pointer to where the full notice is found.

\begin{quote}
one line to give the program's name and a brief idea of what it does. \\
Copyright (C) yyyy  name of author \\

This program is free software; you can redistribute it and/or modify
it under the terms of the GNU General Public License as published by
the Free Software Foundation; either version 2 of the License, or
(at your option) any later version.

This program is distributed in the hope that it will be useful,
but WITHOUT ANY WARRANTY; without even the implied warranty of
MERCHANTABILITY or FITNESS FOR A PARTICULAR PURPOSE.  See the
GNU General Public License for more details.

You should have received a copy of the GNU General Public License
along with this program; if not, write to the Free Software
Foundation, Inc., 51 Franklin Street, Fifth Floor, Boston, MA  02110-1301, USA.
\end{quote}

Also add information on how to contact you by electronic and paper mail.

If the program is interactive, make it output a short notice like this
when it starts in an interactive mode:

\begin{quote}
Gnomovision version 69, Copyright (C) yyyy  name of author \\
Gnomovision comes with ABSOLUTELY NO WARRANTY; for details type `show w'. \\
This is free software, and you are welcome to redistribute it
under certain conditions; type `show c' for details.
\end{quote}


The hypothetical commands {\tt show w} and {\tt show c} should show the
appropriate parts of the General Public License.  Of course, the commands
you use may be called something other than {\tt show w} and {\tt show c};
they could even be mouse-clicks or menu items---whatever suits your
program.

You should also get your employer (if you work as a programmer) or your
school, if any, to sign a ``copyright disclaimer'' for the program, if
necessary.  Here is a sample; alter the names:

\begin{quote}
Yoyodyne, Inc., hereby disclaims all copyright interest in the program \\
`Gnomovision' (which makes passes at compilers) written by James Hacker. \\

signature of Ty Coon, 1 April 1989 \\
Ty Coon, President of Vice
\end{quote}


This General Public License does not permit incorporating your program
into proprietary programs.  If your program is a subroutine library, you
may consider it more useful to permit linking proprietary applications
with the library.  If this is what you want to do, use the GNU Library
General Public License instead of this License.




\chapter{Supported RFC's}
\label{chap:rfcs}

\begin{longtable}{ | l | p{15cm} | }
\hline
{\bf RFC} & 
{\bf Description} \\ \hline 

RFC 768 &
User Datagram Protocol (UDP) \\ \hline

RFC 791 &
Internet Protocol (IP) \\ \hline

RFC 792 &
Internet Control Message Protocol (ICMP) \\ \hline

RFC 793 &
Transmission Control Protocol (TCP) \\ \hline

RFC 816 &
Fault Isolation and Recovery \\ \hline

RFC 826 &
Address Resolution Protocol (ARP) \\ \hline

RFC 879 &
The TCP Maximum Segment Size and Related Topics \\ \hline

RFC 894 &
IP over Ethernet \\ \hline

RFC 896 &
Congestion Control in IP/TCP Internetworks \\ \hline

RFC 919 &
Broadcasting Internet Datagrams \\ \hline

RFC 922 &
Broadcasting Internet Datagrams in the Presence of Subnets \\ \hline

RFC 950 &
Internet Standard Subnetting Procedure \\ \hline

RFC 1009 &
Requirements for Internet Gateways \\ \hline

RFC 1034 &
Domain NamesConcepts and Facilities \\ \hline

RFC 1035 &
Domain NamesImplementation and Specification \\ \hline

RFC 1071 &
Computing the Internet Checksum \\ \hline

RFC 1112 &
Internet Group Management Protocol (IGMP) \\ \hline

RFC 1122 &
Requirements for Internet HostsCommunication Layers \\ \hline

RFC 1191 &
Path MTU Discovery \\ \hline

RFC 1323 &
TCP Extensions for High Performance \\ \hline

RFC 1337 &
TIME-WAIT Assassination Hazards in TCP \\ \hline

RFC 1350 &
THE TFTP PROTOCOL (REVISION 2) \\ \hline

RFC 1534 &
Interoperation Between DHCP and BOOTP \\ \hline

RFC 1542 &
Clarifications and Extensions for the Bootstrap Protocol \\ \hline

RFC 1812 &
Requirements for IP Version 4 Routers \\ \hline

RFC 1878 &
Variable Length Subnet Table For IPv4 \\ \hline

RFC 1886 &
DNS Extensions to Support IP Version 6 (\textsuperscript{1}) \\ \hline

RFC 2018 &
TCP Selective Acknowledgment Options \\ \hline

RFC 2131 &
Dynamic Host Configuration Protocol (DHCP) \\ \hline

RFC 2132 &
DHCP Options and BOOTP Vendor Extensions \\ \hline

RFC 2236 &
Internet Group Management Protocol, Version 2 \\ \hline

RFC 2460 &
Internet Protocol, Version 6 (IPv6) Specification (\textsuperscript{1}) \\ \hline

RFC 2581 &
TCP Congestion Control \\ \hline

RFC 2616 &
Hypertext Transfer Protocol -- HTTP/1.1 \\ \hline

RFC 2663 &
IP Network Address Translator (NAT) Terminology and Considerations \\ \hline

RFC 3042 &
Enhancing TCP's Loss Recovery Using Limited Transmit \\ \hline

RFC 3315 &
Dynamic Host Configuration Protocol for IPv6 (DHCPv6) (\textsuperscript{1}) \\ \hline

RFC 3376 &
Internet Group Management Protocol, Version 3 (\textsuperscript{2}) \\ \hline

RFC 3517 &
A Conservative Selective Acknowledgment (SACK)-based Loss Recovery Algorithm for TCP \\ \hline

RFC 3782 &
The NewReno Modification to TCP's Fast Recovery Algorithm \\ \hline

RFC 3927 &
Dynamic Configuration of IPv4 Link-Local Addresses \\ \hline

RFC 4291 &
IP Version 6 Addressing Architecture (\textsuperscript{1}) \\ \hline

RFC 6691 &
TCP Options and Maximum Segment Size (MSS) \\ \hline

\end{longtable}

(\textsuperscript{1}) Work in progress
(\textsuperscript{2}) Experimental


\end{document}
