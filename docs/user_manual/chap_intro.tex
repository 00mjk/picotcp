PicoTCP is a complete TCP/IP stack designed for embedded devices and
intended to be run on several different architectures and networking
hardware. The architecture of the stack allows to select the features
needed for any particular use easily, taking into account the sizing and
the performance of the platform where the code runs. Even if it is designed
to respect size and performance constraint, the chosen approach is to
respect the latest standards in the telecommunications research, including
the latest proposals, in order to respect the highest standards for
today's inter-networking communications. PicoTCP is distributed in the form
of a library to be integrated with application and form a combination for
any hardware-specific firmware.


The man characteristic of the library are the following:
\begin{itemize}
\item \textbf{Modularity} Each component of the stack is deployed in a
separate module, allowing the user to select at compile time what needs to
be included for any specific platform, depending on the particular use case.
We know that saving memory and resources is often mission-critical for a
project, and the approach used in the PicoTCP is fully oriented to saving
up to the last byte of memory.
\item \textbf{Code Quality} Every single component that is added to the
stack goes through a complete set of validation tests. All the newly
introduced code gets scanned and proof-checked by three separate levels of
quality enforcement. The process related to the validation of the code is
one of the largest task for the engineering team. The top-down design of a
new module has to pass the review of our senior architects, because it has
to comply with the general guidelines. The development of the smaller
components is done in a test-driven fashion, where each function call has
its own unit test. Finally, functional non-regression tests are performed
when the feature development is complete, and all the tests are automatically
scheduled to run several times per day to check for functional regressions.
\item \textbf{Adherence to the standards} The design of the protocols
included in the stack are done following step by step the guidelines
provided by the International Engineering Task Force (IETF) with regards to
inter-networking communication. A strong adherence to the standards guarantees a
good integration with all the existing TCP/IP stacks, when communicating
both with other embedded devices and with the PC/server world.
\item \textbf{Features} Our engineering team is never satisfied until all
the corners of the protocols specifications are covered in the code.
A fully-featured protocol implementation including all those non-mandatory
features means better data-transfer performances, coverage of rare/unique 
network scenarios and topologies and a better integration with all types of
networking hardware devices.
\item \textbf{Transparency} The availability of the source code to the Free
Software community is an important added value for PicoTCP. Our programmers
are proud of the aestethic of their code, and they show it with no
hesitation to the attention of the rest of the world.
PicoTCP constantly receives peer-reviews and constructive comments on the
design and the development choices from the academic world and from several
hundreds of hobbists and professionals who read the code. We strongly
believe that software is not about keeping things secret: whenever one is
convinced by the quality of their work, there is absolutely nothing to hide.
\item \textbf{Simplicity} The APIs provided to access the library
facilities, both from the applications and from the device drivers, are
small and well documented. The goal of such a library must be to facilitate
the integration with the surroundings and minimize the time used to combine
the stack with existing code. The support required to port to a new
architecture is so small that is reduced to a set of macros defined in a
header file specific for the platform.
\end{itemize}


