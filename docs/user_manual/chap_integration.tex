
\section{Requirements and Configuration}

PicoTCP is designed to be portable and versatile. Modules can be activated at
compile-time, or excluded from the compilation in order to reduce the build size,
or to save resources at runtime. This characteristic allows an embedded
application to create different types of appliances, starting from a small
forwarding multi-protocol switch, to fully-featured TCP hosts, supporting
internal applets as well as generic POSIX-compliant socket interfaces.


\section{Supported features}
\begin{itemize}
\item \textbf{Device layer} Facilities for device driver are offered in a simple
					structure and API.
\item \textbf{ARP} The stack can use the "Address Resolution Protocol" to retrieve
					the MAC addresses of other hosts in the network.
\item \textbf{IPv4} The network layer supports the IPv4 protocol. The API provides
					functions for routing, set network adresses to devices, ...
\item \textbf{ICMP} Also the "Internet Control Message Protocol" is implemented. This
					protocol provides the system to send error messages, do a ping, ...
\item \textbf{NAT} The stack supports "Network Address Translation" to hide addresses
					from internal networks to the outside. The API also supports functions
					for port forwarding.
\item \textbf{IGMP} ... BLA ... <<< TODO >>>
\item \textbf{multicasting} ... BLA ... <<< TODO >>> (or with IGMP)
\item \textbf{UDP} The stack can use the "User Datagram Protocol" as a transport protocol
					for connection-less communication between sockets.
\item \textbf{TCP} The stack also supports the connection-oriented "Transport Control Protocol"
					for reliable communications.
\item \textbf{Sockets} The user applications on different host use the socket API to communicate.
					The socket API is POSIX compliant <<< TODO CHECK >>>.
\item \textbf{DNS client} A small DNS client is provided to resolve an IP address for a given name.
					The API supports setting several DNS servers and a small cache.
\item \textbf{DHCP client} A DHCP client can request an IP lease from a DHCP server to set the IP
					adress of the device.
\item \textbf{DHCP server} Also a small DHCP server is included to hand out IP addresses to hosts
					in the network.
\item \textbf{Linux development and test facilities} The stack is developed entirely on a Linux system.
					Several tools are easily available and/or included to develop and test user applications.
					(tun/tap devices, vde, tcp benchmark test, ...)
\end{itemize}

\section{Module dependencies}
(WIP)


\section{Target requirements}
PicoTCP can run on several different hardware architectures, and can be
integrated with virtually any operating system, or within a standalone
application. It is possible to run PicoTCP on big-endian as well as
little-endian CPU configurations. PicoTCP uses gcc-specific tags that may
not be compatible with other compilers. The amount of resources needed
may vary depending on the modules that are compiled-in. However, adapting
to a specific hardware platform or for a particular use may require some
integration effort.

\section{The pico$\_$config.h file}
The configuration for all the modules and their features is located in a
the pico$\_$config.h file. The files has two sections: "user configuration"
and "architecture specific". The first one is used to enable features to
include. Uncommenting a line starting with a \texttt{$\#$define} statement
will enable the feature described in the comment. The pico$\_$config file
is located under the "/include" directory. The second section is used to
include the specific implementation of the system interface required by
PicoTCP to run on the target as described in the next section. As an
example, some architecture implementations are provided:
\begin{itemize}
\item a virtual posix target that allows the stack to be run on top of a
POSIX compliant operating system (like Linux or other unix-like implementations).
\item Arm M3/M4 Cortex board support
\end{itemize} 

The architecture-specific configuration is located in the "/include/arch"
directory.

\subsection{Porting to a target system}
PicoTCP relies on a simple set of system-specific calls that must be
implemented externally from the target. Briefly, the interface needed for
the stack to run is composed by:
\begin{itemize}
\item A mechanism to allocate dynamic memory on the system
\item A stable time-source to update its internal counters
\end{itemize} 

For the memory allocation interface, two symbols have to be defined by the system:
\begin{verbatim}
  void *pico_zalloc(int size) - (memory allocation) 
  void pico_free(void *ptr) - (memory release)
\end{verbatim}

\begin{itemize}
\item \texttt{pico$\_$zalloc} Must allocate an object of the given size size in memory
and set the content of the allocated memory to zero. A pointer to the address 0 will
indicate an allocation failure.
\item \texttt{pico$\_$free} Must must release the memory assigned to the object previously
allocated at the address ptr.
\end{itemize}


For the time keeping, the following objects must be defined by the system:	% MOVE TO TABLE ???
\begin{verbatim}
   static inline unsigned long PICO_TIME(void) - returns current time expressed in seconds
   static inline unsigned long PICO_TIME_MS(void)  - returns current time expressed in milliseconds
   static inline void PICO_IDLE(void) - sleep between two consecutive iterations inside the main protocol loop (e.g. to yield the CPU to some other functionality on the sytem)
\end{verbatim}

Finally, whenever debug informations are needed, the system will have to provide a
\texttt{dbg()} function that accepts the same variadic arguments model as a standard \texttt{printf()}.


\section{Network devices integration}
Every device driver must define its own interface to communicate with the stack.
This interface is accessed via the \texttt{pico$\_$device} structure. Every device implements
an instance of this structure by populating the following mandatory fields:

\begin{itemize}
\item \textbf{overhead} - A positive integer indicating the amount of bytes required by the
device driver to implement its header. This is used whenever a network layer allocates a new
packet to be sent through this device. If a valued is specified here, it will be possible for
the device to seek back in the frame scheduled for sending, and subsequently copy any header
information in front of it. Devices dealing with pure stack frames or subparts of it
(e.g. Ethernet) should have overhead set to 0.

\item The callback \textbf{send} - must be a pointer to a function internally defined in the
device driver module. This function will be called every time a frame must be injected in the
network. The module can implement a generic send function for all the registered devices, as
the device field will be passed as the first argument. The callback prototype is the following:

 \texttt{int (*send)(struct pico$\_$device *self, void *buf, int len);}

If the device can immediately inject the frame at address \texttt{buf} of length \texttt{len},
it returns back to the caller the length of the frame injected. If the device is currently busy,
this function can safely return 0, and the stack will retry the same operation again later.

\item The callback \textbf{poll} - must be a pointer to a function internally defined in the
device driver module. This function will be called periodically by the stack, to request a
synchronization on the incoming frames. The prototype is the following:

  \texttt{int (*poll)(struct pico$\_$device *self, int loop$\_$score);}

The poll function must check if the device is ready to receive frames, and for each frame that
is directed to the stack, it will call the library function \texttt{pico$\_$stack$\_$recv()}.
This function will deliver the received frame to the stack.

The loop$\_$score variable represents the maximum amount of frames that the stack can process
during this call, i.e. the maximum amount of calls to \texttt{pico$\_$stack$\_$recv()} that
can be performed during this iterations. The device driver should loop around the packet
delivery operation and decrease the loop$\_$score by one every time a frame is delivered to
the stack. If during the iteration all the score was used, poll will return 0.

\textbf{NOTE:} The poll function must return \textbf{immediately} and must never block on
hardware-specific operations. If the device is interrupt-driven, the integration will have
to provide a mechanism to defer the reception until the next call back to poll. Calling
\texttt{pico$\_$stack$\_$recv()} is only allowed from inside the \texttt{poll()} callback,
thus a two-halves interface interrupt management design is required, and any memory structure
shared between the two halves must be protected against concurrent access accordingly.

\item The callback \textbf{destroy} - a pointer to a function that deallocates the device
structure itself and frees all the eventual structures allocated by the driver during device
creation.

A device driver will have a simple two-functions library API exported in a header file using
the same name, in the modules directory. The two functions to export will be:

\item A create function, accepting any argument required for the internal device configuration,
that returns a pointer to the newly allocated device. The function must allocate the device and
finally call the library function \texttt{pico$\_$device$\_$init()} in order to register the
device into the stack.
The \texttt{pico$\_$device$\_$init()} function accepts the following arguments:

\begin{itemize}
\item the device allocated just before
\item a null-terminated string containing a unique device name for the device to be inserted in
the system (e.g. "eth0")
\item a pointer to an Ethernet address in the form of a previously allocated \texttt{pico$\_$ethdev}
structure, containing the hardware address to be used by the stack for datalink addressing.
If no hardware-specific address is provided to \texttt{pico$\_$device$\_$init()} is provided
(i.e. a \texttt{NULL} pointer is passed), the newly created device will be directly attached
to the network layer and it will have to provide and process valid IP packets with no further
encapsulation.
\end{itemize}

\item A destroy routine, accepting the previously allocated device pointer to free all the associated structures.

\end{itemize}

The way to expand the device driver interface is by simply creating a new specific structure
that contains it, so it inherits all the capabilities of the standard structure but also holds
the required hardware-specific information. The three callbacks will always receive a pointer
to the beginning of the \texttt{pico$\_$device} structure, but the memory area that follows
the structure can be used to keep track of the device hardware-specific context.

Naming convention must be followed for the two functions exposed to the user interface to
create and destroy the device. The functions must be named \texttt{pico$\_$X$\_$create()} and
\texttt{pico$\_$X$\_$destroy()}, where X is the unique name of the device driver.

As an example of a very simple device driver, directly attached to the networking layer using
the valid naming convention for the send/poll/create/destroy interface is contained in the source
file "modules/pico$\_$dev$\_$null.c" and its header "modules/pico$\_$dev$\_$null.h".