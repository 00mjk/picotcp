\section{Requirements and Configuration}

PicoTCP is designed to be portable and versatile. Modules can be activated at
compile-time, or excluded from the compilation in order to reduce the build size
or save resources at runtime. This characteristic allows an embedded
application to create different types of appliances, starting from a small
forwarding multi-protocol switch, to fully-featured TCP hosts, supporting
internal applets as well as generic POSIX-compliant socket interfaces.


\section{Supported features}
\begin{itemize}
\item \textbf{Device layer} Facilities for device driver are offered in a simple
					structure and API.
\item \textbf{ARP} The stack can use the "Address Resolution Protocol" to retrieve
					the MAC addresses of other hosts in the network.
\item \textbf{IPv4} The network layer supports the IPv4 network layer
          protocol. An API is provided in order to access all the
          addressing and routing related functionalities.
\item \textbf{ICMP} Also the "Internet Control Message Protocol" is implemented. This
					protocol provides the system to send error messages, do a ping, ...
\item \textbf{NAT} The stack supports "Network Address Translation" to hide addresses
					from internal networks to the outside. The API also supports functions
					for port forwarding.
\item \textbf{multicast sockets} The stack supports multicast (one-to-many)
          sockets and addresses in order to send and receive data to/from
          multicast groups.
\item \textbf{IGMP} As an integration for the multicast features above, IGMP version 2
          is supported to manage the membership to multicast groups.
\item \textbf{UDP} The stack can use the "User Datagram Protocol" as a transport protocol
					for connection-less communication between sockets.
\item \textbf{TCP} The stack supports the connection-oriented "Transport Control Protocol"
					for reliable communications. The TCP implementation is fully
          featured and the most commonly used extensions are included.
\item \textbf{Sockets} The user applications on different host use the socket API to communicate.
					The socket API is based on the latest POSIX (1-2008)
          specifications, while not being fully compliant due to the fact
          that it is designed to run in a single threading unit. Blocking
          functionalities are reproduced via callback triggering as
          described in the socket API documentation.
\item \textbf{DNS client} A small DNS client is provided to resolve an IP address for a given name.
					The API supports setting several DNS servers and a small cache.
\item \textbf{MDNS client} picoTCP has a mDNS responder on which records can be registered and resolved without the need of a centralised DNS-server. Supports possible caching of records and defending of hostnames.
\item \textbf{DNS-SD client} The stack can register services on the network via Multicast DNS, which allows Zero Configuration Networking capabilities.
\item \textbf{SNTP client} The stack supports synchronizing time over the network using sntp with a precision of at least 10msec. Similar to unix a gettimeofday function is implemented.
\item \textbf{DHCP client} A DHCP client can request an IP lease from a DHCP server to set the IP
					adress of the device.
\item \textbf{DHCP server} Also a small DHCP server is included to hand out IP addresses to hosts
					in the network.
\item \textbf{Linux development and test facilities} The stack is developed entirely on a Linux system.
					Several tools are easily available and/or included to develop and test user applications.
					(tun/tap devices, vde, tcp benchmark test, ...)
\end{itemize}

\section{Enabling modules}
Each module, option and feature included in the code base must be explicitly
enabled by defining a specific PICO$\_$SUPPORT$\_$ preprocessor variable.
If the default Makefile is used to compile PicoTCP, this can be done using
command line options when running make. The syntax required to compile the
protocol in a library (the default Makefile target) is the following:

\texttt{make [MAKE$\_$ARG=VALUE] [...] }

\subsection{Compile-time options}
A few compile-time options can be specified using the command line
arguments of make to modify the result of the build. Global options that
affect the build are the following:

\begin{longtable}{ | l | l | l | p{5cm} | }
\hline
{\bf Argument} &
{\bf Possible values} &
{\bf Default value} &
{\bf Description} 
\\ \hline

DEBUG&
0,1&
1&
When enabled (=1), the resulting library will contain debug symbols. The
size of the library will be much larger than the production build, but it
will be possible to run the stack into a debugger to inspect its behaviour.
When the option is disabled (=0), the library will be optimized for size in
flash, resulting in a smaller binary to be used in production. \\ \hline

PREFIX&
any valid path&
./build&
The target directory where the library and all the objects will be placed
after the compilation. \\ \hline

ENDIAN&
little, big &
little&
Force to build against little-endian or big-endian architecture. \\ \hline

CROSS$\_$COMPILE&
compiler prefix&
&
Use a cross compile prefix when calling the binaries needed to build.
\\ \hline

TCP&
0,1&
1&
Enables the support for Transmission Control Protocol by allowing the usage 
of stream sockets.
\\ \hline

UDP&
0,1&
1&
Enables the support for User Datagram Protocol by allowing the usage 
of datagram sockets.
\\ \hline

IPV4&
0,1&
1&
Enables the support for basic IP networking functionalities. At least one network protocol
is required for most of the features to work, as all types of sockets depend
on the networking layer.
\\ \hline

NAT&
0,1&
1&
Activates the support for network address translation to IPv4.
\\ \hline

ICMP4&
0,1&
1&
Enables the support for control messages over IPv4, (not including the ping functionalities).
\\ \hline

MCAST&
0,1&
1&
If enabled, the support for multicast sockets will be included in the resulting library.
\\ \hline

DEVLOOP&
0,1&
1&
If enabled, a loopback device will be added to the stack, and can be configured to run local traffic.
\\ \hline

PING&
0,1&
1&
When activated, the ping API will be available to test whether the hosts on the network are reachable.
Requires ICMP4 support.
\\ \hline

CRC&
0,1&
0&
If enabled, CRC values are validated at IP and transport layer.
\\ \hline

IPV4FRAG&
0,1&
1&
Enables the support for fragmentation and reassembly of IPV4 packets.
\\ \hline

IPV6FRAG&
0,1&
1&
Enables the support for fragmentation and reassembly of IPV6 packets.
\\ \hline

IPFILTER&
0,1&
1&
If enabled, provides basic filtering.
\\ \hline

DNS$\_$CLIENT&
0,1&
1&
This feature is required to resolve host names into IP addresses and vice-versa.
\\ \hline

MDNS&
0,1&
1&
If enabled, registering and resolving DNS records on the network via Multicast DNS is possible.
\\ \hline

DNS$\_$SD&
0,1&
1&
If enabled, it is possible to register services on the network via Multicast DNS.
\\ \hline

SNTP$\_$CLIENT&
0,1&
1&
Enables snychronising the local time to a given ntp server.
\\ \hline

DHCP$\_$CLIENT&
0,1&
1&
When activated, it will be possible to get the IP address for network devices automatically, when a DHCP
server is present on the network.
\\ \hline

DHCP$\_$SERVER&
0,1&
1&
If activated, it will be possible to run a small DHCP server to provide addresses for automatic configuration
to the other hosts in the network.
\\ \hline

HTTP$\_$CLIENT&
0,1&
1&
Activates a basic HTTP client.
\\ \hline

HTTP$\_$SERVER&
0,1&
1&
Activates a basic HTTP server.
\\ \hline

\end{longtable}

\subsection{Architecture support}
By default, the stack will be compiled to run in a process on a POSIX
system, e.g. to be linked to a Linux application. To change this behavior
and produce a library linked to a specific board-support package (BSP)
among those supported, it is sufficient to set the command line argument
variable ARCH to a specific value. The architectures supported by the stack
are the following:


\begin{longtable}{ | l | l | p{5cm} | }
\hline
{\bf ARCH keyword} &
{\bf CPU} &
{\bf Reference hardware}
\\ \hline

stm32&
ARM Cortex M4-F&
ST Microelectronics evaluation board "STM32f4 Discovery"
\\ \hline

stm32f1xx&
ARM Cortex M3&
muRata SN820X
\\ \hline

stellaris&
ARM Cortex LM3S-6965&
Texas Instrument Evaluation Kit "Codesourcery LM3S6965 ETH"
\\ \hline

lpc18xx&
ARM Cortex M3&
NXP LPC1837 Xplorer board
\\ \hline

lpc43xx&
ARM Cortex M4/M0&
Hitex Evaluation Board LPC4357 
\\ \hline

msp430&
Texas Instruments MSP430&
Texas Instruments EZ430-Chronos
\\ \hline

pic24&
Microchip PIC24FJ256GA106&
openPicus flyportPRO
\\ \hline

atmega128&
Atmel ATmega128&
Ethernut 2
\\ \hline

\end{longtable}



\section{Target requirements}
PicoTCP can run on several different hardware architectures and can be
integrated with virtually any operating system or within a standalone
application. It is possible to run PicoTCP on big-endian as well as
little-endian CPU configurations. PicoTCP uses gcc-specific tags that may
not be compatible with other compilers. The amount of resources needed
may vary depending on the modules that are compiled-in. However, adapting
to a specific hardware platform or for a particular use may require some
integration effort.

\subsection{Porting PicoTCP to a target system}

\begin{center}
\textbf{Warning: ensure that the Board Support Package provided by your
hardware supplier is distributed under the terms of a license compatible
with the PicoTCP license, described in the Appendix of this document.}
\end{center}

PicoTCP relies on a simple set of system-specific calls that must be
implemented externally from the target. Briefly, the interface needed for
the stack to run is composed by:
\begin{itemize}
\item A mechanism to allocate dynamic memory on the system
\item A stable time-source to update its internal counters
\end{itemize} 

For the memory allocation interface, two symbols have to be defined by the system:
\begin{verbatim}
  void *pico_zalloc(int size) - (memory allocation) 
  void pico_free(void *ptr) - (memory release)
\end{verbatim}

\begin{itemize}
\item \texttt{pico$\_$zalloc} Must allocate an object of the given size size in memory
and set the content of the allocated memory to zero. A pointer to the address 0 will
indicate an allocation failure.
\item \texttt{pico$\_$free} Must release the memory assigned to the object previously
allocated at the address ptr.
\end{itemize}


For the time keeping, the following objects must be defined by the system:
\begin{itemize}%[noitemsep]
\item
%\begin{verbatim}
\texttt{static inline unsigned long PICO$\_$TIME(void)}\\
%\end{verbatim}
Returns current time expressed in seconds

\item
%\begin{verbatim}
\texttt{static inline unsigned long PICO$\_$TIME$\_$MS(void)}\\
%\end{verbatim}
Returns current time expressed in milliseconds

\item
%\begin{verbatim}
\texttt{static inline void PICO$\_$IDLE(void)}\\
%\end{verbatim}
Sleep between two consecutive iterations inside the main protocol loop
(e.g. to yield the CPU to some other functionality on the sytem)
\end{itemize}

As an alternative to defining the time-keeping procedure in the asynchronous
functions \texttt{PICO$\_$TIME()} and \texttt{PICO$\_$TIME$\_$MS()}, it is possible
to use an interrupt handler linked to a fixed interval time source, increasing the
volatile global variable \texttt{pico$\_$tick}. If done this way, the two
functions may return the values of \texttt{(pico$\_$tick / 1000)} and
\texttt{pico$\_$tick}, respectively.

Finally, whenever debug information is needed, the system will have to provide a
\texttt{dbg()} function that accepts the same variadic arguments model as a standard \texttt{printf()}.


\subsection{Defining a new architecture support}
If all the above requirements are satisfied, PicoTCP expects those
functions to be mapped to existing code in the BSP of the architecture.
An easy way to do so is by means of a new architecture-specific header
file under the \texttt{include/arch} subdirectory.
Since all the functions above must already be implemented outside the
PicoTCP tree, the library will have to be linked to the system support
library, either during compilation or at at a subsequent stage when the resulting
firmware is being generated. For this reason, a prototype of all the
functions used to implement the functionalities requested by the BSP
must be included from the architecture support header file or incorporated
into the file itself.

For instance, if the BSP for an architecture called "foobar" provides the 
following functions:
\begin{verbatim}
  void *custom_allocate_and_zero(int size);
  void *custom_free(void *mem);
  int print_serial_debug(...);
\end{verbatim}
and an interrupt handler is attached to a time source in order to increment
the \texttt{pico$\_$tick} variable every millisecond, a possible architecture-specific 
file (under \texttt{arch/pico$\_$foobar.h}) should look like the following:

\begin{verbatim}
  /* repeat the prototypes used */
  extern void *custom_allocate_and_zero(int size);
  extern void *custom_free(void *mem);
  extern int print_serial_debug(...);

  #define dbg print_serial_debug
  #define pico_zalloc(x) custom_allocate_and_zero(x)
  #define pico_free(x) custom_free(x)

  static inline unsigned long PICO_TIME(void)
  {
    return pico_tick / 1000;
  }

  static inline unsigned long PICO_TIME_MS(void)
  {
    return pico_tick;
  }

  static inline void PICO_IDLE(void)
  {
    unsigned long tick_now = pico_tick;
    while(tick_now == pico_tick);
  }

\end{verbatim}

Once the architecture-specific file is created, it is time to add the
architecture-specific support to the \texttt{pico$\_$config.h} file, the
same way it is done for the existing architectures, using an additional 
preprocessor elif block:

\begin{verbatim}
#elif defined FOOBAR
#include "arch/pico_foobar.h"
\end{verbatim}

From this point on, it is sufficient to define a preprocessor variable with
the keyword chosen for the architecture, all in capitals (\texttt{FOOBAR} in 
this example case). The final step is to create a block in the main PicoTCP 
makefile that also sets the compiler flags needed to produce objects that are
compatible with and/or optimized for the foobar architecture. Additionally,
this block also contains the definition of the keyword preprocessor macro in
order to have the correct arch-specific header included:

\begin{verbatim}
ifeq ($(ARCH),foobar)
  CFLAGS+=-mcustom-foobar-code -DFOOBAR
endif
\end{verbatim}

To compile for the foobar architecture, it is now sufficient to run

\begin{verbatim}
  make ARCH=foobar
\end{verbatim}


\section{Network devices integration}
Every device driver must define its own interface to communicate with the stack.
This interface is accessed via the \texttt{pico$\_$device} structure. Every device implements
an instance of this structure by populating the following mandatory fields:

\begin{itemize}
\item \textbf{overhead} - A positive integer indicating the amount of bytes required by the
device driver to implement its header. This is used whenever a network layer allocates a new
packet to be sent through this device. If a value is specified here, it will be possible for
the device to seek back in the frame scheduled for sending, and subsequently copy any header
information in front of it. Devices dealing with pure stack frames or subparts of it
(e.g. Ethernet) should have overhead set to 0.

\item The callback \textbf{send} - must be a pointer to a function internally defined in the
device driver module. This function will be called every time a frame must be injected in the
network. The module can implement a generic \textbf{send} function for all the registered devices, as
the device field will be passed as the first argument. The callback prototype is the following:

 \texttt{int (*send)(struct pico$\_$device *self, void *buf, int len);}

If the device can immediately inject the frame at address \texttt{buf} of length \texttt{len},
it returns back to the caller the length of the frame injected. If the device is currently busy,
this function can safely return 0, and the stack will retry the same operation again later.

\item The callback \textbf{poll} - must be a pointer to a function internally defined in the
device driver module. This function will be called periodically by the stack, to request a
synchronization on the incoming frames. The prototype is the following:

  \texttt{int (*poll)(struct pico$\_$device *self, int loop$\_$score);}

The poll function must check if the device is ready to receive frames, and for each frame that
is directed to the stack, it will call the library function \texttt{pico$\_$stack$\_$recv()}.
This function will deliver the received frame to the stack.

The \texttt{loop$\_$score} variable represents the maximum amount of frames that the stack can process
during this call, i.e. the maximum amount of calls to \texttt{pico$\_$stack$\_$recv()} that
can be performed during this iterations. The device driver should loop around the packet
delivery operation and decrease the loop$\_$score by one every time a frame is delivered to
the stack. If during the iteration all the score was used, poll will return 0.

\textbf{NOTE:} The poll function must return \textbf{immediately} and must never block on
hardware-specific operations. If the device is interrupt-driven, the integration will have
to provide a mechanism to defer the reception until the next call back to poll. Calling
\texttt{pico$\_$stack$\_$recv()} is only allowed from inside the \texttt{poll()} callback,
thus a two-halves interface interrupt management design is required, and any memory structure
shared between the two halves must be protected against concurrent access accordingly.

\item The callback \textbf{destroy} - a pointer to a function that deallocates the device
structure itself and frees all the structures that were possibly allocated by the driver 
during device creation.
\end{itemize}

A device driver will have a simple two-functions library API exported in a header file using
the same name, in the modules directory. The two functions to export will be:

\begin{itemize}
\item A \textbf{create} function, accepting any argument required for the internal device configuration,
that returns a pointer to the newly allocated device. The function must allocate the device and
finally call the library function \texttt{pico$\_$device$\_$init()} in order to register the
device into the stack.
The \texttt{pico$\_$device$\_$init()} function accepts the following arguments:

\begin{itemize}
\item the device allocated just before
\item a null-terminated string containing a unique device name for the device to be inserted in
the system (e.g. "eth0")
\item a pointer to an Ethernet address in the form of a previously allocated \texttt{pico$\_$ethdev}
structure, containing the hardware address to be used by the stack for datalink addressing.
If no hardware-specific address is provided to \texttt{pico$\_$device$\_$init()} is provided
(i.e. a \texttt{NULL} pointer is passed), the newly created device will be directly attached
to the network layer and it will have to provide and process valid IP packets without further
encapsulation.
\end{itemize}

\item A destroy routine, accepting the previously allocated device pointer to free all the associated structures.

\end{itemize}

The way to expand the device driver interface is by simply creating a new specific structure
that contains it and thus inherits all the capabilities of the standard structure but also holds
the required hardware-specific information. The three callbacks will always receive a pointer
to the beginning of the \texttt{pico$\_$device} structure, but the memory area that follows
the structure can be used to keep track of the device hardware-specific context.

Naming conventions must be followed for the two functions exposed to the user interface to
create and destroy the device. The functions must be named \texttt{pico$\_$X$\_$create()} and
\texttt{pico$\_$X$\_$destroy()}, where X is the unique name of the device driver.

As an example of a very simple device driver, directly attached to the networking layer using
the valid naming convention for the \textbf{send/poll/create/destroy} interfaces are contained in the source
file \texttt{modules/pico$\_$dev$\_$null.c} and its header \texttt{modules/pico$\_$dev$\_$null.h}.
