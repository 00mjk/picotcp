\section{ZMQ calls} 
% Short description/overview of module functions
The zmq library has been implemented in a slightly different way because it is not
possible to use blocking calls.

\subsection{zmq$\_$socket}

\subsubsection*{Description}
This function creates a new zmq socket of a particular type.

\subsubsection*{Function prototype}
\begin{verbatim}
void *zmq_socket(void* context, int type);
\end{verbatim}

\subsubsection*{Parameters}
\begin{itemize}[noitemsep]
\item \texttt{context} - Context of the socket
\item \texttt{type} - Possible types at this moment:
\begin{itemize}[noitemsep]
\item \texttt{ZMQ$\_$PUB}
\end{itemize}
\end{itemize}

\subsubsection*{Possible events for sockets}
\begin{itemize}[noitemsep]
\item \texttt{PICO$\_$SOCK$\_$EV$\_$RD} - bla bla.
\end{itemize}

\subsubsection*{Return value}
On success, this call returns a pointer to the declared socket (\texttt{struct pico$\_$socket *}).
On error the socket is not created, \texttt{NULL} is returned, and \texttt{pico$\_$err} is set appropriately.

\subsubsection*{Errors}
\begin{itemize}[noitemsep]
\item \texttt{PICO$\_$ERR$\_$EINVAL} - invalid argument
\item \texttt{PICO$\_$ERR$\_$EPROTONOSUPPORT} - protocol not supported
\item \texttt{PICO$\_$ERR$\_$ENETUNREACH} - network unreachable 
\end{itemize}

\subsubsection*{Example}
\begin{verbatim}
sk_tcp = pico_socket_open(PICO_PROTO_IPV4, PICO_PROTO_TCP, &wakeup);
\end{verbatim}


\subsection{pico$\_$socket$\_$read}

