\section{ZeroMQ}

% Description
ZeroMQ is a Messaging protocol for many applications. In this version of the API, only Publish/Subscribe paradigm is s
supported.

To create and use a ZeroMQ Publisher, refer to:
\begin{verbatim}
ZMQ zmq_publisher(uint16_t port, void (*cb)(ZMQ z))
int zmq_send(ZMQ z, char *txt, int len)
\end{verbatim}

To create a subscriber, connect and receive messages, use:
\begin{verbatim}
ZMQ zmq_subscriber(void (*cb)(ZMQ z))
int zmq_connect(ZMQ z, char *address, uint16_t port)
int zmq_recv(ZMQ z, char *txt)
\end{verbatim}

To terminate a ZMQ object, use:
\begin{verbatim}
void zmq_close(ZMQ z)
\end{verbatim}

\subsection{zmq\_publisher}
\subsubsection*{Description}
Create a publisher ZeroMQ object, and bind it to a specific local TCP port.

\subsubsection*{Function prototype}
\begin{verbatim}
ZMQ zmq_publisher(uint16_t port, void (*cb)(ZMQ z))
\end{verbatim}

\subsubsection*{Parameters}
\begin{itemize}[noitemsep]
\item \texttt{port} - the local port the publisher will be bound to. Subscribers in the network will specify this port when connecting to ths publisher.
\item \texttt{cb} - callback indicating READY state: will be called by the stack when all the connected subscribers are ready to receive the next message. If the application is interested in this kind of event, it must provide a callback accepting a ZMQ object with this argument.
\end{itemize}

\subsubsection*{Errors}
In case of failure, NULL is returned, and the value of pico$\_$err
is set as follows:

\begin{itemize}[noitemsep]
\item \texttt{PICO$\_$ERR$\_$EINVAL}          - Invalid argument provided
\item \texttt{PICO$\_$ERR$\_$EFAULT}          - Internal error
\item \texttt{PICO$\_$ERR$\_$ENOMEM}          - No memory available to allocate the object
\end{itemize}

%\subsubsection*{Example}

\subsection{zmq\_send}
\subsubsection*{Description}
Send a message using a previously created ZMQ Publisher object.

\subsubsection*{Function prototype}
\begin{verbatim}
int zmq_send(ZMQ z, char *txt, int len)
\end{verbatim}

\subsubsection*{Parameters}
\begin{itemize}[noitemsep]
\item \texttt{z} - the ZMQ publisher object instance, previously created via \texttt{zmq\_publisher} 
\item \texttt{txt} - the message content to be delivered to registered subscribers 
\item \texttt{len} - the lenght of the message \texttt{txt} 
\end{itemize}

\subsubsection*{Errors}
This function returns the number of copies of the messages that were delivered. When the ZMQ publisher object is in its ready state, this should be the number of registered subscribers at the time of the call.
Zero will be returned if no subscribers are registered, or if the sockets are busy.


\subsection{zmq\_subscriber}
\subsubsection*{Description}
Create a subscriber ZeroMQ object, and bind it to a specific local TCP port.

\subsubsection*{Function prototype}
\begin{verbatim}
ZMQ zmq_subscriber(void (*cb)(ZMQ z))
\end{verbatim}

\subsubsection*{Parameters}
\begin{itemize}[noitemsep]
\item \texttt{cb} - callback indicating a new incoming message: will be called by the stack when at least one of the connected publishers has delivered a new message. If the application is interested in this kind of event, it must provide a callback accepting a ZMQ object with this argument.
\end{itemize}

\subsubsection*{Errors}
In case of failure, NULL is returned, and the value of pico$\_$err
is set as follows:

\begin{itemize}[noitemsep]
\item \texttt{PICO$\_$ERR$\_$EFAULT}          - Internal error
\item \texttt{PICO$\_$ERR$\_$ENOMEM}          - No memory available to allocate the object
\end{itemize}

%\subsubsection*{Example}

\subsection{zmq\_connect}
\subsubsection*{Description}
Connect a ZMQ Subscriber object previously created with the \texttt{zmq\_subscriber} call to a remote publisher at the given TCP address.

\subsubsection*{Function prototype}
\begin{verbatim}
int zmq_connect(ZMQ z, char *address, uint16_t port)
\end{verbatim}

\subsubsection*{Parameters}
\begin{itemize}[noitemsep]
\item \texttt{z} - the ZMQ publisher object instance, previously created via \texttt{zmq\_subscriber} 
\item \texttt{address} - the IP address or hostname of the remote ZeroMQ Publisher 
\item \texttt{port} - the TCP port number of the remote Publisher
\end{itemize}

\subsubsection*{Errors}
This function returns 0 on success, or -1 on failure. \texttt{pico\_err} is set accordingly from the underlying \texttt{pico\_socket\_connect} call. Additionally, \texttt{pico\_err} is set to \texttt{PICO\_ERR\_ENOMEM} when there is not enough memory availabl to add a connector to the ZMQ object.

\subsection{zmq\_recv}
\subsubsection*{Description}
Receive a message using the ZMQ Subscriber object previously created with the \texttt{zmq\_subscriber}, and connected to one or more publishers using \texttt{zmq\_connect}. 

\subsubsection*{Function prototype}
\begin{verbatim}
int zmq_recv(ZMQ z, char *msg)
\end{verbatim}

\subsubsection*{Parameters}
\begin{itemize}[noitemsep]
\item \texttt{z} - the ZMQ publisher object instance, previously created via \texttt{zmq\_subscriber} 
\item \texttt{msg} - the text of the message received, if any.
\end{itemize}

\subsubsection*{Errors}
On success, return the lenght of the received message. This function assumes that the buffer provided by the caller is big enough to contain a whole message (i.e. at least 256 Bytes).
If there is no data to read, or the ZMQ object is disconnected from the Publisher, or not ready, this function returns 0.
In case of failure, -1 is returned, and the value of pico$\_$err
is set accordingly.

\subsection{zmq\_close}
\subsubsection*{Description}
Close and destroy an instance of ZMQ object previously created using \texttt{zmq\_subscriber} or \texttt{zmq\_publisher}. All connected instances related to the object are terminated. No other API calls shall be valid on the same object after this call.

\subsubsection*{Function prototype}
\begin{verbatim}
void zmq_close(ZMQ z)
\end{verbatim}

\subsubsection*{Parameters}
\begin{itemize}[noitemsep]
\item \texttt{z} - the ZMQ publisher object instance, previously created via \texttt{zmq\_subscriber} or \texttt{zmq\_publisher}
\end{itemize}

\subsubsection*{Errors}
This routine never fails. Invalid argument is ignored.
